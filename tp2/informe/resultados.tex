\section{Resultados}

Conociendo los valores iniciales de la se\~nal limpia, el ruido proporcionado y
habiendo obtenido el resultado luego de aplicar los filtros, resulta muy
sencilla una primera aproximaci\'on a los resultados. La idea del experimento es
intentar que la se\~nal recuperada se parezca lo m\'as posible a la original
(sin ruido).

Fue simple, entonces, observar la diferencia entre la se\~nal recuperada y
la original de manera gr\'afica, ya sea a trav\'es de ploteos de se\~nales de 1
\'o 2 dimensiones como de la fuente misma como son las im\'agenes en 2
dimensiones.

Acompa\~nando esto por m\'etodos anal\'iticos como el PSNR (Relaci\'on Se\~nal a
Ruido de Pico) fue posible analizar que filtros resultan mejores que otros y en
que casos es mejor aplicar cada uno.

\subsection{Primeros resultados}

Experimentalmente pudimos observar que el filtro cero es bueno, pero el filtro 
exponencial y el promediador son bastante mejores en todos los casos. 
Para poder obtener un mejor filtrado del ruido, decidimos aplicar primero el 
filtro exponencial y luego el promediador, lo que result\'o en un filtrado muy 
bueno dando resultados muy parecidos a la se\~nal original. 

Los adjetivos utilizados se apoyan tanto en el campo sensorial (en las
im\'agenes y gr\'aficos) como en la medida deL PSNR. 

\subsection{Se\~nales en una dimensi\'on}

\subsubsection{Filtro Cero}

El primer filtro analizado fue el m\'as sencillo de implementar, el filtro cero.
Como mencionamos anteriormente, este filtro utiliza un umbral y env\'ia los
puntos que lo sobrepasan (tanto por arriba como por debajo) a cero.

\begin{figure}[H]
\begin {center}
\includegraphics[width=299pt]{imagenes/dopp1024-sin100-zero-spec.png}
\end {center}
\caption{Se\~nal doppler provista por la c\'atedra, transformada, gr\'aficada
luego de aplicar ruido senoidal utilizando el multiplicador 1000 (rojo) y 
habiendo aplicado el filtro cero (azul)}
\label{fig:Dopp1024zero}
\end{figure}


\subsubsection{Filtro Exponencial}



\begin{figure}[H]
\begin {center}
\includegraphics[width=299pt]{imagenes/g450-sin100-exp-spec.png}
\end {center}
\caption{Se\~nal \texttt{g450} provista por la c\'atedra, transformada, gr\'aficada
luego de aplicar ruido senoidal utilizando el multiplicador 1000 (rojo) y 
habiendo aplicado el filtro exponencial (azul)}
\label{fig:GexpSpec}
\end{figure}

\begin{figure}[H]
\begin {center}
\includegraphics[width=299pt]{imagenes/s64-sin100-exp.png}
\end {center}
\caption{Se\~nal \texttt{s64} provista por la c\'atedra, no transformada, gr\'aficada
luego de aplicar ruido senoidal utilizando el multiplicador 1000 (rojo) y 
habiendo aplicado el filtro promedio (azul)}
\label{fig:SexpSig}
\end{figure}


\subsubsection{Filtro Promedio}


\begin{figure}[H]
\begin {center}
\includegraphics[width=299pt]{imagenes/dopp512-sin100-avg-spec.png}
\end {center}
\caption{Se\~nal \texttt{doppler512} provista por la c\'atedra, transformada, gr\'aficada
luego de aplicar ruido senoidal utilizando el multiplicador 1000 (rojo) y 
habiendo aplicado el filtro promedio (azul)}
\label{fig:SinProm}
\end{figure}

\subsubsection{M\'ultiples filtros}

\begin{figure}[H]
\begin {center}
\includegraphics[width=299pt]{imagenes/dopp1024-gauss100-both.png}
\end {center}
\caption{Se\~nal \texttt{doppler1024} provista por la c\'atedra, no transformada, gr\'aficada
luego de aplicar ruido gaussiano utilizando una amplitud de 100 (rojo) y 
habiendo aplicado los filtros promedio y exponencial (verde)}
\label{fig:dopcomb}
\end{figure}

\begin{figure}[H]
\begin {center}
\includegraphics[width=299pt]{imagenes/g450-sin100-both.png}
\end {center}
\caption{Se\~nal \texttt{g450} provista por la c\'atedra, no transformada, gr\'aficada
luego de aplicar ruido senoidal utilizando el multiplicador 1000 (rojo) y 
habiendo aplicado los filtros promedio y exponencial (verde)}
\label{fig:gcomb}
\end{figure}

\begin{figure}[H]
\begin {center}
\includegraphics[width=299pt]{imagenes/s64-gauss100-both-spec.png}
\end {center}
\caption{Se\~nal \texttt{s64} provista por la c\'atedra, transformada, gr\'aficada
luego de aplicar ruido gaussiano utilizando una amplitud de 100 (rojo) y 
habiendo aplicado los filtros promedio y exponencial (verde)}
\label{fig:scomb}
\end{figure}

\subsection{PNSR de 1D}
\subsubsection{Tabla de resultados para ruido de senoidal de multiplicador de 50}

\begin{table}[H]
        \begin{tabular}{|l|llll|}
                \hline
                \textbf{PSNR [dB]} & Se\~nal + 50*Sin & F(Exp) & F(Prom.) & F(Ambos) \\ \hline
                    Dopp512 & 2.8847 & 19.8146 & 9.46651 & 20.9871 \\
                    Dopp1024 & 2.88818 & 19.3163 & 9.45959 & 21.2796 \\
                    s64 & 14.8158 & 7.69053 & 13.1417 & 8.11751 \\
                    Ramp1234 & 11.7625 & 29.7482 & 18.2714 & 28.9325 \\
                    g450 & 13.8037 & 29.786 & 19.7515 & 27.1512 \\ \hline
                    \end{tabular}
                \end{table}

\subsubsection{Tabla de resultados para ruido blanco de varianza 100}

\begin{table}[H]
        \begin{tabular}{|l|llll|}
                \hline
                \textbf{PSNR [dB]} & Se\~nal + Gaussian (0,100) & F(Exp) & F(Prom.) & F(Ambos) \\ \hline
                    Dopp512 & 13.7462 & 12.8369 & 19.1109 & 14.0987 \\
                    Dopp1024 & 13.9075 & 14.0229 & 19.5712 & 17.4598 \\
                    s64 & 26.1171 & 8.39132 & 13.7514 & 26.0954 \\
                    Ramp1234 & 22.7055 & 27.593 & 27.1689 & 28.8198 \\
                    g450 & 24.7124 & 24.764 & 26.6028 & 8.2662 \\ \hline
                    \end{tabular}
                \end{table}


\newpage

        \subsection{Se\~nales en dos dimensiones}

Como fue comentado, durante el desarrollo del experimento se utilizaron
im\'agenes en escala de grises. La imagen tomada como est\'andar y homenaje es 
la de Brian Kernighan.

\begin{figure}[H]
\begin {center}
\includegraphics[width=299pt]{imagenes/brian_kernighan.png}
\end {center}
\caption{Imagen original en escala de grises de Brian Kernighan}
\label{fig:SinProm}
\end{figure}

Habiendo mostrado las diferencias entre la implementaci\'on de los filtros en
una dimensi\'on y habiendo portado el filtrado a trav\'es de la toma de cada
fila de la im\'agen como una se\~nal de una dimensi\'on, procedemos a mostrar
los resultados obtenidos en funci\'on de los ruidos aplicados, a diferencia de
lo realizado con una dimensi\'on, donde el foco est\'a puesto en discriminar los
resultados por tipo de filtro.
\subsubsection{Ruido senoidal con multiplicador de 50}

Este ruido es controlado y es as\'i como tambi\'en los filtros pueden volver a
una imagen muy similar a la original.

\begin{figure}[H]
    \begin{center}

    $\begin{array}{cc}
\includegraphics[width=150pt]{imagenes/kern-sin50-noisy.png}
\includegraphics[width=150pt]{imagenes/kern-sin50-recovered-avg.png}
\end{array}$
    $\begin{array}{cc}
\includegraphics[width=150pt]{imagenes/kern-sin50-recovered-exp.png}
\includegraphics[width=150pt]{imagenes/kern-sin50-recovered.png}
\end{array}$
\caption{\textbf{Arriba Izquierda}: Im\'agen con ruido Senoidal con multiplicador de 50, \textbf{Arriba Derecha}: Recuperada con filtro promedio, \textbf{Abajo Izquierda}: Recuperada con filtro exponencial, \textbf{Abajo Derecha}: Recuperada aplicando filtro exponencial y promedio}
 \end{center}
\label{fig:Sin50All}
\end{figure}

\subsubsection{Ruido senoidal con multiplicador de 1000}

\begin{figure}[H]

    \begin{center}
     $\begin{array}{cc}
\includegraphics[width=150pt]{imagenes/kern-sin1000-noisy.png}
\includegraphics[width=150pt]{imagenes/kern-sin1000-recovered-avg.png}
\end{array}$
    $\begin{array}{cc}
\includegraphics[width=150pt]{imagenes/kern-sin1000-recovered-exp.png}
\includegraphics[width=150pt]{imagenes/kern-sin1000-recovered.png}
\end{array}$
 \caption{\textbf{Arriba Izquierda}: Im\'agen con ruido Senoidal con multiplicador de 1000, \textbf{Arriba Derecha}: Recuperada con filtro promedio, \textbf{Abajo Izquierda}: Recuperada con filtro exponencial, \textbf{Abajo Derecha}: Recuperada aplicando filtro exponencial y promedio}
 \end{center}
\label{fig:Sin1000All}
\end{figure}


\subsubsection{Ruido gaussiano de varianza 100}


\begin{figure}[H]
     \begin{center}
     $\begin{array}{cc}
\includegraphics[width=150pt]{imagenes/kern-gauss100-noisy.png}
\includegraphics[width=150pt]{imagenes/kern-gauss100-recovered-avg.png}
\end{array}$
    $\begin{array}{cc}
\includegraphics[width=150pt]{imagenes/kern-gauss100-recovered-exp.png}
\includegraphics[width=150pt]{imagenes/kern-gauss100-recovered.png}
\end{array}$
\end{center}
 \caption{\textbf{Arriba Izquierda}: Im\'agen con ruido blanco de varianza 100, \textbf{Arriba Derecha}: Recuperada con filtro promedio, \textbf{Abajo Izquierda}: Recuperada con filtro exponencial, \textbf{Abajo Derecha}: Recuperada aplicando filtro exponencial y promedio}
\label{fig:gauss2d}
\end{figure}

