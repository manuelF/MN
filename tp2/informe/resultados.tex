\section{Resultados}

Conociendo los valores iniciales de la se\~nal limpia, el ruido proporcionado y
habiendo obtenido el resultado luego de aplicar los filtros, resulta muy
sencilla una primera aproximaci\'on a los resultados. La idea del experimento es
intentar que la se\~nal recuperada se parezca lo m\'as posible a la original
(sin ruido).

Fue muy simple, entonces, observar la diferencia entre la se\~nal recuperada y
la original de manera gr\'afica, ya sea a trav\'es de ploteos de se\~nales de 1
\'o 2 dimensiones como de la fuente misma como son las im\'agenes en 2
dimensiones.

Acompa\~nando esto por m\'etodos anal\'iticos como el PSNR (Relaci\'on Se\~nal a
Ruido de Pico) fue posible analizar que filtros resultan mejores que otros y en
que casos es mejor aplicar cada uno.

\subsection{Primeros resultados}

Experimentalmente pudimos observar que el filtro cero es bueno, pero el filtro 
exponencial y el promediador son bastante mejores en todos los casos. 
Para poder obtener un mejor filtrado del ruido, decidimos aplicar primero el 
filtro exponencial y luego el promediador, lo que result\'o en un filtrado muy 
bueno dando resultados muy parecidos a la se\~nal original. 

Los adjetivos utilizados se apoyan tanto en el campo sensorial (en las
im\'agenes y gr\'aficos) como en la medida deL PSNR. 

\subsection{Se\~nales en una dimensi\'on}

\subsubsection{Ruido senoidal}

El primer tipo de ruido analizado, y tambi\'en el m\'as sencillo 

\subsection{Se\~nales en dos dimensiones}

