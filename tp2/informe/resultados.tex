\section{Resultados}

Conociendo los valores iniciales de la se\~nal limpia, el ruido proporcionado y
habiendo obtenido el resultado luego de aplicar los filtros, resulta muy
sencilla una primera aproximaci\'on a los resultados. La idea del experimento es
intentar que la se\~nal recuperada se parezca lo m\'as posible a la original
(sin ruido).

Fue muy simple, entonces, observar la diferencia entre la se\~nal recuperada y
la original de manera gr\'afica, ya sea a trav\'es de ploteos de se\~nales de 1
\'o 2 dimensiones como de la fuente misma como son las im\'agenes en 2
dimensiones.

Acompa\~nando esto por m\'etodos anal\'iticos como el PSNR (Relaci\'on Se\~nal a
Ruido de Pico) fue posible analizar que filtros resultan mejores que otros y en
que casos es mejor aplicar cada uno.

\index{Resultados!Nota previa}
\subsection{Nota previa}

