\section{Conclusiones}

La primer conclusi\'on que pudimos sacar de este TP fue que el filtro que apliquemos para detecci\'on y correci\'on
de ruido siempre va a depender del tipo de ruido que querramos eliminar de la se\~nal.

Los ruidos que estudiamos fueron los gaussianos  y salt \& pepper. En todos los casos pudimos comparar los filtros
cero, exponenciales, y de la mediana que desarrollamos contra las sen\~nales originales, y alguno de estos o
la combinaci\'on de los dos ultimos, resultaron razonablemente efectivos a la hora de restaurar la mayor integridad posible.

\subsection{Tipos de ruido}

Durante los experimentos trabajamos con 2 tipos diferentes de ruido. 
El  Ruido Gaussiano representa (del modo m\'as fiel que encontramos) valores aleatorios  que aparecen tipicamente en
un ruido blanco. Es conocido que la forma correcta de modelar este ruido es siguiendo la distribuci\'on normal,
que fue lo que hicimos.
Por su parte el salt \& pepper posee un nivel de incertidumbre (en el sentido de ausencia de patrones)
menor que el gaussiano, y tambien afecta de una manera bastante distinta la sen\~nal, como se pudo ver en las
figuras.

Correr los mismos filtros sobre se\~nales con esos tipos de ruido sirvi\'o como
refuerzo a la intuici\'on de que es m\'as simple lidiar con ruidos que siguen
cierto patr\'on a hacerlo con el denominado ``ruido blanco''.

\subsection{Extensi\'on a 2D}

En se\~nales 1D los filtros que aplicamos reconstruyeron se\~nales que se parecieron mucho m\'as a la se\~nal
original que la se\~nal con ruido aunque en muchos casos la diferencia era notable. Se puede percibir esto en 

las zonas de baja frecuencia de las se\~nales. Peque\~nas perturbaciones en este area, connllevan a grandes 
diferencias en la reconstrucci\'on, por lo cual decidimos intentar evitarlas a la hora de aplicar los filtros.

En se\~nales 2D, en cambio, ya que trabajabamos con im\'agenes, pudimos observar que la imagen con ruido en muchos
casos no se distingu\'ia (es decir, no se pod\'ia reconocer la imagen comparandola con la original), aunque despu\'es
de aplicar los filtros se reconoce una imagen bastante fiel a la original en comparaci\'on con la imagen ruidosa. Igualmente,
los filtros dise\~nados presentan importantes alteraciones a las im\'agenes originales. Esto problablemente sea un subproducto de como
implementamos los filtros por bloques e indica un interesante camino a seguir a la hora de tunear los parametros de los distintos
filtros.

\subsection{Elecci\'on de filtros}

Es importante saber los casos de uso de las funciones para definir el rango de calidad \textit{aceptable}. No es lo mismo
la precisi\'on necesaria para identificar si hay, o no, una cara en la imagen que para un sistema de reconocimiento \'optico
de caracteres. Esto es clave para poder ver si se justifica aplicar filtros m\'as refinados (por ejemplo, multiples pasadas de 
Filtro exponencial, y tal vez intercalarlo con filtro Mediana), donde se pierden un poco de detalles como suavizado
pero se gana en velocidad de identificaci\'on humana. Otros filtros tambien que se pueden considerar pueden ser un Pasa Bajos o 
Pasa Banda, pero habria que tener en claro que los ruidos en la se\~nal siguen cierta distribuci\'on, sino seria lo mismo que ir
tanteando (mecanismo no ideal pero a veces \'util).

\subsection{Conclusion}
En resumen, con este trabajo pudimos comprender un m\'inimo de la importancia de la transformacion de Fourier y similares,
un \'area que ha sido crucial para el desarrollo de las telecomunicaciones alrededor del mundo. Nos ha servido para plantear
el problema desde otro aspecto, como un ``simple'' cambio de base puede revelarnos mucha m\'as informaci\'on de la que 
consideramos existente en una se\~nal. 
