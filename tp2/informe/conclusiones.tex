\section{Conclusiones}

La primer conclusi\'on que pudimos sacar de este TP fue que el filtro que apliquemos para detecci\'on de ruido
siempre va a depender del tipo de ruido que se haya aplicado a la se\~nal.

Los ruidos que aplicamos fueron ruidos gaussianos y ruidos sinoidales. En ambos casos vimos que los filtros
exponencial y promediador fueron los que resultaron m\'as eficientes para los ruidos que aplicamos.

En se\~nales 1D los filtros que aplicamos reconstruyeron se\~nales que se parecieron mucho m\'as a la se\~nal
original que la se\~nal con ruido aunque en muchos casos la diferencia era notable.

En se\~nales 2D, en cambio, ya que trabajabamos con im\'agenes, pudimos observar que la imagen con ruido en muchos
casos no se distingu\'ia (es decir, no se pod\'ia reconocer la imagen comparandola con la original), aunque despu\'es
de aplicar los filtros se reconoce una imagen bastante fiel a la original en comparaci\'on con la imagen ruidosa.
