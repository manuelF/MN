\documentclass[12pt, a4paper]{article}
\usepackage[spanish]{babel}
%\usepackage{a4wide}
\usepackage{amsmath}
\usepackage{amssymb}
\usepackage{amsthm}
\usepackage{color}
\usepackage{hyperref}
\usepackage{makeidx}
\usepackage{float}
\usepackage[margin=1.0in]{geometry}
\usepackage{caratula}
\usepackage{pdfpages}

\usepackage{listings}\usepackage{color}
\usepackage{textcomp}\definecolor{listinggray}{gray}{0.9}

\definecolor{lbcolor}{rgb}{0.9,0.9,0.9}
\usepackage{a4wide}

\lstset{
      language=C++,
      basicstyle=\small\sffamily,
      columns=fullflexible,
      upquote=true,
      extendedchars=true,
      texcl=true,
      mathescape=true,
      showspaces=false
                }
            


\makeindex

\begin{document}

\index{Caratula}
%Estos son los parametros para la caratula
\materia{M\'etodos Num\'ericos}
\submateria{Recuperatorio Trabajo Pr\'actico 2}
\titulo{Filtrado de se\~nales usando DCT}
\fecha{19 de Julio de 2013}
\integrante{Zajdband, Dan}{144/10}{dan.zajdband@gmail.com}
\integrante{Ferreria, Manuel}{199/10}{m.ferreria@gmail.com}
\integrante{Taravilse, Leopoldo}{464/08}{ltaravilse@gmail.com}
\resumen{Estudio comparativo de filtros de se\~nales mediante la utilizaci\'on
de la DCT. An\'alisis del efecto de diversos filtros para distintos casos de
ruido (Senoidal y Blanco). Extensi\'on del problema a dos dimensiones. Estudio
del error de recuperaci\'on de la se\~nal original mediante la utilizaci\'on del
PSNR.}
\kwagregar{DCT, FFT, Fourier, Noise Reduction, Image analysis}

\maketitle

\pagebreak

\tableofcontents

\pagebreak

\index{Introducci\'on te\'orica}
\section{Introducci\'on te\'orica}

Previo al detalle del experimento realizado se detallan una serie de conceptos te\'oricos de necesario
conocimeinto para el entendimiento del trabajo.

\subsection{Reconocimiento \'optico de caracteres (OCR)}

El reconocimiento \'optico de caracteres es el proceso de conversi\'on de caracteres de un formato que puede
ser escritura a mano alzada, im\'agenes de libros u otros formatos complejos de entender para una computadora
a otro que sea reconocible por esta.

Este area combina disciplinas como inteligencia articifial, an\'alisis num\'erico y machine learning. Entre
las aplicaciones de OCR se encuentran:

\begin{itemize}
  \item Digitalizaci\'on de libros y documentos
  \item Reconocimiento de tarjetas de cr\'edito y facturas
  \item Generaci\'on de im\'agenes a partir de documentos digitales
  \item Resoluci\'on autom\'atica de CAPTCHA
\end{itemize}

\subsection{Matriz de covarianza}

Dado un conjunto de datos de $n$ componentes, se define la matriz de covarianza a la matriz de $n*n$ que
contiene en su elemento $(i, j)$ la covarianza entre las componentes $i$ y $j$ de la matriz original.

Siendo la covarianza entre 2 componentes definida como:

$\Sigma_{ij} = \mathrm{cov}(X_i, X_j) = \mathrm{E}\begin{bmatrix} (X_i - \mu_i)(X_j - \mu_j) \end{bmatrix}$

con $\mu_i = \mathrm{E}(X_i)$.

\subsection{Descomposici\'on en valores singulares (SVD)}

La descomposici\'on en valores singulares de una matriz $\mathbf{M} \in \mathbb{R^{mxn}}$ es una factorizaci\'on de la forma $\mathbf{M} =
\mathbf{U} \boldsymbol{\Sigma} \mathbf{V}^T$ donde:

\begin{itemize}
  \item $\mathbf{U} \in \mathbb{R^{mxn}}$ con columnas ortogonales.
  \item $\boldsymbol{\Sigma} \in \mathbb{R^{nxn}}$ es una matriz diagonal con elementos no negativos y los elementos de la
diagonal constituyen los valores singulares de la matriz original.
  \item $\mathbf{V} \in \mathbb{R^{nxn}}$ es una matriz ortogonal.
\end{itemize}

La descomposici\'on en valores singulares tiene diversos usos entre los que se encuentra la resoluci\'on de
sistemas lineales y la b\'usqueda de matrices pseudoinversas.

\subsection{Introducci\'on al trabajo}

El siguiente trabajo consiste en el estudio del mecanismo de reconocimiento autom\'atico
en im\'agenes mediante el an\'alisis de componentes principales. Para simplificar la
experimentaci\'on se utilizan im\'agenes que representan tan solo 3 s\'imbolos distintos
(3, 5 y 8).

Este tipo de an\'alisis proviene del trabajo llevado acabo en [Sirovich89] y [Turk91]. Estos
trabajos reconocen al an\'alisis de componentes principales como herramientas te\'oricas poderosas
a la hora de buscar caracterizaciones autom\'aticas en im\'agenes. Estos papers se enfocan en el
an\'alisis de rostros, una disciplina con un grado de complejidad superior al estudiado en este
trabajo.

La t\'ecnica se basa fundamentalmente en el hecho que las im\'agenes en $\mathbb{R}^{n * m}$ las cuales se
quiere reconocer no son variables aleatorias uniformemente distribuidas, sino que existe una funci\'on caja
negra que genera estas im\'agenes (en nuestro caso, que los digitos se dibujan de maneras parecidas,
como siguiendo un trazo mental).

Como hemos podido observar en el caso de, por ejemplo, la esteganograf\'ia, si a una matriz se le alteran
las componentes principales de menor magnitud, la alteraci\'on en la matriz reconstruida es baja.
Este fen\'omeno nos la pauta de que estas componentes aportan menor cantidad de ``informaci\'on''
que las de mayor magnit\'ud a la hora de representar la im\'agen.

Una conclusi\'on sacada a priori indica que para reconstruir las im\'agenes de la forma m\'as fiel posible,
y consecuentemente perder la menor cantidad de informaci\'on posible, es menester conservar la mayor cantidad
de componentes principales de la im\'agen.

El objetivo principal del trabajo es, dada una im\'agen conteniendo la representaci\'on del d\'igito (3,5,8),
identificar a cual corresponde. La hipotesis asumida a lo largo del trabajo es que las distintas im\'agenes
de un mismo digito van poseen caracter\'isticas ``similares'' en sus componentes principales.
El procedimiento se encarga de buscar, dada una im\'agen cualquiera, y realizar el matching correspondiente
para determinar a que d\'igito parece m\'as y as\'i poder identificarlo.

El procedimiento consta, en t\'erminos generales, de la generaci\'on de una m\'atriz de covarianza surgente
de las im\'agenes de entrenamiento. Esto se obtiene con una matriz $M \in \mathbb{R}^{n * m}$, con $n$ cantidad
de im\'agenes de entrenamiento y $m$ cantidad de pixeles por im\'agen. Luego, se define M como:

$M_i = \frac{(Imagen_i - \mu_{Imagenes})}{\sqrt{n-1}}$

Para obtener la covarianza, se debe obtener $M^t M$, esto resulta en una matriz $M' \in \mathbb{R}^{m * m}$.
De esta matriz $M'$ es posible extraer los autovectores, para poder hacer al descomposici\'on apropiadamente.
Los autovectores de $M'$ tambien se pueden obtener como la matriz $V^t$ de la factorizaci\'on SVD, $A=U\Sigma V^t$.

Una vez obtenida la matriz $V^t$ de autovectores de la covarianza, se procede a uno de los diferentes m\'etodos de
identificaci\'on de im\'agenes.


\pagebreak

\index{Desarrollo}
\section{Desarrollo}
La dificultad del desarrollo de este experimento fue un poco m\'as elevada a la de los dos 
trabajos pr\'acticos anteriores. 

En primer lugar, se decidi\'o hacer primero todo el trabajo pr\'actico en Matlab para obtener 
una pronta aproximaci\'on a los resultados, siendo el desarrollo ampliamente m\'as comodo que 
en C++ (dado que la SVD forma parte de la libreria estandar de Matlab). Una vez implementado 
el TP en Matlab, se procedi\'p a realizar el experimento en C++. El primer problema encontrado 
a la hora de implementar el experimento en Matlab fue que al calcular la SVD con 60.000 imagenes, 
se generaba una matriz de $60.000 \times 60.000$ elementos de tipo double, que no entraba en
memoria, es por eso que se decidi\'o utilizar s\'olo 40.000 imagenes para el experimento en Matlab.

Luego de desarrollar y contar con el experimento funcionando en Matlab, se empez\'o la programaci\'on
del mismo en C++. La primera variaci\'on significativa en la implementaci\'on se refiere al 
descubrimiento de que no es necesario calcular la SVD, sino s\'olo la matriz $V$, resultando 
en el c\'alculo de una matriz de $60.000 \times 784$ en vez de una de $60.000 \times 60.000$.
A\'un as\'i, consideraciones posteriores sugirieron que 60.000 im\'agenes era un n\'umero por 
dem\'as excesivo y por lo tanto el experimento pas\'o operar con s\'olo 5.000 imagenes. 
Al analizar los resultados se corrobor\'o que 5000 im\'agenes conforman una muestra significativa, 
y por lo tanto se realiz\'o el trabajo pr\'actico usando 5.000 imagenes de entrenamiento y 1.000 imagenes de test.

\subsection{Matriz de Covarianza}
El primer paso consisti\'o en generar la matriz de covarianza con las 5.000 imagenes tomadas como datos 
de entrenamiento. Para eso se utiliz\'o la f\'ormula del enunciado. Se gener\'o la matriz $X$ a trav\'es 
de la sustracci\'on a cada entrada de la matriz del promedio del pixel que representa (promedio de ese 
pixel en todas las imagenes) y la posterior divisi\'on la matriz por $\sqrt{n-1}$. Para calcular la
matriz de covarianza, a la que llamamos $Mx$, se realiz\'o el producto $X^tX$.

\subsection{Factorizaci\'on QR}
A la hora de calcular la SVD se present\'o la idea de que s\'olo es necesaria la matriz $V$, es por 
eso que en vez de calcular la SVD, se procedi\'o a calcular la matriz $V$ utilizando la factorizaci\'on 
QR de la matriz de covarianza. El m\'etodo utilizado para calcular la matriz $V$ fue la factorizaci\'on
QR de $Mx$ (la matriz de covarianza) mediante la aplicaci\'on de transformaciones de Householder, y la
posterior multiplicaci\'on RQ (es decir, en el orden inverso), hasta que la suma de los elementos debajo 
de la diagonal principal suman menos de 15. La elecci\'on del 15 no fue caprichosa, sino que es el 
resultado de la corrida del experimento con $10^{-5}$ en lugar de 15, donde converg\'ia a un n\'umero 
entre 14 y 15 por temas de precisi\'on.

Para calcular la matriz $V$, se utiliz\'o la factorizaci\'on de Householder y la se implement\'o en $O(n^3)$ 
siendo $n = 784$. Como el programa resultante tard\'o m\'as de 5 horas en calcular la matriz $V$, se gener\'o 
el archivo V.txt con esta informaci\'on precalculada, si el archivo no exist\'ia previamente entonces se 
genera la matriz y luego se guarda en el archivo, caso contrario, se realiza la lectura del archivo.


\pagebreak


\index{Resultados}
\section{Resultados}

\subsection{Iteraciones del c\'alculo de autovectore}
El mecanismo de reconocimiento de im\'agenes esta basado en los autovectores
de la matriz de covarianza, como detallamos antes. Para obtenerlos, se aplica
el m\'etodo iterativo de la factorizacion QR reiterada, descripto anterioremente.
La condicion de corte de este m\'etodo esta basada en la suma de los elementos bajo
la diagonal. Estudiamos esta suma y la distribucion de los valores bajo la diagonal,
para intentar entender el comportamiento y definir una buena cota. 

\def \hrwidth {500pt}

\begin{figure}[H]
\begin {center}
\includegraphics[width=\hrwidth]{plots/SUM.png}
\end {center}
\caption{Suma de los elementos bajo la diagonal para 5000, 15000 y 30000 im\'agenes
en funcion de la cantidad de iteraciones de QR.}
\label{fig:SUM}
\end{figure}

\begin{figure}[H]
\begin {center}
\includegraphics[width=\hrwidth]{plots/PROM.png}
\end {center}
\caption{Promedio de los elementos bajo la diagonal para 5000, 15000 y 30000 im\'agenes
en funcion de la cantidad de iteraciones de QR.}
\label{fig:PROM}
\end{figure}


\subsection{Promedio de Reconocimiento}
La m\'etodologia de implementaci\'on fue la siguiente: Se construy\'o una
m\'atriz de covarianza utilizando los primeros $n$ im\'agenes del dataset
\texttt{Training Data}. Luego, se tomaban $30000-n$ im\'agenes que se descartaban; para poder tomar 
siempre las mismas $t$ im\'agenes siguientes como im\'agenes de test, con sus 
correspondientes labels. A estos se les aplicaron las t\'ecnicas
de detecci\'on que hemos detallado antes y fuimos variando los $k$ componentes principales 
de las tuplas que tomabamos para hacer las cuentas de distancia.
Los gr\'aficos de HitRate est\'an hechos en funci\'on de $k$.
El Hitrate se define como el porcentaje de aciertos en la detecci\'on sobre el total de 
experimentos realizados.

Tom\'amos siempre las mismas $t=500$ im\'agenes para todos los test, que representan
de manera proporcional a cada d\'igito.

Testeamos usando matrices de covarianza entrenadas con cantidad variable de im\'agenes.
\def \hrwidth {500pt}

\begin{figure}[H]
\begin {center}
\includegraphics[width=\hrwidth]{plots/hitrate-1kcv.png}
\end {center}
\caption{Hitrate de detecci\'on de 500 im\'agenes de acuerdo a los varios m\'etodos implementados
usando la matriz de covarianza entrenada con 1000 im\'agenes}
\label{fig:HR1kcv}
\end{figure}

\begin{figure}[H]
\begin {center}
\includegraphics[width=\hrwidth]{plots/hitrate-5kcv.png}
\end {center}
\caption{Hitrate de detecci\'on de 500 im\'agenes de acuerdo a los varios m\'etodos implementados
usando la matriz de covarianza entrenada con 5000 im\'agenes}
\label{fig:HR5kcv}
\end{figure}

\begin{figure}[H]
\begin {center}
\includegraphics[width=\hrwidth]{plots/hitrate-15kcv.png}
\end {center}
\caption{Hitrate de detecci\'on de 500 im\'agenes de acuerdo a los varios m\'etodos implementados
usando la matriz de covarianza entrenada con 15000 im\'agenes}
\label{fig:HR15kcv}
\end{figure}

\begin{figure}[H]
\begin {center}
\includegraphics[width=\hrwidth]{plots/hitrate-30kcv.png}
\end {center}
\caption{Hitrate de detecci\'on de 500 im\'agenes de acuerdo a los varios m\'etodos implementados
usando la matriz de covarianza entrenada con 30000 im\'agenes}
\label{fig:HR30kcv}
\end{figure}
%%%%%%%%%%%%%%%%%%%%%%%%%%%%%%%%%%%%%%%%%%%%%%%%%%%%%%%%%%%%%%%%

\subsection{Reconocimiento en funcion de iteraciones}
Una vez definido el m\'etodo de detecci\'on, nos interesa ver como varia el hitrate
de acuerdo a la matriz generada con distinta cantidad de iteraciones. En particular
utilizamos el m\'etodo de distancia a los promedios y 100 vecinos m\'as cercanos, todos
medidos usando norma 2.

\begin{figure}[H]
\begin {center}
\includegraphics[width=\hrwidth]{plots/HR-10-avg.png}
\end {center}
\caption{Hitrate en funci\'on de la cantidad de iteraciones en la generacion de la matriz
usando distancia a los promedios tomando 10 componentes principales}
\label{fig:HR10Avg}
\end{figure}


\begin{figure}[H]
\begin {center}
\includegraphics[width=\hrwidth]{plots/HR-100-avg.png}
\end {center}
\caption{Hitrate en funci\'on de la cantidad de iteraciones en la generacion de la matriz
usando distancia a los promedios tomando 100 componentes principales}
\label{fig:HR100Avg}
\end{figure}


\begin{figure}[H]
\begin {center}
\includegraphics[width=\hrwidth]{plots/HR-10-neig.png}
\end {center}
\caption{Hitrate en funci\'on de la cantidad de iteraciones en la generacion de la matriz
usando vecinos m\'as cercanos tomando 10 componentes principales}
\label{fig:HR10Neig}
\end{figure}


\begin{figure}[H]
\begin {center}
\includegraphics[width=\hrwidth]{plots/HR-100-neig.png}
\end {center}
\caption{Hitrate en funci\'on de la cantidad de iteraciones en la generacion de la matriz
usando vecinos m\'as cercanos tomando 100 componentes principales}
\label{fig:HR10Neig}
\end{figure}







%%%%%%%%%%%%%%%%%%%%%%%%%%%%%%%%%%%%%%%%%%%%%%%%%%%%%%%%%%%%%%%%
\subsection{Reconocimiento por d\'igito}
Fijamos la matriz de covarianza entrenada con 30000 im\'agenes, ya que est\'a precalculada.
Siguiendo la misma metodolog\'ia, nos concentramos ahora en el HitRate individual
de cada d\'igito. Analizamos el HitRate de acuerdo a la metodolog\'ia de detecci\'on usada.
\def \pdwidth {500pt}

\begin{figure}[H]
\begin {center}
\includegraphics[width=\pdwidth]{plots/pordig-30kcv-norma2.png}
\end {center}
\caption{Hitrate por d\'igito de detecci\'on de 500 im\'agenes usando la matriz de covarianza entrenada con 30000 im\'agenes
clasificados por distancia a los promedios usando norma 2}
\label{fig:HRD30kcv-n2}
\end{figure}

\begin{figure}[H]
\begin {center}
\includegraphics[width=\pdwidth]{plots/pordig-30kcv-norma1.png}
\end {center}
\caption{Hitrate por d\'igito de detecci\'on de 500 im\'agenes usando la matriz de covarianza entrenada con 30000 im\'agenes
clasificados por distancia a los promedios usando norma 1}
\label{fig:HRD30kcv-n1}
\end{figure}

\begin{figure}[H]
\begin {center}
\includegraphics[width=\pdwidth]{plots/pordig-30kcv-normaInf.png}
\end {center}
\caption{Hitrate por d\'igito de detecci\'on de 500 im\'agenes usando la matriz de covarianza entrenada con 30000 im\'agenes
clasificados por distancia a los promedios usando norma infinito}
\label{fig:HRD30kcv-ninf}
\end{figure}

\begin{figure}[H]
\begin {center}
\includegraphics[width=\pdwidth]{plots/pordig-30kcv-BestOf3.png}
\end {center}
\caption{Hitrate por d\'igito de detecci\'on de 500 im\'agenes usando la matriz de covarianza entrenada con 30000 im\'agenes
clasificados por distancia a los promedios usando la norma que m\'as repetida entre Norma 1, 2 e Infinito}
\label{fig:HRD30kcv-bo3}
\end{figure}

\begin{figure}[H]
\begin {center}
\includegraphics[width=\pdwidth]{plots/pordig-30kcv-top100.png}
\end {center}
\caption{Hitrate por d\'igito de detecci\'on de 500 im\'agenes usando la matriz de covarianza entrenada con 30000 im\'agenes
clasificados por distancia a 100 m\'as cercanos}
\label{fig:HRD30kcv-dist100}
\end{figure}

\section{Heatmap de reconocimiento por d\'igito}
Aca mostramos las equivocaciones y aciertos al reconocer cada d\'igito, variando la cantidad de columnas, hecho con la matriz de covarianza de
30000 im\'agenes y variando la cantidad de columnas tomadas. Eje X el valor detectado, eje Y el valor verdadero. El color marca la cantidad
de veces que se detect\'o un valor y a cual correspondia.
\def \hmwidth {500pt}
\begin{figure}[H]
\includegraphics[width=\hmwidth]{plots/heatmap-30kcv-k5-norma_2.png}
\caption{Heatmap de aciertos por d\'igito de detecci\'on de 500 im\'agenes usando la matriz de covarianza entrenada con 30000 im\'agenes
clasificados por norma 2 tomando K=5}
\label{fig:HM30kcv-k5}
\end{figure}

\begin{figure}[H]
\includegraphics[width=\hmwidth]{plots/heatmap-30kcv-k10-norma_2.png}
\caption{Heatmap de aciertos por d\'igito de detecci\'on de 500 im\'agenes usando la matriz de covarianza entrenada con 30000 im\'agenes
clasificados por norma 2 tomando K=10 }
\label{fig:HM30kcv-k10}
\end{figure}

\begin{figure}[H]
\includegraphics[width=\hmwidth]{plots/heatmap-30kcv-k25-norma_2.png}
\caption{Heatmap de aciertos por d\'igito de detecci\'on de 500 im\'agenes usando la matriz de covarianza entrenada con 30000 im\'agenes
clasificados por norma 2 tomando K=25 }
\label{fig:HM30kcv-k25}
\end{figure}

\begin{figure}[H]
\includegraphics[width=\hmwidth]{plots/heatmap-30kcv-k50-norma_2.png}
\caption{Heatmap de aciertos por d\'igito de detecci\'on de 500 im\'agenes usando la matriz de covarianza entrenada con 30000 im\'agenes
clasificados por norma 2 tomando K=50 }
\label{fig:HM30kcv-k50}
\end{figure}

\begin{figure}[H]
\includegraphics[width=\hmwidth]{plots/heatmap-30kcv-k100-norma_2.png}
\caption{Heatmap de aciertos por d\'igito de detecci\'on de 500 im\'agenes usando la matriz de covarianza entrenada con 30000 im\'agenes
clasificados por norma 2 tomando K=100 }
\label{fig:HM30kcv-k100}
\end{figure}


\pagebreak


\index{Discusi\'on}
\section{Discusion}


\subsection{Clasificaci\'on de los me\'todos empleados}
Los metodos que implementamos fueron b\'asicamente 3. La norma 2 de distancia a
los centros de masas de cada d\'igito, la mejor de las 3 normas (norma 2, norma 1 y norma infinito)
hacia los centros de masas y el m\'as repetido de los 100 vecinos m\'as cercanos al punto.

El m\'etodo de la norma dos, es sorprendentemente bueno. Cuando se superan las 30
columnas mayormente el hitrate estabiliza en un valor razonablmente alto. Si lo
comparamos contra norma 1, es un poco mas alto el hitrate pero no excesivo. Ahora
cuando lo miramos contra norma infinito, vemos que la norma 2 es un m\'etodo
claramente superior. Como la norma infinito se define como el m\'aximo componente,
es de esperar que alguna de las componentes de los puntos transformados tenga mayor
peso que otras, por lo que no es extra\~no ver que aumentando la cantidad de componentes
no varia en nada la detecci\'on.

Viendo que las mediciones de norma 1 y norma infinito a veces discrepaban de lo que
decia norma 2, consideramos utilizarlas tambien para hacer otro m\'etodo, el de \textit{mejor
de 3}. Es decir, si coinciden 2 de las 3 normas, la usamos (por m\'as que coincidan la
infinito y la norma 1). Si no hay ningun match, hacemos fallback en norma 2. En este caso
nos sorprendio que los resultados no mejoraban con respecto al uso \'unico de norma 2.
Esto se puede ver en \ref{fig:HR30kcv} y en \ref{fig:HRD30kcv-bo3}, en comparaci\'on con
el de norma 2 en \ref{fig:HRD30kcv-n2}.

Notablemente, la idea de conseguir el m\'as repetido de los 100 vecinos m\'as cercanos
parece tener sentido, pero experimentalmente no es para nada buena, en varios aspectos.
No provee una mejor detecci\'on, de hecho es consistentemente peor para cualquier $k>15$.
Su runtime ademas es enorme, ya que debe hacer norma 2 para todos los otros puntos que
deseemos considerar, pueden andar en el orden de miles tal vez. En comparacion, el m\'etodo
de la norma 2 anterior solamente comprueba contra 10 puntos (los centros de masas).


\subsection{Tama\~no de los corpus de entrenamiento }
Un punto fundamental del trabajo consistio en el calculo de la matriz de autovectores
de la covarianza de las imagenes. En las secciones anteriores se discutio su construcci\'on
y su calculo. Ac\'a comprobamos la relevancia del tama\~no del corpus usado a la hora
de generarla.

Experimentamos tomando 1000, 5000, 15000 y 30000 imagenes (m\'as no usamos porque era
realmente excesivo el tiempo de c\'alculo. Con m\'as tiempo hubieramos podido paralelizar
y poder correr en un cluster o en otros dispositivos esto.)



\subsection{Detecci\'on por d\'igito}
Calcular el Hitrate por digito se nos ocurri\'o tarde, pero son resultados
sumamente interesantes. Ac\'a podemos apreciar considerablemente que, dentro de
la detecci\'on de d\'igitos, hay problemas m\'as dificiles que otros.

Por un lado, el digito 1 es el m\'as facil de reconocer. Nuestra hip\'otesis
es que no es normal dibujarlo con m\'as de un trazo, y eso simplifica mucho la
descomposici\'on en componentes principales, al tener que capturar unicamente ese
fen\'omeno.

Por el otro, el digito 5 es much\'isimo m\'as complicado de reconocer, teniendo
hitrates de una fracci\'on de los otros digitos, incluso con m\'as componentes
principales. Esto se debe al fenomeno de que es a veces sut\'il la diferencia
entre los d\'igitos 5, 3 y 8.

%%HACER GRAFICO DE HITRATE CONFUNDIDO ENTRE 5 y 3.



\pagebreak

\section{Conclusiones}

La primer conclusi\'on que pudimos sacar de este TP fue que el filtro que apliquemos para detecci\'on y correci\'on
de ruido siempre va a depender del tipo de ruido que querramos eliminar de la se\~nal.

Los ruidos que estudiamos fueron ruidos gaussianos y ruidos senodales. En ambos casos pudimos comparar los filtros
exponenciales y promediadores que desarrollamos contra las sen\~nales originales, y alguno de estos o la combinaci\'on de los dos,
resultaron razonablemente efectivos a la hora de restaurar la mayor integridad posible.

En se\~nales 1D los filtros que aplicamos reconstruyeron se\~nales que se parecieron mucho m\'as a la se\~nal
original que la se\~nal con ruido aunque en muchos casos la diferencia era notable. Se puede percibir esto en las zonas
de alta frecuencia de la imagen. Se pudo notar ahi que el filtro promediador, incluso con coeficientes que preservaran
bastante las se\~nales originales en las frecuencia no alteradas por ruido, que estos peque\~nos cambios alcanzaban a
variar muchisimo la se\~nal reconstruida. Esto nos dejo bien claro que hay que tener cuidado a la hora de hacer alteraciones
en la base de los cosenos, no es para intuito el resultado que produce una perturbaci\'on en la reconstruccion de las se\~nales.

En se\~nales 2D, en cambio, ya que trabajabamos con im\'agenes, pudimos observar que la imagen con ruido en muchos
casos no se distingu\'ia (es decir, no se pod\'ia reconocer la imagen comparandola con la original), aunque despu\'es
de aplicar los filtros se reconoce una imagen bastante fiel a la original en comparaci\'on con la imagen ruidosa.
Es importante saber los casos de uso de las funciones para definir el rango de calidad \textit{aceptable}. No es lo mismo
la precision necesaria para identificar si hay, o no, una cara en la imagen que para un sistema de reconocimiento \'optico
de caracteres. Esto es clave para poder ver si se justifica aplicar filtros como promediador, donde se pierden un poco de detalles como suavizado
pero se gana en velocidad de identificaci\'on humana.

Lamentablemente, nos hubiera gustado contar con un tiempo adicional para intentar los filtros por bloques, como sugeria el ejercicio
opcional. \textit{A priori}, nos damos cuenta que son excelentes herramientas para mejorar el tiempo de ejecuci\'on
de las transformaciones, ya que el crecimiento cuadr\'atico y c\'ubico de la resoluci\'on de sistemas se not\'o claramente
Sin estas optimizaciones, seria imposible realizar sistemas de filtros por software en tiempo real, eliminando un campo de 
aplicaci\'on gigantesco. Sin embargo, sabemos que productos comerciales con estas caracteristicas existen, por lo que
seria interesante conocer los detalles de implementaci\'on para poder reconocer donde estan los cuellos de botella y como
eliminarlos.

En resumen, con este trabajo pudimos comprender un m\'inimo de la importancia de la transformacion de Fourier y similares,
un \'area que ha sido crucial para el desarrollo de las telecomunicaciones alrededor del mundo. Nos ha servido para plantear
el problema desde otro aspecto, como un ``simple'' cambio de base puede revelarnos mucha mas informaci\'on de la que 
consideramos existente en una se\~nal. 


\pagebreak

\section{Ap\'endices}
\begin{lstlisting}
#include<iostream>
#include<algorithm>
#include<vector>
#include<cstring>
#include<cstdio>
#include<cmath>
#include<cassert>

using namespace std;

vector<vector<double> > input, av; //av = autovectores

#define ERRCANT(x) {if(g!=(x)){printf ("Error de lectura %d \n",__LINE__); exit(1);}}
#define ERRCANT(x) 

void transpose(vector<vector<double> > &mat)
{
    vector<vector<double>> matrizAuxiliar;
    /** Transposicion de matrices **/
    int n = mat.size();
    int m = mat[0].size();
    matrizAuxiliar.clear();
    matrizAuxiliar.resize(m,vector<double>(n));
#pragma omp parallel for
	for(int i=0;i<m;i++)
    for(int j=0;j<n;j++)
	{
	    matrizAuxiliar[i][j] = mat[j][i];
    }
    mat = matrizAuxiliar;
	return;
}

vector<vector<double> > mult(const vector<vector<double> > &A, const vector<vector<double> > &B)
{
    /** Multiplicacion de matrices standard en o(n^3) **/
	int n = A.size();
	int m = B[0].size();
	int t = A[0].size(); // tiene que ser igual a B.size();
    vector<vector<double> > B2 = B;
    transpose(B2);
	vector<vector<double> > res(n,vector<double>(m,0));
#pragma omp parallel for
	for(int i=0;i<n;i++)
	{
    	for(int j=0;j<m;j++)
        {
            double sum =0.0;
	        for(int k=0;k<t;k++)
            {
		        sum += A[i][k]*B2[j][k];
            }
            res[i][j]=sum;
        }
    }
	return res;
}

void generateX(vector<vector<double> >& X)
{
	int n = input.size();
	int m = input[0].size();
    vector<double> average(m,0);	

    for(int i=0;i<n;i++)   
    	for(int j=0;j<m;j++)
	    	average[j] += input[i][j];
    #pragma omp parallel for
	for(int j=0;j<m;j++)
		average[j] /= (double)n;
    /** Calculo el promedio de cada pixel **/
	X.clear();
	X.resize(n,vector<double>(m));
#pragma omp parallel for
    for(int i=0;i<n;i++)
    	for(int j=0;j<m;j++)
	    	X[i][j] = (input[i][j]-average[j])/sqrt(n-1);
            /** A cada pixel le asigno el pixel en su imagen menos el promedio sobre la raiz de la cantidad de imagenes menos uno**/
	return;
}

vector<vector<double> > generateMx()
{
    vector<vector<double> > X;
    generateX(X); /** Genero la matriz X **/
    vector<vector<double> > Xt(X);	
	transpose(Xt);
	return mult(Xt,X); /** Genero Mx matriz de covarianza como X^t por X **/    
}

double norm(vector<double> &vec) /** Calculo norma 2 de vec **/
{
	double res = 0;
    #pragma omp parallel for reduction (+:res)
	for(int i=0;i<(int)vec.size();i++)
		res += vec[i]*vec[i];
	return sqrt(res);
}

vector<vector<double> > Id(int n) /** Identidad de n x n **/
{
    vector<vector<double> > A(n,vector<double>(n,0));
    for(int i = 0; i< n; i++)
        A[i][i]=1;
    return A;
}




void householder(vector<vector<double> > &A,vector<vector<double> > &Q,vector<vector<double> > &R)
{
    /** Factorizacion QR de Householder en o(n^3) **/
	int n = A.size();
    vector<vector<double> > aux(n,vector<double>(n)),aux2(n,vector<double>(n));
	
	R = A;
	Q = Id(n);
    vector<double> u,v;
    v=vector<double>(n);
	for(int i=0;i<n-1;i++)
	{        
        u=vector<double>(i,0.0);        
	    for(int j=i;j<n;j++)
            u.push_back(R[j][i]);
        double alpha = norm(u);
        if(abs(abs(alpha)-abs(u[i])) < 1e-6) /** Si debajo de la diagonal son todos ceros no itero **/
            continue;
        if(alpha*u[i]>0) /** Cambio el signo si es necesario **/
            alpha *= -1.;
        u[i] += alpha;
        alpha = norm(u);

        
#pragma omp parallel for         
        for(int j=0;j<n;j++)
            v[j]=u[j]/alpha;
        
        /** Inicio calculo R **/
#pragma omp parallel for 
        for(int j=0;j<n;j++)
            u[j] = 0;

#pragma omp parallel for
        for(int j=0;j<n;j++)
        for(int t=0;t<n;t++)        
            u[j] += v[t]*R[t][j];

#pragma omp parallel for
        for(int j=0;j<n;j++)
        for(int t=0;t<n;t++)
            aux[j][t] = 2*v[j]*u[t];

#pragma omp parallel for
        for(int j=0;j<n;j++)
        for(int t=0;t<n;t++)
            R[j][t] -= aux[j][t];
        /** Fin calculo R **/
        /** Inicio calculo Q **/
#pragma omp parallel for
        for(int j=0;j<n;j++)
            u[j] = 0;
            
#pragma omp parallel for    
        for(int j=0;j<n;j++)
        for(int t=0;t<n;t++)        
            u[j] += Q[j][t]*v[t];
#pragma omp parallel for        
        for(int j=0;j<n;j++)
        for(int t=0;t<n;t++)
            aux2[j][t] = 2*u[j]*v[t];
#pragma omp parallel for                
        for(int j=0;j<n;j++)
        for(int t=0;t<n;t++)
            Q[j][t] -= aux2[j][t];
        /** Fin calculo Q **/
	}
	return;
}

const double delta = 5000;

int iteraciones=0;

bool sigoIterando(vector<vector<double> > &A)
{
	double res = 0.;
	int n = A.size();
    #pragma omp parallel for reduction(+:res)
    for(int i=0;i<n;i++)
    {
        double localRes=0.;
	    for(int j=0;j<i;j++)
		    localRes += fabs(A[i][j]);
        res+=localRes;
    }
    printf("Iteracion %d: Suma %lf\t Promedio %lf\n",++iteraciones, res, res/((n*(n-1))/2));
	return res>delta;
	/** Itero hasta que los elementos debajo de la diagonal sumen menos de 15 **/
}
#define MAXITERACIONES 500

vector<vector<double> > Q,R;
vector<vector<double> > allAuVec; 
void eig(vector<vector<double> > &A, vector<vector<double> > &auVec)
{
	int n = A.size(); /// A es cuadrada
    vector<vector<double> > matrizAuxiliar;
    auVec = vector<vector<double> >(allAuVec);
    householder(A,Q,R); /** Calculo QR con Householder **/
    A = mult(R,Q); /** Multiplico RQ para obtener la nueva A que es la matriz de covarianza **/
    allAuVec = mult(allAuVec,Q); /** Multiplico todas las Q para obtener los autovectores **/
	/** Ordeno los autovectore segun la magnitud de los autovalores **/
	vector<pair<int,int> > aux(n);

#pragma omp parallel for
    for(int i=0;i<n;i++)
		aux[i] = make_pair(A[i][i],i);
	sort(aux.begin(),aux.end()); /** Ordeno los autovalores **/
	reverse(aux.begin(),aux.end());
	matrizAuxiliar = allAuVec;

#pragma omp parallel for
    for(int i=0;i<n;i++)
	for(int j=0;j<n;j++)
	{
		auVec[i][j] = matrizAuxiliar[aux[i].second][j];
		/** Asigno a la i-esima columna el autovector correspondiente al i-esimo autovalor **/
	}
	/** Fin ordenamiento de autovectores **/
	return;
}

vector<vector<double> > tc;

vector<double> calctc(vector<double> imagen, int k) /** Calculo la transfomacion caracteristica de imagen con parametro k **/
{
	vector<double> res(k,0);
	int n = imagen.size();
    
	for(int i=0;i<k;i++)
	for(int j=0;j<n;j++)
	{
		res[i]+=av[i][j]*imagen[j];
	}
	return res;
}

void fillTC(int k) /** LLeno tc con las transformaciones caracteristicas **/
{
	int n = input.size();
	tc.resize(n,vector<double>(k,0));
#pragma omp parallel for
	for(int i=0;i<n;i++)
	{
		tc[i] = calctc(input[i],k);
	}
	return;
}

double dist(vector<double> &v1, vector<double> &v2, int norm)
{
	double res = 0;
	if(norm==0) /** Norma infinito de v1 - v2 | Distancia infinito de v1 a v2 **/
	{
	    for(int i=0;i<(int)v1.size();i++)
            res = max(res,v1[i]-v2[i]);
	}
	if(norm==1) /** Norma 1 de v1 - v2 | Distancias 1 de v1 a v2 **/
	{
	    for(int i=0;i<(int)v1.size();i++)
            res += abs(v1[i]-v2[i]);
	}
    
    if(norm==2) /** Norma 2 de v1 - v2 | Distancia 2 de v1 a v2 **/
    {        
        for(int i=0;i<(int)v1.size();i++)
            res += (v1[i]-v2[i])*(v1[i]-v2[i]);
        res = sqrt(res);
    }
	return res;
}

void usage()
{
    cout << "Uso: ./OCR <k> <imp> <norma> <training> <test>" << endl;
    cout << "donde k = cantidad de componentes principales a tomar de las transformaciones "<< endl;
    cout << "      imp = 0 (usando nearest neighbours, 1 usando distancia al promedio " << endl;
    cout << "      norma = 0 (norma infinito), 1 (norma 1), 2 (norma 2) " << endl;
    cout << "      training = (default 10000) cantidad de imagenes de entrenamiento, 1 a 30000 " << endl;
    cout << "      test = (default 500) cantidad de imagenes de test, 1 a 2000 " << endl;
    return;
}

int main(int argc, char* argv[])
{
    if(argc<4){ usage(); exit(1);}
    if(argc>6){ usage(); exit(1);}
    int max_k = atoi(argv[1]), imp = atoi(argv[2]), norm = atoi(argv[3]);
    int min_k = 1;
    /** k es el parametro k del enunciado, imp es la implementacion y norm es la norma que usamos para medir distancias **/
    
    int training_count = 10000; /** Usamos 10000 imagenes de entrenamiento **/    
    int test_count = 500; /** Usamos 500 imagenes de test **/
    if(argc>4)
        training_count=atoi(argv[4]); /** a menos que el parametro lo indique **/
    if(argc>5)
        test_count=atoi(argv[5]); /** a menos que el otro parametro lo indique **/
    int padding_count = 30000-training_count; /** paddeamos con imagenes para siempre usar las mismas de test **/


	FILE* v = fopen("../datos/trainingImages.txt","r");
	int n, t; /** dims de la matriz de training **/
    int g;  /** absorbedora de errores **/
	g = fscanf(v,"%d %d",&n,&t);ERRCANT(2);
    printf("Se van a leer: %d imgs Training, %d imgs Test \n",training_count,test_count);
    vector<vector<double> > testImages;
    
    input.clear();
	input.resize(training_count,vector<double>(t));
	testImages.resize(test_count,vector<double>(t));
    /** Comienzo lectura imagenes de entrenamiento y test **/
	for(int i=0;i<training_count;i++)
    {
        for(int j=0;j<t;j++)
        {
            g = fscanf(v,"%lf",&input[i][j]);ERRCANT(1);
        }
    }
    for(int i=0;i<padding_count;i++)
    {        
        for(int j=0;j<t;j++)
        {
            g = fscanf(v,"%*lf");
        }
    }
    for(int i=0;i<test_count;i++)
    {
		for(int j=0;j<t;j++)
		{
			g = fscanf(v,"%lf",&testImages[i][j]);ERRCANT(1);
		}
	}
    fclose(v);
	/** Fin lectura imagenes de entrenamiento y test **/
	/** Comienzo lectura labels de entrenamiento y test **/
    v = fopen("../datos/trainingLabels.txt","r");
    g= fscanf(v,"%d",&n); ERRCANT(1);
    vector<int> labels, testLabels;
    labels.resize(training_count);
    testLabels.resize(test_count);
    for(int i=0;i<training_count;i++)
    {
        g = fscanf(v,"%d",&labels[i]);ERRCANT(1);
    }
    
    for(int i=0;i<padding_count;i++)
    {
        g = fscanf(v,"%*d");
    }

    for(int i=0;i<test_count;i++)
    {
		g = fscanf(v,"%d",&testLabels[i]);ERRCANT(1);
	}
    
	fclose(v);
	/** Fin lectura labels de entrenamiento y test **/
	v = fopen("V.txt","r");
	if(v==NULL) /** Si V no existe la genero **/
	{
#ifdef PRECALC
        vector<vector<double> > Mx = generateMx(); // Genero Mx la matriz de covarianza 
        allAuVec = Id(Mx.size()); /** Inicializo la matriz de autovectores **/
        bool b = true;
        for(int its=1;its<MAXITERACIONES;its++)
        {
            
            eig(Mx,av); // Calculo los autovectores de la matriz de covarianza 
            b = sigoIterando(Mx);
            if (!b) break;
            
        }

        v = fopen("V.txt","w"); // Escribo la matriz V en un archivo 
        fprintf(v,"%d %d\n",(int)av.size(),(int)av[0].size());
        for(int i=0;i<(int)av.size();i++)
        {
            for(int j=0;j<(int)av[0].size();j++)
                fprintf(v,"%.6lf ",av[i][j]);
            fprintf(v,"\n");
        }
        fclose(v);
#endif
	}
	else /** Si V ya fue generada previamente la levanto del archivo **/
	{
	    int N,M;
	    g=fscanf(v,"%d %d",&N,&M);ERRCANT(2);
	    av.clear();
	    av.resize(N,vector<double>(M));
	    for(int i=0;i<N;i++)
        {
            for(int j=0;j<M;j++)
            {
                g=fscanf(v,"%lf",&av[i][j]);ERRCANT(1);
            }
        }

	}
#ifndef PRECALC
    vector<vector<double> > Mx = generateMx(); /** Genero Mx la matriz de covarianza **/
    allAuVec = Id(Mx.size()); /** Inicializo la matriz de autovectores **/
    for(int its=1;its<MAXITERACIONES;its++)
    {
        eig(Mx,av); /** Calculo los autovectores de la matriz de covarianza **/        
        bool b = sigoIterando(Mx);
#endif
        for(int k=min_k;k<max_k;k++)
        {
            fillTC(k); /** Genero las transformaciones caracteristicas de cada imagen de entrenamiento **/
            vector<double> vec;
            vector<pair<double,int> > distancias;    
            vector<int> cant(10);
            int bien = 0;
            int mal = 0;

            if(imp==0)
            {
                distancias.resize(training_count);
                /** La implementacion 0 toma las 100 imagenes mas cercanas y toma la mas repetida de esas 100 **/
                for(int i=0;i<test_count;i++) /** Iteramos sobre la imagenes de test **/
                {
                    vec = calctc(testImages[i],k); /** Calculamos la transformacion caracteristica de la imagen dada **/
                                
                    for(int j=0;j<training_count;j++)
                    {
                        distancias[j] = make_pair(dist(tc[j],vec,norm),j);
                        /** Guardamos en un par la distancia a cada imagen de entrenamiento con la imagen **/
                    }
                    sort(distancias.begin(),distancias.end()); /** Ordenamos por distancia **/
                    for(int j=0;j<10;j++)
                        cant[j] = 0;            
                    for(int j=0;j<100;j++)
                    {
                        cant[labels[distancias[j].second]]++;
                        /** Contamos cuantas imagenes hay de las 100 mas cercanas con cada label **/
                    }
                    int cualEs = 0;
                    for(int j=0;j<10;j++)
                        if(cant[j]>cant[cualEs])
                            cualEs = j;
                    if(testLabels[i]==cualEs)
                        bien++;
                    else
                        mal++;
                }
            }
            else if(imp == 1)
            {
                /** La implementacion 1 toma el promedio de las transformaciones de cada digito y compara distancia a cada promedio **/
                vector<double> promedio[10];
                int cuantos[10];
                memset(cuantos,0,sizeof(cuantos));
                /** Comienzo calculo promedios **/
                for(int i=0;i<10;i++)
                {
                    promedio[i].clear();
                    promedio[i].resize(k,0);
                }
                for(int i=0;i<training_count;i++)
                {
                    cuantos[labels[i]]++;
                    for(int j=0;j<k;j++)
                        promedio[labels[i]][j] += tc[i][j];
                }
                for(int i=0;i<10;i++)
                for(int j=0;j<k;j++)
                if(cuantos[i]!=0)
                    promedio[i][j] /= (double)cuantos[i];
                /** Fin calculo promedios **/
                for(int i=0;i<test_count;i++)
                {
                    vec = calctc(testImages[i],k);
                    distancias.resize(10);
                    for(int j=0;j<10;j++)
                    {
                        distancias[j] = make_pair(dist(promedio[j],vec,norm),j);
                        /** Comparamos distancia al promedio de cada digito **/
                    }
                    sort(distancias.begin(),distancias.end());
                    int cualEs = distancias[0].second;
                    /** Ordenamos y nos quedamos con el mas cercano **/
                    if(testLabels[i]==cualEs)
                        bien++;
                    else
                        mal++;
                }
            }
            cout << k << " "<< bien <<" "<<mal<<" " << (double)bien/(double)(bien+mal)<< endl;
        /** Imprimimos cantidad de hits, cantidad de misses y proporcion de hits **/
        }
#ifndef PRECALC
    }
#endif
    return 0;
}

\end{lstlisting}


\pagebreak

\section{Referencias}

\begin{itemize}
  \item \nocite {Burden} Burden, Richard; Faires, Douglas. An\'alisis 
num\'erico. Brooks Cole, 2011.
  \item \nocite {MN} Clases de M\'etodos Num\'ericos. Departamento de 
Computaci\'on, UBA. Primer Cuatrimestre de 2013.
\item \nocite {Turk91} Turk, Matthew A., and Alex P. Pentland. "Face recognition using eigenfaces." Computer Vision and Pattern Recognition, 1991. Proceedings CVPR'91., IEEE Computer Society Conference on. IEEE, 1991.
\item \nocite {Sirovich87} Sirovich, Lawrence, and Michael Kirby. "Low-dimensional procedure for the characterization of human faces." JOSA A 4.3 (1987): 519-524.
\end{itemize}



\end{document}
