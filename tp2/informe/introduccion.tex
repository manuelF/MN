\section{Introducci\'on te\'orica}

El siguiente trabajo consiste en la experimentaci\'on de la denominada 
Transformada Discreta del Coseno (DCT). La exploraci\'on de la DCT se da
mediante la implementaci\'on de una de sus aplicaciones m\'as utilizadas como lo
es la reducci\'on de ruido en se\~nales ruidosas. A continuaci\'on se explicita
un breve repaso de los conceptos clave para el entendimiento de los
experimentos.

\subsection{Transformada del Coseno (DCT)}

La Transformada del Coseno expresa una secuencia finita de datos (o se\~nal) 
con cierta informaci\'on en una suma de funciones coseno con distintas 
frecuencias.

Formalmente la DCT es una funci\'on $f:\mathbb{R}^\mathbb{N} \to \mathbb{R}^
\mathbb{N}$ o analogamente una
Matriz cuadrada de $\mathbb{N} X \mathbb{N}$. La f\'ormula de la transformada es
(revisar)

$$X_k = \sum_{n=0}^{N-1} x_n \cos \left[\frac{\pi}{N} \left(n+\frac{1}{2}\right) 
k \right] \quad \quad k = 0, \dots, N-1.$$ 

La DCT es muy parecida a la Transformaci\'on discreta de Fourier (DFT), solo que
la DCT se limita a trabajar con n\'umeros reales. Esto la hace m\'as propicia a
la hora de trabajar con muestras digitales.

El uso del coseno (en contraposici\'on con la funci\'on seno) en el uso de este
tipo de transformaciones a la hora de realizar compresi\'on de audio e imagen
est\'a dado por la menor cantidad de funciones a calcular al trabajar con
cosenos.

Entre las aplicaciones m\'as utilizadas se encuentran:

\begin{itemize}
\item Compresi\'on de audio (MP3, WMA, Vorbis, AAC)
\item Compresi\'on de imagen (JPEG)
\item Compresi\'on de video (MPEG)
\item Procesamiento de se\~nales multimedia (Eliminaci\'on de ruido)
\end{itemize}

\subsection {Relaci\'on Señal a Ruido de Pico (PSNR)}

El PSNR, definido como:

$$
\mathit{PSNR} = 10 \cdot \log_{10} \left( \frac{\mathit{MAX}^2_x}{\mathit{ECM}}
\right)
$$

define la relaci\'on entre la m\'axima ''energ\'ia'' posible de una se\~nal y el
ruido que la afecta. Debido a que las se\~nales tienen por lo general rangos
amplios, se suele representar en una escala logar\'itimica. La utilizaci\'on del
 Error Cuadr\'atico Medio permite que el valor retornado no dependa de los
valores a calcular, con lo cual un valor determinado (como por ejemplo el 1) nos
da una noci\'on del error cometido sin importar de que se\~nal estamos hablando.
