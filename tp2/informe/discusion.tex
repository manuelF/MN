\section{Discusi\'on}

\subsection{Filtros en 1D}

Pudimos observar en la figura \ref{fig:Dopp1024zero}, que el filtro Cero
aplicado a la se\~nal detecta frecuencias fuera del umbral y
las manda a cero. Este filtro carece de una caracter\'istica fundamental
seg\'un demuestran los experimentos:
{\bf detecci\'on de ``zonas afectadas''}. Como el algoritmo no tiene en
cuenta el contexto a la hora de modificar un punto, la decisi\'on es mandar los
puntos fuera del umbral al cero. Esto lleva a ``saltos'' pronunciados en
el espectro de la se\~nal recuperada.

Sin embargo, como se aprecia en la im\'agen, los resultados son suficientemente
buenos teniendo en cuenta la simplicidad de la implementaci\'on del algoritmo.\\

El filtro Exponencial surge como una mejora directa del filtro Cero.
La idea que proviene de observar los resultados provistos por el Filtro Cero est\'a relacionada con la
noci\'on de agregado de ``zona afectada'' ante la detecci\'on de un punto fuera
del umbral determinado. Como se ve a continuaci\'on, esta modificaci\'on trae
acarreada una mayor preservacion de la informaci\'on en los puntos lindantes al que nosotros
definimos que debe ser filtrado (el punto que cruza el umbral de tolerancia). Esto conlleva 
a una mayor suavidad a la se\~nal recuperada.



Otra variante utilizada fue la combinaci\'on de los filtros exponencial y
de la mediana. Al filtrar la se\~nal ruidosa con ambos filtros obtuvimos interesantes
resultados.

En este punto quisimos probar la potencia de la combinaci\'on de filtros para resolver
el ruido blanco. Como no hay un claro pico, vemos si la combinaci\'on de ambos
filtros puede encontrar una forma de eliminar los peque\~nos picos y suavizarlos.
Los podemos apreciar en la figura \ref{fig:dopcomb} y en la figura \ref{fig:scomb}.
En estos podemos ver que, el ruido en la figura \ref{fig:scomb} es atenuado
en el espectro y en la figura \ref{fig:dopcomb} es recuperada una se\~nal que
comparte muchas similitudes con la original, salvando las frecuencias
m\'as altas. Probablemente con una elecci\'on m\'as cuidadosa de parametros
se hubieran podido rescatar m\'as detalles.

Otra idea interesante que surge de la implementaci\'on del ruido impulsivo y tambi\'en del
filtro de la mediana que lo mitiga con \'exito es que la detecci\'on del tipo de ruido que afecta
a la se\~nal puede ahorrar tiempo y aumentar la eficiencia del algoritmo, realizando un filtrado m\'as
inteligente.

\subsection{Filtros en 2D}


En la Figura \ref{fig:Sin50All} vemos los efectos de ruido senoidal
de coeficiente 50. Vemos que perturba sobre todo las areas mas oscuras
de la fotograf\'ia. El filtro de promedios logra solamente recuperar la mitad
derecha de la imagen y con un fuerte efecto de difuminado.
Esto es as\'i porque los detalles se encuentran en las mas altas frecuencias
y las estamos aplanando con los promedios, por eso el efecto borroso.
Los plenos, en cambio, se encuentran en las frecuencias mas bajas, pero
por el filtro que hicimos, que arranca desde el 5\% de la señal,
no siempre se logra recuperar todo.

En cambio, con el filtro exponencial, podemos detectar exactamente ese
tipo de ruido y disminuirlo, haciendo un gran trabajo y preservando tanto
los plenos como los detalles. Luego, combinarlos solo elimina definici\'on,
no recomendable.\\


Como se puede ver en la Figura \ref{fig:Sin1000All}, el ruido senoidal
de coeficiente 1000 hace cosas extra\'nas en la fotograf\'ia.
Este ruido es controlado pero el multiplicador es muy alto, haciendo muy 
dif\'icil la interpretaci\'on de la im\'agen. A simple vista, no se puede distinguir ning\'un rasgo,
salvo el tenue contorno de la persona.
Esta im\'agen ser\'ia totalmente in\'util en la gran mayoria de los casos.

La mejor\'ia obtenida aplicando los filtros es asombrosa.
El filtro exponencial logra captar los fuertes picos en el espectro de
frecuencias y con su atenuaci\'on, permite recuperar los rasgos principales
de la figura humana.
Hay datos que parecen perdidos, o que al menos con esta t\'ecnica y
parametros de configuraci\'on no logramos captar, como el fondo de
libros oscuros. No se detectan mejoras usando ambos filtros simultaneamente.

Eliminar ruido gaussiano de la im\'agen fue la actividad m\'as desafiante
ya que el ruido blanco es muy d\'ificil de capturar por su condici\'on de ``antipatron''.

En la Figura \ref{fig:gauss2d} vemos en este caso, que aplicar el filtro
promediador luego de haber aplicado el exponencial, es un error. 
Se reintroducen muchos de los \textit{artifacts} que se habian podido
eliminar antes, sin ninguna mejoria de sharpness en la imagen. 

Sin embargo, aplicandolo solo se consigue una buena mejor\'ia en la calidad
de la im\'agen, aunque se difuminen muchos rasgos, la recuperaci\'on puede ser aceptable.
