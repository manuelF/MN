\section{Discusi\'on}

\subsection{Filtros en 1D}

Pudimos observar en la figura \ref{fig:Dopp1024zero}, que el filtro Cero
aplicado a la se\~nal detecta frecuencias fuera del umbral y
las manda a cero. Este filtro carece de una caracter\'istica fundamental
seg\'un demuestran los experimentos:
{\bf detecci\'on de ``zonas afectadas''}. Como el algoritmo no tiene en
cuenta el contexto a la hora de modificar un punto, la decisi\'on es mandar los
puntos fuera del umbral al cero. Esto lleva a ``saltos'' pronunciados en
el espectro de la se\~nal recuperada.

Sin embargo, como se aprecia en la im\'agen, los resultados son suficientemente
buenos teniendo en cuenta la simplicidad de la implementaci\'on del algoritmo.\\

El filtro Exponencial surge como una mejora directa del filtro Cero.
La idea que proviene de observar los resultados provistos por el Filtro Cero est\'a relacionada con la
noci\'on de agregado de ``zona afectada'' ante la detecci\'on de un punto fuera
del umbral determinado. Como se ve a continuaci\'on, esta modificaci\'on trae
acarreada una mayor preservacion de la informaci\'on en los puntos lindantes al que nosotros
definimos que debe ser filtrado (el punto que cruza el umbral de tolerancia). Esto conlleva 
a una mayor suavidad a la se\~nal recuperada.

Otra idea interesante que surge de la implementaci\'on del ruido impulsivo y tambi\'en del
filtro de la mediana que lo mitiga con \'exito es que la detecci\'on del tipo de ruido que afecta
a la se\~nal puede ahorrar tiempo y aumentar la eficiencia del algoritmo, realizando un filtrado m\'as
inteligente.
%%FALTA REFERENCIA A ALGUNA IMAGEN ACA


Otra variante utilizada fue la combinaci\'on de los filtros exponencial y
de la mediana. Al filtrar la se\~nal ruidosa con ambos filtros obtuvimos interesantes
resultados.

En este punto quisimos probar la potencia de la combinaci\'on de filtros para resolver
el ruido blanco. Como no hay un claro pico, vemos si la combinaci\'on de ambos
filtros puede encontrar una forma de eliminar los peque\~nos picos y suavizarlos.
Los podemos apreciar en la figura \ref{fig:dopcomb} y en la figura \ref{fig:scomb}.
En estos podemos ver que, el ruido en la figura \ref{fig:scomb} es atenuado
en el espectro y en la figura \ref{fig:dopcomb} es recuperada una se\~nal que
comparte muchas similitudes con la original, salvando las frecuencias
m\'as altas. Probablemente con una elecci\'on m\'as cuidadosa de parametros (por ejemplo
el valor del umbral de detecci\'on de pico) se hubieran podido rescatar m\'as detalles.


\subsection{Filtros en 2D}

La im\'agen presentada contiene tanto areas muy claras como muy oscuras, y buena definici\'on
en algunos sectores y mas borrosa en otros. Esto hace que la im\'agen sea interesante para probar
con los diferentes ruidos.

Eliminar ruido gaussiano de la im\'agen fue la actividad m\'as desafiante
ya que el ruido blanco es muy d\'ificil de capturar por su condici\'on de ``antipatron''.

En la Figura \ref{fig:gauss2d} vemos en este caso, los resultados de aplicar los 
distintos filtros, y las combinaciones. El filtro exponencial en este caso

que aplicar el filtro
promediador luego de haber aplicado el exponencial, es un error. 
Se reintroducen muchos de los \textit{artifacts} que se habian podido
eliminar antes, sin ninguna mejoria de sharpness en la imagen. 

Sin embargo, aplicandolo solo se consigue una buena mejor\'ia en la calidad
de la im\'agen, aunque se difuminen muchos rasgos, la recuperaci\'on puede ser aceptable.
