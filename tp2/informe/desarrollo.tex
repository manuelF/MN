
\section{Desarrollo}

El desarrollo del experimento no tuvo grandes dificultades. La gran cantidad de
variantes de implementaciones encontradas del algoritmo a utilizar debido al
amplio uso de la DCT para la reducci\'on de ruido en se\~nales, sumado a lo
aprendido sobre el trabajo con matrices en el Trabajo Pr\'actico N\'umero 1
posibilitaron el desarrollo de un experimento limpio y de simple ejecuci\'on.

El desarrollo exitoso del c\'odigo permiti\'o poner el foco del experimento en
el an\'alisis de los datos obtenidos y en la progresiva del tiempo de corrida
de los algoritmos.

En este trabajo decidimos experimentar con instrucciones del compilador que
permiten ejecutar c\'odigo en paralelo. Esta variante ayuda a bajar tiempos de
corrida con matrices de tama\~o considerable.

\index{Desarrollo!Herramientas}
\subsection{Herramientas}

Para la realizaci\'on de los experimentos se utiliz\'o C++ como lenguaje de
programaci\'on para los algoritmos principales. Esta elecci\'on se apoya en
nuestro conocimiento del lenguaje, la eficiencia que aporta y las estructuras de
datos que proporciona, haciendo simple y clara la implementaci\'on de la
resoluci\'on del problema. Como fue comentado anteriormente, fue utilizada la
directiva ``\#pragma omp parallel'' en ciertas porciones de c\'odigo
paralelizables, con el fin de reducir el tiempo de ejecuci\'on donde es posible.

El an\'alisis de los resultados y el ploteo de los gr\'aficos se realiz\'o
mediante el uso del entorno \href{http://www.gnu.org/software/octave/}{GNU
Octave}, libre y compatible con MATLAB.

El control de los experimentos y las ``recetas'' de las experiencias se ven
reflejadas en scripts de shell que levantan archivos de datos con tests
preprogramdos, ejecutan los algoritmos y luego analizan los resultados,
resumiendolos en output en forma de informaci\'on tabular y gr\'aficos.
La existencia de un Makefile fue clave a la hora de automatizar tareas de
compilaci\'on y corrida de tests.

El informe fue realizado mediante la utilizaci\'on de \href{http://www.latex-project.org/}{LaTeX},
 manteniendo la estructura de informe del Trabajo Pr\'actico N\'umero 1 y
permitiendo expresar estructuras y f\'ormulas matem\'aticas de una forma simple.

\subsection{Desarrollo del experimento de reducci\'on de ruido}

Para reducir el ruido utilizamos tres tipos de filtros: Filtro cero, filtro exponencial y filtro promediador.
En los tres casos empezamos a filtrar las se\~nales desde el 5\% (es decir, no procesamos el 5\% de se\~nales
de menor frecuencia) ya que experimentando encontramos que en ese 5\% se encuentra generalmente gran parte de
la informaci\'on, y en proporci\'on muy poco ruido.
