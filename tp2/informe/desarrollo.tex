
\section{Desarrollo}

La preparaci\'on del experimento pareci\'o no acarrear grandes dificultades, permitiendo
focalizar los esfuerzos en la b\'usqueda, an\'alisis y mejora de filtros. 
La gran cantidad de variantes de implementaciones encontradas del algoritmo a 
utilizar, debido al amplio uso de la DCT para la reducci\'on de ruido en se\~nales, 
sumado a lo aprendido sobre el trabajo con matrices en el Trabajo Pr\'actico 
N\'umero 1, posibilitaron el desarrollo de un experimento limpio y de simple 
ejecuci\'on.

Sin embargo, luego de la primera correcci\'on del trabajo, encontramos que esta
aparente facilidad correspond\'ia a la simplicidad de los tipos de ruidos simulados.
Para la re-entrega del trabajo, decidimos agregar un ruido que se presenta en general
en se\~nales de audio como lo es el ruido impulsivo acompa/~nado de una implementaci\'on
del filtro de la mediana para mitigarlo (ambas funciones son analizadas en detalle a lo 
largo de esta secci\'on).    	

Para un mejor entendimiento del fen\'omeno del ruido en las se\~nales, dividimos
el an\'alisis en distintos casos seg\'un el tipo de ruido a tratar. 
Por ejemplo decidimos tomar al ruido blanco como un caso de testeo importante 
debido a la \"falta de patrones\" que lo caracteriza. 

En este trabajo decidimos experimentar con instrucciones del compilador que
permiten ejecutar c\'odigo en paralelo. Esta variante ayuda a bajar tiempos de
corrida con matrices de tama\~o considerable.

\index{Desarrollo!Herramientas}
\subsection{Herramientas}

Para la realizaci\'on de los experimentos se utiliz\'o C++ como lenguaje de
programaci\'on para los algoritmos principales. Esta elecci\'on se apoya en
nuestro conocimiento del lenguaje, la eficiencia que aporta y las estructuras de
datos que proporciona, haciendo simple y clara la implementaci\'on de la
resoluci\'on del problema. Como fue comentado anteriormente, fue utilizada la
libreria OpenMP en ciertas porciones cr\'iticas de c\'odigo, con el fin de 
reducir el tiempo de ejecuci\'on donde es posible.

El an\'alisis de los resultados y el ploteo de los gr\'aficos se realiz\'o
mediante el uso del entorno \href{http://www.gnuplot.info/}{GNUPLOT}

El control de los experimentos y las ``recetas'' de las experiencias se ven
reflejadas en scripts de shell que levantan archivos de datos con tests
preprogramdos, ejecutan los algoritmos y luego analizan los resultados,
resumiendolos en output en forma de informaci\'on tabular y gr\'aficos.
La existencia de un Makefile fue clave a la hora de automatizar tareas de
compilaci\'on y corrida de tests, para poder ser lo mas automatizados dentro de
lo posible.

El informe fue realizado mediante la utilizaci\'on de \href{http://www.latex-project.org/}{\LaTeX},
 manteniendo la estructura de informe del Trabajo Pr\'actico N\'umero 1 y
permitiendo expresar estructuras y f\'ormulas matem\'aticas de una manera simple
y clara.

\subsection{Desarrollo del experimento de reducci\'on de ruido}

Para reducir el ruido utilizamos cuatro tipos de filtros: Filtro cero, filtro 
exponencial, filtro promediador y filtro de la mediana. En los cuatro casos empezamos 
a filtrar las 
se\~nales desde el 5\% (es decir, no procesamos el 5\% de se\~nales de menor
frecuencia) ya que experimentando encontramos que en ese 5\% se encuentra 
generalmente gran parte de la informaci\'on, y en proporci\'on muy poco ruido.

Esto es sobre todo notorio en se\~nales de audio. La teor\'ia atr\'as de esto es la amplitud
de frecuencias de la ac\'ustica humana. El rango de audici\'on humana se encuentra entre los 20Hz
y los 20kHz. Este amplio rango concentra la mayor cantidad de informaci\'on en la parte m\'as baja,
normalmente inferior a los 4kHz (el rango vocal humano). Por eso consideramos que
si hubiera ruido en este sector y lo intentaramos sacar, es mucho m\'as probable que elimin\'aramos
informaci\'on crucial del audio. Por eso preferimos tocar lo menos posible ese \'area,
a riesgo de no extraer los m\'aximos niveles de ruido.

\subsection{Filtro Cero}

En el filtro cero tomamos la se\~nal con el mayor coeficiente (en m\'odulo) y 
eliminamos todas las se\~nales que sean al menos el 50\% de la se\~nal de mayor 
coeficiente en m\'odulo. Esto lo hicimos porque cre\'iamos que las se\~nales con 
coeficientes m\'as altos eran las m\'as ruidosas. El primer filtro que aplicamos 
fue el filtro cero sin descartar el primer 5\% y los resultados no fueron muy 
buenos. Cuando se nos ocurrio agregarle a la implementaci\'on la omisi\'on del 
primer 5\% a la hora de filtrar vimos que los resultados fueron mucho mejores.

La formula empleada si el pico estuviera en $i$ es:

\begin{center}
    $y'[j] = 0 \iff j \in [i-contorno; i+contorno]$\\
    $y'[j] = y[j] \iff j \not \in [i-contorno; i+contorno]$\\
\end{center}

donde $contorno$ es el rango de valores alrededor del pico que estamos tomando, y
$j$ es el iterador que va desde $i-contorno$ hasta $i+contorno$, para poder anular
(llevar a 0) a los valores perif\'ericos al punto.

\subsection{Filtro Exponencial}


El filtro exponencial mejora al filtro cero evitando la p\'erdida total
informaci\'on proporcionada por la se\~nal.

El m\'etodo aplicado sobre los picos en este caso es el de Exponential Decay
(decaimiento exponencial). En pos de mantener informaci\'on sobre la se\~nal a
la hora de ajustarla la idea es, en este caso, reducir los picos en vez de
llevarlos a cero pero tambi\'en actuar sobre su entorno, reduciendo los puntos
cercanos en menor medida.

La formula empleada si el pico estuviera en $i$ es:
\begin{center}
$y'[j] = y[j]* (1.2^{(contorno-abs(j-i)+1)})^{-1}$\\
\end{center}
donde $contorno$ es el rango de valores alrededor del pico que estamos tomando, y
$j$ es el iterador que va desde $i-contorno$ hasta $i+contorno$, para poder aplicar
el decaimiento exponencial a todos los valores perif\'ericos al punto.

\subsection{Filtro Promedidador}

El filtro promediador distribuye los coeficientes m\'as grandes 
entre las se\~nales vecinas, es decir, si el coeficiente $k$-\'esimo es muy 
grande, le transfiere parte del valor a sus vecinos localizados en $k-2$, $k-1$, $k+1$ y
$k+2$. La inspiraci\'on de este m\'etodo surge de la discretizaci\'on m\'as sencilla
de las ecuaciones diferenciales de segundo orden de difusi\'on de Poisson, donde
se modela el comportamiento de un punto en un momento posterior en base a sus vecinos.\\

\begin{center}
$y'[i]=0.15*y[i-2] + 0.2*y[i-1] + 0.3*y[i]+0.2*y[i+1]+0.15*y[i+2]$\\
\end{center}

Es de notar que la suma de los coeficientes adem\'as da 1, de modo que no se 
eliminan o agregan valores, sino que se distribuye de forma pareja a sus vecinos.

La idea del filtro promediador surge de la observaci\'on de la aplicaci\'on de
los otros en el caso del ruido blanco. Para poder afrontar casos con ruido en
distintas frecuencias muy distintas al mismo tiempo necesitamos obtener un
contexto mayor en cada punto para saber como manejarlos en relaci\'on con sus
vecinos.

\subsection{Filtro de la Mediana}

El filtro de la mediana actua sobre cada muestra de la se\~nal (luego del 5\% inicial),
reemplazando cada valor a calcular por la mediana del arreglo de valores conformado por
las muestras de $k - 4$ hasta $k + 4$, siendo $k$ el \'indice del valor a calcular en la 
se\~nal. Para hacer esto, se toman los valores del intervalo, se realiza un ordenamiento
de menor a mayor y se toma el valor medio del arreglo resultante. 
En este caso, como el intervalo est\'a determinado por 9 elementos, se toma el 5to valor.

La decisi\'on de tomar 8 valores alrededor del valor a reemplazar est\'a dada por la observaci\'on
de los valores tomados por el PSNR al aplicar distintos ruidos a se\~nales y luego utilizar este 
filtro. Usando 4 valores a cada lado obtuvimos valores que nos resultaron satisfactorios sin 
sacrificar performance (al no elegir intervalos muy grandes). La aplicaci\'on del filtro en cada
posici\'on se puede ver mediante el siguiente pseudoc\'odigo:

\begin{center}
$y'[i]= sort(y, i - 4, i + 4)[4]$\\
\end{center}

La generaci\'on de este filtro est\'a intimamente ligado a la intenci\'on de mitigar se\~nales contaminadas
con ruido impulsivo. Como el ruido impulsivo se caracteriza por manifestarse peri\'odicamente y durante tiempos
cortos, el filtro de la mediana los evita ya que no ser\'an el valor de la mediana del entorno si este es suficientemente
grande como para contener m\'as elementos regulares en la se\~nal que los pertenecientes al pulso.

\subsection{Aplicaci\'on de ruido}

Se experiment\'o con 3 clases de ruido en las se\~nales:

\begin{itemize}
	\begin{item} {\bf Ruido senoidal:} Aplica un ruido formado por la suma de la se\~nal
original con la funci\'on seno aplicada en la posici\'on de la muestra
correspondiente, multiplicado por una constante $k$, para darle mayor amplitud al resultado de
la funci\'on seno.

	Es decir, para cada muestra de nuestra se\~nal aplicamos ruido del siguiente
modo:

$$se\widetilde{n}al_{ruidosa}[i] = se\widetilde{n}al_{original}[i] + k * \sin(i)$$

  
Tomamos como valores de $k$, $k=50$ y $k=1000$ siendo $k$ un valor constante para testear el comportamiento en casos con excesivo ruido, llev\'andolo a valores
como 1000.

El inter\'es te\'orico de este ruido surge de la industria de la
telecomunicaci\'on. Es un ruido que aparece en las im\'agenes por la proximidad de los cables portadores de se\~nales a los cables de corriente alterna. La corriente alterna
es una se\~nal sinusoidal de frecuencia entre  50 y 60Hz. Es un problema fundamental de solucionar, ya que es muy com\'un en
las instalaciones de c\'amaras de seguridad por ejemplo, la existencia de ambos conductores pasando por el mismo ducto, sin separaci\'on
alguna entre s\'i. Esto genera una se\~nal ghost o lavada, perdiendo definici\'on y, en algunos casos, haciendola irreconocible incluso.
	\end{item}

	\begin{item}
		{\bf Ruido Gaussiano:} Este tipo de ruido, tambi\'en denominado ruido
blanco, consiste en sumar a cada punto de la se\~nal un n\'umero aleatorio
proveniente de una distribuci\'on normal de media $\mu = 0$ y varianza $\sigma=100$. 
Este tipo de ruido es muy frecuente e importante en industrias como la radio o television.
Modela apropiadamente la interferencia est\'atica, y es producto de eventos no deterministicos como 
puede ser la radiaci\'on de fondo del Big Bang. Consiste en un ruido totalmente aleatorio, pero con 
una distribuci\'on muy similar a la normal comprendido en todas las frecuencias.

 	Para cada punto se aplica el ruido de la siguiente manera:

    $$se\widetilde{n}al_{ruidosa}[i] = se\widetilde{n}al_{original}[i] + random_{gaussiana}(\mu=0, \sigma=100)$$


	\end{item}

	\begin{item}
		{\bf Ruido Impulsivo:} Este tipo de ruido consiste en sumar a ciertos puntos dispersos 
de la se\~nal un n\'umero altamente mayor en m\'odulo a la media. Este tipo de ruido es muy 
frecuente en se\~nales de audio y se manifiesta como pulsos o clicks.
Generalmente se utiliza el Filtro de la Mediana para mitigarlo.

 	Para cada punto se aplica el ruido de la siguiente manera:

    $$se\widetilde{n}al_{ruidosa}[i] = se\widetilde{n}al_{original}[i] + fSigno() * VALOR_PICO$$

	Donde VALOR\_PICO es una constante y $fSigno$ es una funci\'on que devuelve 1 \'o -1 dependiendo
	de un factor (en nuestro caso cada 3 ticks cambia el signo).

	\end{item}

\end{itemize}

\subsection{Pasaje a dos dimensiones}

A la hora de pasar el experimento a 2 dimensiones, utilizando im\'agenes en
escala de grises (para mayor simplicidad), tomamos un approach que result\'o
natural y poco costoso viniendo de un mundo de una sola dimensi\'on y que a
posteriori demostr\'o ser confiable como se mostrar\'a m\'as adelante en los
resultados.

La idea en el paso a dos dimensiones es simple. La idea es aplicar los mismos
filtros implementados en una dimension y volver a aplicarlos para cada fila de
la matriz de la imagen ingresada.

De este modo los algoritmos utilizados durante la primera parte del experimento
fueron utilizados sin necesidad de cambios en esta segunda etapa.

Despu\'es de esto aplicamos los mismos filtros aplicados por bloques. En cada bloque
aplicabamos los mismos filtros pero en vez de trabajar con un entorno en una dimensi\'on
trabajamos con un entorno en dos dimensiones. El resultado de esto fue pr\'acticamente similar
al que obtuvimos aplicando el filtro s\'olo por filas.

Algo que intentamos pero que nos llev\'o a muy malos resultados fue aplicar el filtro por filas y por columnas.

A medida que diferentes experimentos fueron agregados, la observaci\'on de
im\'agenes resultantes fue de gran ayuda para detectar errores de c\'alculo como
as\'i tambi\'en patrones de comportamiento de los filtros.

Detectamos tambi\'en que utilizando esta aproximaci\'on a la extensi\'on a dos
dimensiones por fila, encontramos que los errores m\'as groseros, es decir las
diferencias m\'as notables entre la im\'agen original y la recuperada, se
encuentran en determinadas columnas que var\'ian en cada im\'agen pero se
revelan como franjas verticales.
