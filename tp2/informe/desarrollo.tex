
\section{Desarrollo}

La preparaci\'on del experimento no tuvo grandes dificultades, permitiendo
focalizar los esfuerzos en la b\'usqueda, an\'alisis y mejora de filtros. 
La gran cantidad de variantes de implementaciones encontradas del algoritmo a 
utilizar debido al amplio uso de la DCT para la reducci\'on de ruido en se\~nales, 
sumado a lo aprendido sobre el trabajo con matrices en el Trabajo Pr\'actico 
N\'umero 1 posibilitaron el desarrollo de un experimento limpio y de simple 
ejecuci\'on.

Para un mejor entendimiento del fen\'omeno del ruido en las se\~nales, dividimos
el an\'alisis en distintos casos de ruido. Por ejemplo decidimos tomar al ruido
blanco como un caso de testeo importante debido a la \"falta de patrones\". 

En este trabajo decidimos experimentar con instrucciones del compilador que
permiten ejecutar c\'odigo en paralelo. Esta variante ayuda a bajar tiempos de
corrida con matrices de tama\~o considerable.

\index{Desarrollo!Herramientas}
\subsection{Herramientas}

Para la realizaci\'on de los experimentos se utiliz\'o C++ como lenguaje de
programaci\'on para los algoritmos principales. Esta elecci\'on se apoya en
nuestro conocimiento del lenguaje, la eficiencia que aporta y las estructuras de
datos que proporciona, haciendo simple y clara la implementaci\'on de la
resoluci\'on del problema. Como fue comentado anteriormente, fue utilizada la
directiva ``\#pragma omp parallel'' en ciertas porciones de c\'odigo
paralelizables, con el fin de reducir el tiempo de ejecuci\'on donde es posible.

El an\'alisis de los resultados y el ploteo de los gr\'aficos se realiz\'o
mediante el uso del entorno \href{http://www.gnu.org/software/octave/}{GNU
Octave}, libre y compatible con MATLAB.

El control de los experimentos y las ``recetas'' de las experiencias se ven
reflejadas en scripts de shell que levantan archivos de datos con tests
preprogramdos, ejecutan los algoritmos y luego analizan los resultados,
resumiendolos en output en forma de informaci\'on tabular y gr\'aficos.
La existencia de un Makefile fue clave a la hora de automatizar tareas de
compilaci\'on y corrida de tests.

El informe fue realizado mediante la utilizaci\'on de \href{http://www.latex-project.org/}{LaTeX},
 manteniendo la estructura de informe del Trabajo Pr\'actico N\'umero 1 y
permitiendo expresar estructuras y f\'ormulas matem\'aticas de una forma simple.

\subsection{Desarrollo del experimento de reducci\'on de ruido}

Para reducir el ruido utilizamos tres tipos de filtros: Filtro cero, filtro exponencial y filtro promediador.
En los tres casos empezamos a filtrar las se\~nales desde el 5\% (es decir, no procesamos el 5\% de se\~nales
de menor frecuencia) ya que experimentando encontramos que en ese 5\% se encuentra generalmente gran parte de
la informaci\'on, y en proporci\'on muy poco ruido.

En el filtro cero tomamos la se\~nal con el mayor coeficiente (en m\'odulo) y eliminamos todas las se\~nales que sean
al menos el 50\% de la se\~nal de mayor coeficiente en m\'odulo. Esto lo hicimos porque cre\'iamos que las se\~nales
con coeficientes m\'as altos eran las m\'as ruidosas.
El primer filtro que aplicamos fue el filtro cero sin descartar el primer 5\% y los resultados no fueron muy buenos.
Cuando se nos ocurrio agregarle a la implementaci\'on la omisi\'on del primer 5\% a la hora de filtrar vimos que los 
resultados fueron mucho mejores.

El filtro exponencial reemlaza los coeficientes de las se\~nales que son m\'as grandes y dividimos los coeficientes de su entorno por 1.2 elevado a 20 menos la distancia a la se\~nal, esto es, si el coeficiente es de la se\~nal n\'umero 35, entonces dividimos al coeficiente de 35 por 1.2 a la 20, al de 34 y el de 36 por 1.2 a la 19, y as\'i sucesivamente hasta dividir por 1.2 a la 1.

El filtro promediador lo que hace es distribuir los coeficientes m\'as grandes entre las se\~nales vecinas, es decir, si el coeficiente n\'umero 20 es muy grande, le ''pasa'' parte del coeficiente a los coeficientes 18, 19, 21 y 22. Para ser exactos, el coeficiente se multiplica por 0.3 y le sumamos a sus vecinos inmediatos 0.2 por el coeficiente y a los coeficientes a distancia 2, 0.15 por el coeficiente.

La idea del filtro promediador surge de la observaci\'on de la aplicaci\'on de
los otros en el caso del ruido blanco. Para poder afrontar casos con ruido en
distintas frecuencias muy distintas al mismo tiempo necesitamos obtener un
contexto mayor en cada punto para saber como manejarlos en relaci\'on con sus
vecinos.

Experimentalmente pudimos observar que el filtro cero es bueno, pero el filtro exponencial y el promediador son bastante mejores. Para poder obtener un mejor filtrado del ruido, decidimos aplicar primero el filtro exponencial y luego el promediador, lo que result\'o en un filtrado muy bueno dando resultados muy parecidos a la se\~nal original.
