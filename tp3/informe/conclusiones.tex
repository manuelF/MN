\section{Conclusiones}

Este trabajo nos ha dado la oportunidad de aprender varias lecciones.

\subsection{Herramientas de desarrollo \'agil}

El uso de Matlab ha sido fundamental a la hora de entender el problema.
Es muy sencillo perderse en los detalles de la implementaci\'on y no ver que
problema se quiere resolver. Una herramienta din\'amica donde se pueden visualizar
en una pasada las factorizaciones de una matriz, las dimensiones y las cuentas
de forma matricial sin que importen los detalles del c\'alculo.

Ayud\'o tambi\'en que la velocidad de procesamiento de Matlab es inesperadamente
r\'apida, por lo que se pudo experimentar r\'apidamente con matrices de covarianza
de pocas y de muchas im\'agenes indistintamente. El costo que se paga por esto es
no saber como es la implementaci\'on a bajo nivel y no tener control sobre los datos generados.
En particular, por ejemplo, si usamos la funci\'on \texttt{svd} de Matlab, se
generan autom\'aticamente las 3 matrices de la factorizaci\'on, pero solamente nos interesa
la $V^t$ (de 784 x 784 elementos). Sin embargo, se genera y guarda tambien la $U$ (de hasta
60000 x 60000 elementos, ~26 Gb) y $\Sigma$ (de hasta 60000 x 784 elementos, ~358Mb), llenando innecesariamente
la mem\'oria RAM, y impidiendonos operar con a\'un m\'as im\'agenes usando este mecanismo.

Con las correcciones del TP3, pudimos aprender que si hubieramos calculado la version \texttt{'econ'}
de \texttt{svd}, hubieramos podido los experimentos de casi cualquier tama\~no que hubieramos querido.
Esto realza a\'un m\'as la importancia del conocimiento de las herramientas de desarrollo a la hora
de enfrentarse a problemas.

La utilizaci\'on de implementaciones confiables sirvi\'o tambi\'en para apoyar nuestros
c\'alculos y tener nociones precisas de posibles errores de implementaci\'on
realizando an\'alisis comparativo de resultados.

\subsection{Importancia de los autovectores}

Leyendo la literatura de m\'etodos computacionales aplicados al \'algebra lineal, es
imposible evitar toparse constantemente con los autovectores y autovalores. Su
uso atraviesa todas las \'areas de los m\'etodos num\'ericos. Son cruciales
en los motores gr\'aficos, a la hora de caracterizar transformaciones t\'ipicas.
Aparecen en las ciencias f\'isicas como formas de resolver problemas de movimiento,
vibraci\'on, y fuerzas mec\'anicas.

El uso dado en este trabajo fue para obtener caraterizaciones de los datos utilizando la
descomposici\'on en autovectores, y de esa manera poder correlacionar distintos puntos que
tengan diferencias dif\'iciles de describir. Aprendimos que la t\'ecnica de descomponer en autovectores
es fundamental a la hora de resolver problemas \textit{fuzzy}, donde
el an\'alisis de componentes principales puede traer a la luz la verdadera variabilidad de los datos.


\subsection{Mecanismo de reconocimiento}

Las t\'ecnicas que utilizamos para experimentar reconocimiento nos dieron
la pauta de que este es el campo que m\'as se puede seguir explorando, sea rastreando la literatura
correspondiente como para intentar innovar con alg\'un m\'etodo \'unico e interesante.

Lo que fue realmente notable es que el uso de ideas que nosotros cre\'iamos m\'as ``inteligentes'',
en realidad resultaron ser peor que la m\'as \textit{naif}, la distancia euclidia a las medias, para
casi cualquier cantidad de componentes principales.

Esto nos lleva a pensar que el centro de masas de las im\'agenes de entrenamiento seg\'un lo calculamos
nosotros (siguiendo el enunciado y [Sirovich87]) es tal vez una forma \'optima de reconocer
dada la creaci\'on de la matriz de transformaci\'on (la matriz $V$). Si se generase de otra manera esa
transformaci\'on $V^t$, creemos que puede haber m\'etodos diferentes que logren un buen reconocimiento tambi\'en.


\subsection{M\'as puntos de datos no implican mejores resultados}

La cantidad de im\'agenes de entrenamiento que debi\'eramos usar fue un motivo de
discusi\'on e hip\'otesis dentro del grupo.
Claramente quisimos usar todas las que pudieramos, intentando emplear al m\'aximo el
uso del corpus brindado. La contrapartida fue el tiempo de ejecuci\'on. Incluso habiendo implementado
la factorizaci\'on de HouseHolder \'optimo (O($n^3$)), resultaba inaceptable que tardara mas de 24hs
el c\'alculo de los autovectores de la matriz de covarianza con un $\epsilon = 1e-5$. Implementamos 
directivas b\'asicas de paralelizaci\'on que pudieran bajar este runtime (sin cambiar el orden de complejidad del problema).
Logramos un speedup de un poco m\'as de 2x con respecto al c\'odigo serial, usando
un procesador con 8 threads, pero el factor c\'ubico predomina mucho todav\'ia.

Logramos calcular matrices de covarianza para un poco m\'as de 30000 im\'agenes de entrenamiento, pero cuando
corrimos los experimentos, el uso de las matrices generadas con m\'as im\'agenes superadas las 30000 no aport\'o casi a un mejor reconocimiento.
Viendo esto, decidimos que, al menos con este corpus, no ibamos a lograr captar m\'as detalles del problema,
asi que no seguimos intentado crecer en tama\~no.

Nuestra hip\'otesis sostiene que esto es as\'i porque en el corpus est\'a apropiadamente
representado cada integrante (d\'igito). Creemos que en problemas de
reconocimiento de rostros, por ejemplo, con muy pocas muestras de cada integrante,
ser\'ia esencial emplear el corpus al m\'aximo (no necesariamente usarlo todo, pero seleccionar los
integrantes inteligentemente) para poder tener todos los detalles posibles para hacer un mejor an\'alisis.
