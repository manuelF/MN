\section{Conclusiones}

Este trabajo nos ha dado la oportunidad de aprender varias lecciones.

\subsection{Herramientas de desarrollo rapido}

El uso de Matlab ha sido fundamental a la hora de entender el problema.
Es muy sencillo perderse en los detalles de la implementaci\'on y no ver que
problema se quiere resolver. Una herramienta din\'amica donde se pueden visualizar
en una pasada las factorizaciones de una matriz, las dimensiones y las cuentas 
de forma matricial sin que importen los detalles del calculo.

Ayud\'o tambien que la velocidad de procesamiento de MATLAB es inesperadamente
r\'apida, por lo que se pudo experimentar rapidamente con matrices de covarianza
de pocas y de muchas im\'agenes indistintamente. El costo que se paga por esto es
no saber como lo esta resolviendo por atras y no tener control sobre los datos generados.
En particular, por ejemplo, si usamos la funci\'on \texttt{svd} de MATLAB, se genera
autom\'aticamente las 3 matrices de la factorizaci\'on, pero solamente nos interesa
la V' (de 784 x 784 elementos). Sin embargo, se genera y guarda tambien la U ( de hasta
60000 x 60000 elementos, ~26Gb) y S (de hasta 60000 x 784 elementos, ~358Mb), llenando innecesariamente
la mem\'oria RAM, y impidiendonos operar con aun m\'as im\'agenes.


\subsection{Importancia de los autovectores}

Leyendo la literatura de m\'etodos computaciones aplicados al algebra lineal, es
imposible evitar toparse constantemente con los autovectores y autovalores. Su
uso atravieza todas las \'areas de los m\'etodos num\'ericos. Son cruciales
en los motores gr\'aficos, a la hora de caracterizar transformaciones t\'ipicas.
Aparecen en las ciencias f\'isicas como formas de resolver problemas de movimiento,
vibraci\'on, y fuerzas mec\'anicas.

El uso dado en este trabajo fue para obtener caraterizaciones de los datos utilizando la 
descomposici\'on en autovectores, y de esa manera poder correlacionar distintos puntos que
tengan diferencias dificiles de describir. Aprendimos que la t\'ecnica de descomponer en autovectores 
es fundamental para tener en cuenta a la hora de resolver problemas \textit{fuzzy}, donde
el an\'alisis de componentes principales puede traer a luz la verdadera variabilidad de los datos.


\subsection{Mecanismo de reconocimiento}
Idea sencilla norma 2 anda mejor que cosas mas rebuscadas.

\subsection{M\'as puntos de datos, no implican mejores respuestas}
