\documentclass[11pt, a4paper]{article}
\usepackage{a4wide}
\usepackage{amsmath}
\usepackage{amsfonts}
\usepackage{graphicx}
\usepackage[ruled,vlined]{algorithm2e}

\parskip = 11pt

\newcommand{\real}{\mathbb{R}}

\begin{document}


\begin{centering}
\large\bf Laboratorio de M\'etodos Num\'ericos - Primer Cuatrimestre 2013 \\
\large\bf Trabajo Pr\'actico N\'umero 3: OCR+SVD\\
\end{centering}


\vskip 0.5 cm
\hrule
\vskip 0.1 cm

{\bf Introducci\'on}

El reconocimiento \'optico de caracteres (OCR, por sus siglas en ingl\'es) es el proceso por el cual se traducen o convierten im\'agenes de d\'igitos o caracteres (sean \'estos manuscritos o de alguna tipograf\'ia especial) a un formato representable en nuestra computadora (por ejemplo, ASCII). Esta tarea puede ser m\'as sencilla (por ejemplo, cuando tratamos de determinar el texto escrito en una versi\'on escaneada a buena resoluci\'on de un libro) o tornarse casi imposible (recetas indescifrables de m\'edicos, algunos parciales manuscritos de alumnos de m\'etodos num\'ericos, etc).

El objetivo del trabajo pr\'actico es implementar un m\'etodo de reconocimiento de d\'igitos manuscritos basado en la descomposici\'on en valores singulares, y analizar emp\'iricamente los par\'ametros principales del m\'etodo.

Como instancias de entrenamiento, se tiene un conjunto de $n$ im\'agenes de d\'igitos ma\-nus\-cri\-tos en escala de grises del mismo tama\~no y resoluci\'on (varias im\'agenes de cada d\'igito). Cada una de estas im\'agenes sabemos a qu\'e d\'igito se corresponde.
En este trabajo consideraremos la popular base de datos MNIST, utilizada como referencia en esta \'area de investigaci\'on\footnote{\texttt{http://yann.lecun.com/exdb/mnist/}}. 

Para $i = 1,\ldots, n$, sea $x_i \in \real^{m}$ la $i$-\'esima imagen de nuestra base de datos almacenada por filas en un vector, y sea $\mu = (x_1 + \ldots + x_n)/n$ el promedio de las im\'agenes. Definimos $X\in\real^{n\times m}$ como la matriz que contiene en la $i$-\'esima fila al vector $(x_i - \mu)^{t}/\sqrt{n-1}$, y $$X=U \Sigma V^t$$ a su descomposici\'on en valores singulares, con $U\in\real^{n\times n}$ y $V\in\real^{m\times m}$ matrices ortogonales, y $\Sigma\in\real^{n\times m}$ la matriz diagonal conteniendo en la posici\'on $(i,i)$ al $i$-\'esimo valor singular $\sigma_i$.
Siendo $v_i$ la columna $i$ de $V$, definimos para $i = 1,\ldots,n$ la \textsl{transformaci\'on caracter\'istica} del d\'igito $x_{i}$ como el vector $\mathbf{tc}(x_i) = (v_1^t\, x_i, v_2^t\, x_i,\ldots,v_k^t\, x_i) \in\real^k$, donde $k \in\{1,\ldots,m\}$ es un par\'ametro de la implementaci\'on. Este proceso corresponde a extraer las $k$ primeras \textit{componentes principales} de cada imagen. La intenci\'on es que $\mathbf{tc}(x_i)$ resuma la informaci\'on m\'as relevante de la imagen, descartando los detalles o las zonas que no aportan rasgos distintivos.


Dada una nueva imagen $x$ de un d\'igito manuscrito, que no se encuentra en el conjunto inicial de im\'agenes de entrenamiento, el problema de reconocimiento consiste en determinar a qu\'e d\'igito corresponde. Para esto, se calcula $\mathbf{tc}(x)$ y se compara con $\mathbf{tc}(x_i)$, para $i = 1,\ldots, n$.


{\bf Enunciado}

Se pide implementar un programa que lea desde archivos las im\'agenes de entrenamiento de distintos d\'igitos manuscritos y que, utilizando la descomposici\'on en valores singulares, se calcule la transformaci\'on caracter\'istica de acuerdo con la descripci\'on anterior. Para ello se deber\'a implementar alg\'un m\'etodo de estimaci\'on de autovalores/autovectores. Dada una nueva imagen de un d\'igito manuscrito, el programa deber\'a determinar a qu\'e d\'igito co\-rres\-pon\-de.
El formato de los archivos de entrada y salida queda a elecci\'on del grupo. Si no usan un entorno de desarrollo que incluya bibliotecas para la lectura de archivos de im\'agenes, sugerimos que utilicen im\'agenes en formato \textsc{Raw}. 


Se deber\'an realizar experimentos para medir la efectividad del reconocimiento, analizando tanto la influencia de la cantidad $k$ de componentes principales seleccionadas como la influencia de la precisi\'on en el c\'alculo de los autovalores.

\vskip 0.5 cm
\hrule
\vskip 0.1 cm

{\bf Fecha de entrega} 
\begin{itemize}
\item \textsl{Formato electr\'onico:} viernes 21 de junio de 2013, \underline{hasta las 23:59 hs.}, enviando el trabajo (informe+c\'odigo) a \texttt{metnum.lab@gmail.com}. El subject del email debe comenzar con el texto \verb|[TP3]| seguido de la lista de apellidos de los integrantes del grupo. 
\item \textsl{Formato f\'isico:} lunes 24 de junio de 2013, de 18 a 20hs (en la clase de la pr\'actica).
\end{itemize}


\end{document}
