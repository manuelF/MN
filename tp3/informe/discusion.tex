\section{Discusion}

\subsection{Tama\~no de los corpus de entrenamiento }
Un punto fundamental del trabajo consistio en el calculo de la matriz de autovectores
de la covarianza de las imagenes. En las secciones anteriores se discutio su construcci\'on
y su calculo. Ac\'a comprobamos la relevancia del tama\~no del corpus usado a la hora
de generarla.

Experimentamos tomando 1000, 5000, 15000 y 30000 imagenes (m\'as no usamos porque era
realmente excesivo el tiempo de c\'alculo. Con m\'as tiempo hubieramos podido paralelizar
y poder correr en un cluster o en GPGPU.). Como se ve en las figuras \ref{fig:HR1kcv} hasta \ref{fig:HR30kcv},
no hay mejores resultados utilizando matrices de transformaci\'on basadas en menos im\'agenes, por lo que
si ya hemos generado una grande, conviene seguir usandola mientras tengamos que resolver
el mismo problema.

El \'unico punto delicado en estos casos, de entrenamiento offline, es tener cuidado
de no mezclar datos de entrenamiento con datos de test. Esto podria ocasionar que
los hitrates sean excesivamente m\'as altos. Es preferible usar matrices generadas
con menos puntos de datos que con m\'as pero ``contaminados''.


\subsection{Clasificaci\'on de los me\'todos empleados}
Los metodos que implementamos fueron b\'asicamente 3. La norma 2 de distancia a
los centros de masas de cada d\'igito, la mejor de las 3 normas (norma 2, norma 1 y norma infinito)
hacia los centros de masas y el m\'as repetido de los 100 vecinos m\'as cercanos al punto.

El m\'etodo de la norma dos, es sorprendentemente bueno. Cuando se superan las 30
columnas mayormente el hitrate estabiliza en un valor razonablmente alto. Si lo
comparamos contra norma 1, es un poco mas alto el hitrate pero no excesivo. Ahora
cuando lo miramos contra norma infinito, vemos que la norma 2 es un m\'etodo
claramente superior. Como la norma infinito se define como el m\'aximo componente,
es de esperar que alguna de las componentes de los puntos transformados tenga mayor
peso que otras, por lo que no es extra\~no ver que aumentando la cantidad de componentes
no varia en nada la detecci\'on.

Viendo que las mediciones de norma 1 y norma infinito a veces discrepaban de lo que
decia norma 2, consideramos utilizarlas tambien para hacer otro m\'etodo, el de \textit{mejor
de 3}. Es decir, si coinciden 2 de las 3 normas, la usamos (por m\'as que coincidan la
infinito y la norma 1). Si no hay ningun match, hacemos fallback en norma 2. En este caso
nos sorprendio que los resultados no mejoraban con respecto al uso \'unico de norma 2.
Esto se puede ver en \ref{fig:HR30kcv} y en \ref{fig:HRD30kcv-bo3}, en comparaci\'on con
el de norma 2 en \ref{fig:HRD30kcv-n2}.

Notablemente, la idea de conseguir el m\'as repetido de los 100 vecinos m\'as cercanos
parece tener sentido, pero experimentalmente no es para nada buena, en varios aspectos.
No provee una mejor detecci\'on, de hecho es consistentemente peor para cualquier $k>15$.
Su runtime ademas es enorme, ya que debe hacer norma 2 para todos los otros puntos que
deseemos considerar, pueden andar en el orden de miles tal vez. En comparacion, el m\'etodo
de la norma 2 anterior solamente comprueba contra 10 puntos (los centros de masas).


\subsection{Detecci\'on por d\'igito}
Calcular el Hitrate por d\'igito se nos ocurri\'o tarde, pero son resultados
sumamente interesantes. De la figura \ref{fig:HRD30kcv-n2} a \ref{fig:HRD30kcv-dist100}
podemos apreciar considerablemente que, dentro de
la detecci\'on de d\'igitos, hay algunos que son m\'as dificiles que otros.

Por un lado, el d\'igito 1 es el m\'as facil de reconocer. Nuestra hip\'otesis
es que no es normal dibujarlo con m\'as de un trazo, y eso simplifica mucho la
descomposici\'on en componentes principales, al tener que capturar unicamente ese
fen\'omeno.

Por el otro, el d\'igito 5 es much\'isimo m\'as complicado de reconocer, teniendo
hitrates de una fracci\'on de los otros d\'igitos, incluso con m\'as componentes
principales. Esto se debe al fen\'omeno de que es a veces sut\'il la diferencia
entre los d\'igitos como 2 y 8.

En las figuras \ref{fig:HM30kcv-k5} hasta \ref{fig:HM30kcv-k50}, podemos apreciar
en un formato HeatMap como varian los aciertos. Usamos norma 2 para la detecci\'on ac\'a
ya que antes pudimos apreciar que era m\'etodo m\'as efectivo. Vemos que cuando tomamos pocas
columnas, la diagonal del heatmap (que es el acierto) esta muy difusa, y vemos tambien
otros puntos fuera de la diagonal que tienen un valor bastante alto. Estos son los d\'igitos
que se confunden. Podemos apreciar, por ejemplo, en la figura \ref{fig:HM30kcv-k5}, que
los d\'gitos que m\'as se pueden confundir, con 5 columnas, son el 8 por el 2 y el 4 por el 1.

Seria interesante poder combinar este mecanismo de detecci\'on con algun otro basado en un
criterio diferente, pero que tenga m\'as poder clasificar entre los d\'igitos problematicos
previamente comentados.


\subsection{Cantidad de columnas a tomar para la comparaci\'on}
Un punto clave de este trabajo fue la determinaci\'on de la cantidad de valores que debemos
utilizar para comparar im\'agenes. En todas las figuras pudimos observar, en mayor o menor
grado, un crecimiento del hitrate con respecto a la cantidad de columnas tomadas.
Obviamente, tomar mas elementos lleva a una mejor caracterizaci\'on', pero se vuelve un problema
de que uno esta b\'asicamente identificando una im\'agen pixel por pixel. Lo interesante de un
sistema de reconocimiento de im\'agenes es tener una alto grado de detecci\'on utilizando la menor
cantidad de recursos.


Se puede apreciar tanto en las figuras \ref{fig:HM30kcv-k5} hasta \ref{fig:HM30kcv-k100} como
en \ref{fig:HR30kcv}, que hay una diferencia considerable en la elecci\'on del m\'etodo usado,
pero sobre todo en la cantidad de columnas elegidas. Sin embargo, a partir de ciertos valores
ya el crecimiento del hitrate es muy lento ($K=50$ por ejemplo). Esto en realidad importa poco
si la cantidad de im\'agenes que tenemos que detectar no es grande, la diferencia de cantidad
de operaciones que hay entre $K=50$ y $K=150$ no es enorme. Dicho esto, hay que tener en cuenta que
en sistemas \textit{real-time} esto puede ser excesivamente costoso. Seg\'un nuestros experimentos,
se deberia hacer un analisis de los requerimientos de forma cuidadosa y ver si el \textit{trade-off} tiempo/precisi\'on
es aceptable o no.