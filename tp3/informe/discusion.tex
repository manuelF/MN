\section{Discusion}


\subsection{Clasificaci\'on de los me\'todos empleados}
Los metodos que implementamos fueron b\'asicamente 3. La norma 2 de distancia a
los centros de masas de cada d\'igito, la mejor de las 3 normas (norma 2, norma 1 y norma infinito)
hacia los centros de masas y el m\'as repetido de los 100 vecinos m\'as cercanos al punto.

El m\'etodo de la norma dos, es sorprendentemente bueno. Cuando se superan las 30
columnas mayormente el hitrate estabiliza en un valor razonablmente alto. Si lo
comparamos contra norma 1, es un poco mas alto el hitrate pero no excesivo. Ahora
cuando lo miramos contra norma infinito, vemos que la norma 2 es un m\'etodo
claramente superior. Como la norma infinito se define como el m\'aximo componente,
es de esperar que alguna de las componentes de los puntos transformados tenga mayor
peso que otras, por lo que no es extra\~no ver que aumentando la cantidad de componentes
no varia en nada la detecci\'on.

Viendo que las mediciones de norma 1 y norma infinito a veces discrepaban de lo que
decia norma 2, consideramos utilizarlas tambien para hacer otro m\'etodo, el de \textit{mejor
de 3}. Es decir, si coinciden 2 de las 3 normas, la usamos (por m\'as que coincidan la
infinito y la norma 1). Si no hay ningun match, hacemos fallback en norma 2. En este caso
nos sorprendio que los resultados no mejoraban con respecto al uso \'unico de norma 2.
Esto se puede ver en \ref{fig:HR30kcv} y en \ref{fig:HRD30kcv-bo3}, en comparaci\'on con
el de norma 2 en \ref{fig:HRD30kcv-n2}.

Notablemente, la idea de conseguir el m\'as repetido de los 100 vecinos m\'as cercanos
parece tener sentido, pero experimentalmente no es para nada buena, en varios aspectos.
No provee una mejor detecci\'on, de hecho es consistentemente peor para cualquier $k>15$.
Su runtime ademas es enorme, ya que debe hacer norma 2 para todos los otros puntos que
deseemos considerar, pueden andar en el orden de miles tal vez. En comparacion, el m\'etodo
de la norma 2 anterior solamente comprueba contra 10 puntos (los centros de masas).


\subsection{Tama\~no de los corpus de entrenamiento }
Un punto fundamental del trabajo consistio en el calculo de la matriz de autovectores
de la covarianza de las imagenes. En las secciones anteriores se discutio su construcci\'on
y su calculo. Ac\'a comprobamos la relevancia del tama\~no del corpus usado a la hora
de generarla.

Experimentamos tomando 1000, 5000, 15000 y 30000 imagenes (m\'as no usamos porque era
realmente excesivo el tiempo de c\'alculo. Con m\'as tiempo hubieramos podido paralelizar
y poder correr en un cluster o en otros dispositivos esto.)



\subsection{Detecci\'on por d\'igito}
Calcular el Hitrate por digito se nos ocurri\'o tarde, pero son resultados
sumamente interesantes. Ac\'a podemos apreciar considerablemente que, dentro de
la detecci\'on de d\'igitos, hay problemas m\'as dificiles que otros.

Por un lado, el digito 1 es el m\'as facil de reconocer. Nuestra hip\'otesis
es que no es normal dibujarlo con m\'as de un trazo, y eso simplifica mucho la
descomposici\'on en componentes principales, al tener que capturar unicamente ese
fen\'omeno.

Por el otro, el digito 5 es much\'isimo m\'as complicado de reconocer, teniendo
hitrates de una fracci\'on de los otros digitos, incluso con m\'as componentes
principales. Esto se debe al fenomeno de que es a veces sut\'il la diferencia
entre los d\'igitos 5, 3 y 8.

%%HACER GRAFICO DE HITRATE CONFUNDIDO ENTRE 5 y 3.

