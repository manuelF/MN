\section{Introducci\'on}

En el presente trabajo vamos a estudiar el mecanismo de reconocimiento autom\'atico de 
im\'agenes mediante el an\'alisis de componentes principales. 

Este \'analisis fue primeramente llevado a cabo en [Sirovich87] y [Turk91]. Estos trabajos
reconocen el an\'alisis de componentes principales como herramientas te\'oricas poderosas
a la hora de buscar caracterizaciones autom\'aticas en im\'agenes (en  sus casos, rostros).

La t\'ecnica se basa fundamentalmente en el hecho que las im\'agenes en $\mathbb{R}^{n * m}$ que queremos reconocer
no son variables aleatorias uniformemente distribuidas, sino que hay funci\'on caja negra que genera
estas im\'agenes (en nuestro caso, que los digitos se dibujan de maneras parecidas, como siguiendo un trazo mental).

Como hemos podido observar en el caso de, por ejemplo, la esteganografia, si a una matriz le alteramos las
componentes principales de menor magnitud, la matriz reconstruida se altera bastante poco. Esto nos da
la pauta de que aportan menor cantidad de ``informaci\'on'' estas componentes que las de mayor magnit\'ud.

Esto nos indica que para poder reconstruir las im\'agenes de la forma m\'as fiel posible,
deberiamos perder la menor cantidad de informaci\'on posible, o sea, quedarnos con la mayor cantidad
de componentes principales de la im\'agen. 

Lo que buscamos nosotros es, dado una im\'agen de un digito, identificar cual es. La hipotesis
que asumimos a lo largo de todo este trabajo es que las distintas im\'agenes de un mismo digito van 
a tener ``similares'' componentes principales. Lo que vamos a hacer es buscar, dada una im\'agen cualquiera
ver a cual se parece m\'as para poder identificarlo.

Para hacer esto, vamos a tener que generar una m\'atriz que contenga las covarianzas de todas las
im\'agenes de entrenamiento. Esto se obtiene con una matriz $M \in \mathbb{R}^{n * m}$, con $n$ cantidad
de im\'agenes de entrenamiento y $m$ cantidad de pixeles por im\'agen. Luego, se define M como:

$M_i = \frac{(Imagen_i - \mu_{Imagenes})}{\sqrt(n-)}$

Para obtener la covarianza, deberiamos obtener $M^t M$, esto va a resultar en una matriz $M' \in \mathbb{R}^{m * m}$.
De esta matriz $M'$ debemos conseguir los autovectores, para poder hacer al descomposici\'on apropiadamente.
Los autovectores de $M'$ tambien se pueden obtener como la matriz $V^t$ de la factorizaci\'on SVD, $A=U\Sigma V^t$.

Una vez obtenida la matriz $V^t$ de autovectores de la covarianza, se puede proceder a los distintos m\'etodos de
identificaci\'on de im\'agenes.
