    \section{Introducci\'on te\'orica}

Previo al detalle del experimento realizado se detallan una serie de conceptos te\'oricos de necesario
conocimiento para el entendimiento del trabajo.

\subsection{Reconocimiento \'optico de caracteres (OCR)}

El reconocimiento \'optico de caracteres es el proceso de conversi\'on de caracteres de un formato que puede
ser escritura a mano alzada, im\'agenes de libros u otros formatos complejos de entender para una computadora
a otro que sea reconocible por \'esta.

Este \'area combina disciplinas como inteligencia artificial, an\'alisis num\'erico y machine learning. Entre
las aplicaciones de OCR se encuentran:

\begin{itemize}
  \item Digitalizaci\'on de libros y documentos
  \item Reconocimiento de tarjetas de cr\'edito y facturas
  \item Generaci\'on de im\'agenes a partir de documentos digitales
  \item Resoluci\'on autom\'atica de CAPTCHA
\end{itemize}

\subsection{Matriz de covarianza}

Dado un conjunto de datos de $n$ componentes, se define la matriz de covarianza a la matriz de $n*n$ que
contiene en su elemento $(i, j)$ la covarianza entre las componentes $i$ y $j$ de la matriz original.

Siendo la covarianza entre 2 componentes definida como:
\\

\centerline{$\Sigma_{ij} = \mathrm{cov}(X_i, X_j) = \mathrm{E}\begin{bmatrix}(X_i - \mu_i)(X_j - \mu_j) \end{bmatrix}$ con $\mu_i = \mathrm{E}(X_i)$.}

\subsection{Descomposici\'on en valores singulares (SVD)}

La descomposici\'on en valores singulares de una matriz $\mathbf{M} \in \mathbb{R}^{m \times n}$ es una factorizaci\'on de la forma $\mathbf{M} =
\mathbf{U} \boldsymbol{\Sigma} \mathbf{V}^T$ donde:

\begin{itemize}
  \item $\mathbf{U} \in \mathbb{R}^{m \times m}$ es una matriz ortogonal.
  \item $\boldsymbol{\Sigma} \in \mathbb{R}^{m \times n}$ es una matriz diagonal con elementos no negativos y los elementos de la
diagonal constituyen los valores singulares de la matriz original.
  \item $\mathbf{V} \in \mathbb{R}^{n \times n}$ es una matriz ortogonal.
\end{itemize}

La descomposici\'on en valores singulares tiene diversos usos entre los que se encuentra la resoluci\'on de
sistemas lineales y la b\'usqueda de matrices pseudoinversas.

\subsection{Introducci\'on al trabajo}

El siguiente trabajo consiste en el estudio del mecanismo de reconocimiento autom\'atico
en im\'agenes mediante el an\'alisis de componentes principales.

Este tipo de an\'alisis proviene del trabajo llevado acabo en [Sirovich89] y [Turk91]. Estos
trabajos reconocen al an\'alisis de componentes principales como herramientas te\'oricas poderosas
a la hora de buscar caracterizaciones autom\'aticas en im\'agenes. Estos papers se enfocan en el
an\'alisis de rostros, una disciplina con un grado de complejidad superior al estudiado en este
trabajo.

La t\'ecnica se basa fundamentalmente en el hecho de que las im\'agenes en $\mathbb{R}^{n \times m}$ las cuales se
quiere reconocer no son variables aleatorias uniformemente distribuidas, sino
que existe una funci\'on ``caja negra'' que genera estas im\'agenes (en nuestro 
caso, que los d\'igitos se dibujan de maneras parecidas, como siguiendo un trazo mental).

Como hemos podido observar en el caso de, por ejemplo, la esteganograf\'ia, si a una matriz se le alteran
las componentes principales de menor magnit\'ud, la alteraci\'on en la matriz reconstruida es baja.
Este fen\'omeno nos da la pauta de que estas componentes aportan menor cantidad de informaci\'on
que las de mayor magnit\'ud a la hora de representar la imagen.

Una conclusi\'on sacada a priori indica que para reconstruir las im\'agenes de la forma m\'as fiel posible,
y consecuentemente perder la menor cantidad de informaci\'on posible, es menester conservar la mayor cantidad
de componentes principales de la imagen.

El objetivo principal del trabajo es, dada una imagen conteniendo la representaci\'on de un d\'igito,
identificar a cual corresponde. La hip\'otesis asumida a lo largo del trabajo se
resume a que las distintas im\'agenes de un mismo d\'igito poseen caracter\'isticas 
similares en sus componentes principales. El procedimiento se encarga, dada una imagen cualquiera
, de buscar similitudes con los datos de entrenamiento y realizar el matching correspondiente
para determinar a que d\'igito se parece m\'as y as\'i poder identificarlo.

El procedimiento consta, en t\'erminos generales, de la generaci\'on de una 
matriz de covarianza surgente de las im\'agenes de entrenamiento. 
Esto se obtiene con una matriz $M \in \mathbb{R}^{n \times m}$, con $n$ cantidad
de im\'agenes de entrenamiento y $m$ cantidad de pixeles por imagen. Luego, se define M como:
\\

\centerline{$M_i = \frac{(imagen_i - \mu_{imagenes})}{\sqrt{n-1}}$}.
\\
Con $i$ una de las filas de la matriz.

Para obtener la matriz de covarianza, se debe calcular $M^t M$, esto resulta en una matriz $M' \in \mathbb{R}^{m \times m}$.
De esta matriz $M'$ es posible extraer sus autovectores, para poder hacer la descomposici\'on apropiadamente.
Los autovectores de $M'$ tambi\'en se pueden obtener como la matriz $V$ de la factorizaci\'on SVD, $A=U\Sigma V^t$.

Una vez obtenida la matriz $V$ de autovectores de la covarianza, se procede a uno de los diferentes m\'etodos de
identificaci\'on de im\'agenes.
