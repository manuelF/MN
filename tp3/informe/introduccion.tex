\section{Introducci\'on te\'orica}

Previo al detalle del experimento realizado se detallan una serie de conceptos te\'oricos de necesario
conocimeinto para el entendimiento del trabajo.

\subsection{Reconocimiento \'optico de caracteres (OCR)}

El reconocimiento \'optico de caracteres es el proceso de conversi\'on de caracteres de un formato que puede
ser escritura a mano alzada, im\'agenes de libros u otros formatos complejos de entender para una computadora
a otro que sea reconocible por esta.

Este area combina disciplinas como inteligencia articifial, an\'alisis num\'erico y machine learning. Entre
las aplicaciones de OCR se encuentran:

\begin{itemize}
  \item Digitalizaci\'on de libros y documentos
  \item Reconocimiento de tarjetas de cr\'edito y facturas
  \item Generaci\'on de im\'agenes a partir de documentos digitales
  \item Resoluci\'on autom\'atica de CAPTCHA
\end{itemize}

\subsection{Matriz de covarianza}

Dado un conjunto de datos de $n$ componentes, se define la matriz de covarianza a la matriz de $n*n$ que
contiene en su elemento $(i, j)$ la covarianza entre las componentes $i$ y $j$ de la matriz original.

Siendo la covarianza entre 2 componentes definida como:

$\Sigma_{ij} = \mathrm{cov}(X_i, X_j) = \mathrm{E}\begin{bmatrix} (X_i - \mu_i)(X_j - \mu_j) \end{bmatrix}$

con $\mu_i = \mathrm{E}(X_i)$.

\subsection{Descomposici\'on en valores singulares (SVD)}

La descomposici\'on en valores singulares de una matriz $\mathbf{M} \in \mathbb{R^{mxn}}$ es una factorizaci\'on de la forma $\mathbf{M} =
\mathbf{U} \boldsymbol{\Sigma} \mathbf{V}^T$ donde:

\begin{itemize}
  \item $\mathbf{U} \in \mathbb{R^{mxn}}$ con columnas ortogonales.
  \item $\boldsymbol{\Sigma} \in \mathbb{R^{nxn}}$ es una matriz diagonal con elementos no negativos y los elementos de la
diagonal constituyen los valores singulares de la matriz original.
  \item $\mathbf{V} \in \mathbb{R^{nxn}}$ es una matriz ortogonal.
\end{itemize}

La descomposici\'on en valores singulares tiene diversos usos entre los que se encuentra la resoluci\'on de
sistemas lineales y la b\'usqueda de matrices pseudoinversas.

\subsection{Introducci\'on al trabajo}

El siguiente trabajo consiste en el estudio del mecanismo de reconocimiento autom\'atico
en im\'agenes mediante el an\'alisis de componentes principales. Para simplificar la
experimentaci\'on se utilizan im\'agenes que representan tan solo 3 s\'imbolos distintos
(3, 5 y 8).

Este tipo de an\'alisis proviene del trabajo llevado acabo en [Sirovich89] y [Turk91]. Estos
trabajos reconocen al an\'alisis de componentes principales como herramientas te\'oricas poderosas
a la hora de buscar caracterizaciones autom\'aticas en im\'agenes. Estos papers se enfocan en el
an\'alisis de rostros, una disciplina con un grado de complejidad superior al estudiado en este
trabajo.

La t\'ecnica se basa fundamentalmente en el hecho que las im\'agenes en $\mathbb{R}^{n * m}$ las cuales se
quiere reconocer no son variables aleatorias uniformemente distribuidas, sino que existe una funci\'on caja
negra que genera estas im\'agenes (en nuestro caso, que los digitos se dibujan de maneras parecidas,
como siguiendo un trazo mental).

Como hemos podido observar en el caso de, por ejemplo, la esteganograf\'ia, si a una matriz se le alteran
las componentes principales de menor magnitud, la alteraci\'on en la matriz reconstruida es baja.
Este fen\'omeno nos la pauta de que estas componentes aportan menor cantidad de ``informaci\'on''
que las de mayor magnit\'ud a la hora de representar la im\'agen.

Una conclusi\'on sacada a priori indica que para reconstruir las im\'agenes de la forma m\'as fiel posible,
y consecuentemente perder la menor cantidad de informaci\'on posible, es menester conservar la mayor cantidad
de componentes principales de la im\'agen.

El objetivo principal del trabajo es, dada una im\'agen conteniendo la representaci\'on del d\'igito (3,5,8),
identificar a cual corresponde. La hipotesis asumida a lo largo del trabajo es que las distintas im\'agenes
de un mismo digito van poseen caracter\'isticas ``similares'' en sus componentes principales.
El procedimiento se encarga de buscar, dada una im\'agen cualquiera, y realizar el matching correspondiente
para determinar a que d\'igito parece m\'as y as\'i poder identificarlo.

El procedimiento consta, en t\'erminos generales, de la generaci\'on de una m\'atriz de covarianza surgente
de las im\'agenes de entrenamiento. Esto se obtiene con una matriz $M \in \mathbb{R}^{n * m}$, con $n$ cantidad
de im\'agenes de entrenamiento y $m$ cantidad de pixeles por im\'agen. Luego, se define M como:

$M_i = \frac{(Imagen_i - \mu_{Imagenes})}{\sqrt{n-1}}$

Para obtener la covarianza, se debe obtener $M^t M$, esto resulta en una matriz $M' \in \mathbb{R}^{m * m}$.
De esta matriz $M'$ es posible extraer los autovectores, para poder hacer al descomposici\'on apropiadamente.
Los autovectores de $M'$ tambien se pueden obtener como la matriz $V^t$ de la factorizaci\'on SVD, $A=U\Sigma V^t$.

Una vez obtenida la matriz $V^t$ de autovectores de la covarianza, se procede a uno de los diferentes m\'etodos de
identificaci\'on de im\'agenes.
