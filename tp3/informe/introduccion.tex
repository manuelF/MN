\section{Introducci\'on}

En el presente trabajo vamos a estudiar el mecanismo de reconocimiento autom\'atico de 
im\'agenes mediante el an\'alisis de componentes principales. 

Este \'analisis fue primeramente llevado a cabo en [Sirovich87] y [Turk91]. Estos trabajos
reconocen el an\'alisis de componentes principales como herramientas te\'oricas poderosas
a la hora de buscar caracterizaciones autom\'aticas en imagenes (en  sus casos, rostros).

La t\'ecnica se basa fundamentalmente en el hecho que las im\'agenes en $\mathbb{R}^{n * m}$ que queremos reconocer
no son variables aleatorias uniformemente distribuidas, sino que hay funci\'on caja negra que genera
estas imagenes (en nuestro caso, que los digitos se dibujan de maneras parecidas, como siguiendo un trazo mental).

Como hemos podido observar en el caso de, por ejemplo, la esteganografia, si a una matriz le alteramos las
componentes principales de menor magnitud, la matriz reconstruida se altera bastante poco. Esto nos da
la pauta de que aportan menor cantidad de ``informaci\'on'' estas componentes que las de mayor magnit\'ud.

Esto nos indica que para poder reconstruir las imagenes de la forma m\'as fiel posible,
deberiamos perder la menor cantidad de informaci\'on posible, o sea, quedarnos con la mayor cantidad
de componentes principales de la imagen. 

Lo que buscamos nosotros es, dado una im\'agen de un digito, identificar cual es. La hipotesis
que asumimos a lo largo de todo este trabajo es que las distintas im\'agenes de un mismo digito van 
a tener ``similares'' componentes principales. Lo que vamos a hacer es buscar, dada una im\'agen cualquiera
ver a cual se parece m\'as para poder identificarlo.
