\section{Resultados}

\subsection{Iteraciones del c\'alculo de autovectore}
El mecanismo de reconocimiento de im\'agenes esta basado en los autovectores
de la matriz de covarianza, como detallamos antes. Para obtenerlos, se aplica
el m\'etodo iterativo de la factorizacion QR reiterada, descripto anterioremente.
La condicion de corte de este m\'etodo esta basada en la suma de los elementos bajo
la diagonal. Estudiamos esta suma y la distribucion de los valores bajo la diagonal,
para intentar entender el comportamiento y definir una buena cota. 

\def \hrwidth {500pt}

\begin{figure}[H]
\begin {center}
\includegraphics[width=\hrwidth]{plots/SUM.png}
\end {center}
\caption{Suma de los elementos bajo la diagonal para 5000, 15000 y 30000 im\'agenes
en funcion de la cantidad de iteraciones de QR.}
\label{fig:SUM}
\end{figure}

\begin{figure}[H]
\begin {center}
\includegraphics[width=\hrwidth]{plots/PROM.png}
\end {center}
\caption{Promedio de los elementos bajo la diagonal para 5000, 15000 y 30000 im\'agenes
en funcion de la cantidad de iteraciones de QR.}
\label{fig:PROM}
\end{figure}


\subsection{Promedio de Reconocimiento}
La m\'etodologia de implementaci\'on fue la siguiente: Se construy\'o una
m\'atriz de covarianza utilizando los primeros $n$ im\'agenes del dataset
\texttt{Training Data}. Luego, se tomaban $30000-n$ im\'agenes que se descartaban; para poder tomar 
siempre las mismas $t$ im\'agenes siguientes como im\'agenes de test, con sus 
correspondientes labels. A estos se les aplicaron las t\'ecnicas
de detecci\'on que hemos detallado antes y fuimos variando los $k$ componentes principales 
de las tuplas que tomabamos para hacer las cuentas de distancia.
Los gr\'aficos de HitRate est\'an hechos en funci\'on de $k$.
El Hitrate se define como el porcentaje de aciertos en la detecci\'on sobre el total de 
experimentos realizados.

Tom\'amos siempre las mismas $t=500$ im\'agenes para todos los test, que representan
de manera proporcional a cada d\'igito.

Testeamos usando matrices de covarianza entrenadas con cantidad variable de im\'agenes.
\def \hrwidth {500pt}

\begin{figure}[H]
\begin {center}
\includegraphics[width=\hrwidth]{plots/hitrate-1kcv.png}
\end {center}
\caption{Hitrate de detecci\'on de 500 im\'agenes de acuerdo a los varios m\'etodos implementados
usando la matriz de covarianza entrenada con 1000 im\'agenes}
\label{fig:HR1kcv}
\end{figure}

\begin{figure}[H]
\begin {center}
\includegraphics[width=\hrwidth]{plots/hitrate-5kcv.png}
\end {center}
\caption{Hitrate de detecci\'on de 500 im\'agenes de acuerdo a los varios m\'etodos implementados
usando la matriz de covarianza entrenada con 5000 im\'agenes}
\label{fig:HR5kcv}
\end{figure}

\begin{figure}[H]
\begin {center}
\includegraphics[width=\hrwidth]{plots/hitrate-15kcv.png}
\end {center}
\caption{Hitrate de detecci\'on de 500 im\'agenes de acuerdo a los varios m\'etodos implementados
usando la matriz de covarianza entrenada con 15000 im\'agenes}
\label{fig:HR15kcv}
\end{figure}

\begin{figure}[H]
\begin {center}
\includegraphics[width=\hrwidth]{plots/hitrate-30kcv.png}
\end {center}
\caption{Hitrate de detecci\'on de 500 im\'agenes de acuerdo a los varios m\'etodos implementados
usando la matriz de covarianza entrenada con 30000 im\'agenes}
\label{fig:HR30kcv}
\end{figure}
%%%%%%%%%%%%%%%%%%%%%%%%%%%%%%%%%%%%%%%%%%%%%%%%%%%%%%%%%%%%%%%%

\subsection{Reconocimiento en funcion de iteraciones}
Una vez definido el m\'etodo de detecci\'on, nos interesa ver como varia el hitrate
de acuerdo a la matriz generada con distinta cantidad de iteraciones. En particular
utilizamos el m\'etodo de distancia a los promedios y 100 vecinos m\'as cercanos, todos
medidos usando norma 2.

\begin{figure}[H]
\begin {center}
\includegraphics[width=\hrwidth]{plots/HR-10-avg.png}
\end {center}
\caption{Hitrate en funci\'on de la cantidad de iteraciones en la generacion de la matriz
usando distancia a los promedios tomando 10 componentes principales}
\label{fig:HR10Avg}
\end{figure}


\begin{figure}[H]
\begin {center}
\includegraphics[width=\hrwidth]{plots/HR-100-avg.png}
\end {center}
\caption{Hitrate en funci\'on de la cantidad de iteraciones en la generacion de la matriz
usando distancia a los promedios tomando 100 componentes principales}
\label{fig:HR100Avg}
\end{figure}


\begin{figure}[H]
\begin {center}
\includegraphics[width=\hrwidth]{plots/HR-10-neig.png}
\end {center}
\caption{Hitrate en funci\'on de la cantidad de iteraciones en la generacion de la matriz
usando vecinos m\'as cercanos tomando 10 componentes principales}
\label{fig:HR10Neig}
\end{figure}


\begin{figure}[H]
\begin {center}
\includegraphics[width=\hrwidth]{plots/HR-100-neig.png}
\end {center}
\caption{Hitrate en funci\'on de la cantidad de iteraciones en la generacion de la matriz
usando vecinos m\'as cercanos tomando 100 componentes principales}
\label{fig:HR10Neig}
\end{figure}







%%%%%%%%%%%%%%%%%%%%%%%%%%%%%%%%%%%%%%%%%%%%%%%%%%%%%%%%%%%%%%%%
\subsection{Reconocimiento por d\'igito}
Fijamos la matriz de covarianza entrenada con 30000 im\'agenes, ya que est\'a precalculada.
Siguiendo la misma metodolog\'ia, nos concentramos ahora en el HitRate individual
de cada d\'igito. Analizamos el HitRate de acuerdo a la metodolog\'ia de detecci\'on usada.
\def \pdwidth {500pt}

\begin{figure}[H]
\begin {center}
\includegraphics[width=\pdwidth]{plots/pordig-30kcv-norma2.png}
\end {center}
\caption{Hitrate por d\'igito de detecci\'on de 500 im\'agenes usando la matriz de covarianza entrenada con 30000 im\'agenes
clasificados por distancia a los promedios usando norma 2}
\label{fig:HRD30kcv-n2}
\end{figure}

\begin{figure}[H]
\begin {center}
\includegraphics[width=\pdwidth]{plots/pordig-30kcv-norma1.png}
\end {center}
\caption{Hitrate por d\'igito de detecci\'on de 500 im\'agenes usando la matriz de covarianza entrenada con 30000 im\'agenes
clasificados por distancia a los promedios usando norma 1}
\label{fig:HRD30kcv-n1}
\end{figure}

\begin{figure}[H]
\begin {center}
\includegraphics[width=\pdwidth]{plots/pordig-30kcv-normaInf.png}
\end {center}
\caption{Hitrate por d\'igito de detecci\'on de 500 im\'agenes usando la matriz de covarianza entrenada con 30000 im\'agenes
clasificados por distancia a los promedios usando norma infinito}
\label{fig:HRD30kcv-ninf}
\end{figure}

\begin{figure}[H]
\begin {center}
\includegraphics[width=\pdwidth]{plots/pordig-30kcv-BestOf3.png}
\end {center}
\caption{Hitrate por d\'igito de detecci\'on de 500 im\'agenes usando la matriz de covarianza entrenada con 30000 im\'agenes
clasificados por distancia a los promedios usando la norma que m\'as repetida entre Norma 1, 2 e Infinito}
\label{fig:HRD30kcv-bo3}
\end{figure}

\begin{figure}[H]
\begin {center}
\includegraphics[width=\pdwidth]{plots/pordig-30kcv-top100.png}
\end {center}
\caption{Hitrate por d\'igito de detecci\'on de 500 im\'agenes usando la matriz de covarianza entrenada con 30000 im\'agenes
clasificados por distancia a 100 m\'as cercanos}
\label{fig:HRD30kcv-dist100}
\end{figure}

\section{Heatmap de reconocimiento por d\'igito}
Aca mostramos las equivocaciones y aciertos al reconocer cada d\'igito, variando la cantidad de columnas, hecho con la matriz de covarianza de
30000 im\'agenes y variando la cantidad de columnas tomadas. Eje X el valor detectado, eje Y el valor verdadero. El color marca la cantidad
de veces que se detect\'o un valor y a cual correspondia.
\def \hmwidth {500pt}
\begin{figure}[H]
\includegraphics[width=\hmwidth]{plots/heatmap-30kcv-k5-norma_2.png}
\caption{Heatmap de aciertos por d\'igito de detecci\'on de 500 im\'agenes usando la matriz de covarianza entrenada con 30000 im\'agenes
clasificados por norma 2 tomando K=5}
\label{fig:HM30kcv-k5}
\end{figure}

\begin{figure}[H]
\includegraphics[width=\hmwidth]{plots/heatmap-30kcv-k10-norma_2.png}
\caption{Heatmap de aciertos por d\'igito de detecci\'on de 500 im\'agenes usando la matriz de covarianza entrenada con 30000 im\'agenes
clasificados por norma 2 tomando K=10 }
\label{fig:HM30kcv-k10}
\end{figure}

\begin{figure}[H]
\includegraphics[width=\hmwidth]{plots/heatmap-30kcv-k25-norma_2.png}
\caption{Heatmap de aciertos por d\'igito de detecci\'on de 500 im\'agenes usando la matriz de covarianza entrenada con 30000 im\'agenes
clasificados por norma 2 tomando K=25 }
\label{fig:HM30kcv-k25}
\end{figure}

\begin{figure}[H]
\includegraphics[width=\hmwidth]{plots/heatmap-30kcv-k50-norma_2.png}
\caption{Heatmap de aciertos por d\'igito de detecci\'on de 500 im\'agenes usando la matriz de covarianza entrenada con 30000 im\'agenes
clasificados por norma 2 tomando K=50 }
\label{fig:HM30kcv-k50}
\end{figure}

\begin{figure}[H]
\includegraphics[width=\hmwidth]{plots/heatmap-30kcv-k100-norma_2.png}
\caption{Heatmap de aciertos por d\'igito de detecci\'on de 500 im\'agenes usando la matriz de covarianza entrenada con 30000 im\'agenes
clasificados por norma 2 tomando K=100 }
\label{fig:HM30kcv-k100}
\end{figure}
