\section{Ap\'endices}
\begin{lstlisting}
#include<iostream>
#include<algorithm>
#include<vector>
#include<cstring>
#include<cstdio>
#include<cmath>
#include<cassert>

using namespace std;

vector<vector<double> > input, av; //av = autovectores

#define ERRCANT(x) {if(g!=(x)){printf ("Error de lectura %d \n",__LINE__); exit(1);}}
#define ERRCANT(x) 

void transpose(vector<vector<double> > &mat)
{
    vector<vector<double>> matrizAuxiliar;
    /** Transposicion de matrices **/
    int n = mat.size();
    int m = mat[0].size();
    matrizAuxiliar.clear();
    matrizAuxiliar.resize(m,vector<double>(n));
#pragma omp parallel for
	for(int i=0;i<m;i++)
    for(int j=0;j<n;j++)
	{
	    matrizAuxiliar[i][j] = mat[j][i];
    }
    mat = matrizAuxiliar;
	return;
}

vector<vector<double> > mult(const vector<vector<double> > &A, const vector<vector<double> > &B)
{
    /** Multiplicacion de matrices standard en o(n^3) **/
	int n = A.size();
	int m = B[0].size();
	int t = A[0].size(); // tiene que ser igual a B.size();
    vector<vector<double> > B2 = B;
    transpose(B2);
	vector<vector<double> > res(n,vector<double>(m,0));
#pragma omp parallel for
	for(int i=0;i<n;i++)
	{
    	for(int j=0;j<m;j++)
        {
            double sum =0.0;
	        for(int k=0;k<t;k++)
            {
		        sum += A[i][k]*B2[j][k];
            }
            res[i][j]=sum;
        }
    }
	return res;
}

void generateX(vector<vector<double> >& X)
{
	int n = input.size();
	int m = input[0].size();
    vector<double> average(m,0);	

    for(int i=0;i<n;i++)   
    	for(int j=0;j<m;j++)
	    	average[j] += input[i][j];
    #pragma omp parallel for
	for(int j=0;j<m;j++)
		average[j] /= (double)n;
    /** Calculo el promedio de cada pixel **/
	X.clear();
	X.resize(n,vector<double>(m));
#pragma omp parallel for
    for(int i=0;i<n;i++)
    	for(int j=0;j<m;j++)
	    	X[i][j] = (input[i][j]-average[j])/sqrt(n-1);
            /** A cada pixel le asigno el pixel en su imagen menos el promedio sobre la raiz de la cantidad de imagenes menos uno**/
	return;
}

vector<vector<double> > generateMx()
{
    vector<vector<double> > X;
    generateX(X); /** Genero la matriz X **/
    vector<vector<double> > Xt(X);	
	transpose(Xt);
	return mult(Xt,X); /** Genero Mx matriz de covarianza como X^t por X **/    
}

double norm(vector<double> &vec) /** Calculo norma 2 de vec **/
{
	double res = 0;
    #pragma omp parallel for reduction (+:res)
	for(int i=0;i<(int)vec.size();i++)
		res += vec[i]*vec[i];
	return sqrt(res);
}

vector<vector<double> > Id(int n) /** Identidad de n x n **/
{
    vector<vector<double> > A(n,vector<double>(n,0));
    for(int i = 0; i< n; i++)
        A[i][i]=1;
    return A;
}




void householder(vector<vector<double> > &A,vector<vector<double> > &Q,vector<vector<double> > &R)
{
    /** Factorizacion QR de Householder en o(n^3) **/
	int n = A.size();
    vector<vector<double> > aux(n,vector<double>(n)),aux2(n,vector<double>(n));
	
	R = A;
	Q = Id(n);
    vector<double> u,v;
    v=vector<double>(n);
	for(int i=0;i<n-1;i++)
	{        
        u=vector<double>(i,0.0);        
	    for(int j=i;j<n;j++)
            u.push_back(R[j][i]);
        double alpha = norm(u);
        if(abs(abs(alpha)-abs(u[i])) < 1e-6) /** Si debajo de la diagonal son todos ceros no itero **/
            continue;
        if(alpha*u[i]>0) /** Cambio el signo si es necesario **/
            alpha *= -1.;
        u[i] += alpha;
        alpha = norm(u);

        
#pragma omp parallel for         
        for(int j=0;j<n;j++)
            v[j]=u[j]/alpha;
        
        /** Inicio calculo R **/
#pragma omp parallel for 
        for(int j=0;j<n;j++)
            u[j] = 0;

#pragma omp parallel for
        for(int j=0;j<n;j++)
        for(int t=0;t<n;t++)        
            u[j] += v[t]*R[t][j];

#pragma omp parallel for
        for(int j=0;j<n;j++)
        for(int t=0;t<n;t++)
            aux[j][t] = 2*v[j]*u[t];

#pragma omp parallel for
        for(int j=0;j<n;j++)
        for(int t=0;t<n;t++)
            R[j][t] -= aux[j][t];
        /** Fin calculo R **/
        /** Inicio calculo Q **/
#pragma omp parallel for
        for(int j=0;j<n;j++)
            u[j] = 0;
            
#pragma omp parallel for    
        for(int j=0;j<n;j++)
        for(int t=0;t<n;t++)        
            u[j] += Q[j][t]*v[t];
#pragma omp parallel for        
        for(int j=0;j<n;j++)
        for(int t=0;t<n;t++)
            aux2[j][t] = 2*u[j]*v[t];
#pragma omp parallel for                
        for(int j=0;j<n;j++)
        for(int t=0;t<n;t++)
            Q[j][t] -= aux2[j][t];
        /** Fin calculo Q **/
	}
	return;
}

const double delta = 5000;

int iteraciones=0;

bool sigoIterando(vector<vector<double> > &A)
{
	double res = 0.;
	int n = A.size();
    #pragma omp parallel for reduction(+:res)
    for(int i=0;i<n;i++)
    {
        double localRes=0.;
	    for(int j=0;j<i;j++)
		    localRes += fabs(A[i][j]);
        res+=localRes;
    }
    printf("Iteracion %d: Suma %lf\t Promedio %lf\n",++iteraciones, res, res/((n*(n-1))/2));
	return res>delta;
	/** Itero hasta que los elementos debajo de la diagonal sumen menos de 15 **/
}
#define MAXITERACIONES 500

vector<vector<double> > Q,R;
vector<vector<double> > allAuVec; 
void eig(vector<vector<double> > &A, vector<vector<double> > &auVec)
{
	int n = A.size(); /// A es cuadrada
    vector<vector<double> > matrizAuxiliar;
    auVec = vector<vector<double> >(allAuVec);
    householder(A,Q,R); /** Calculo QR con Householder **/
    A = mult(R,Q); /** Multiplico RQ para obtener la nueva A que es la matriz de covarianza **/
    allAuVec = mult(allAuVec,Q); /** Multiplico todas las Q para obtener los autovectores **/
	/** Ordeno los autovectore segun la magnitud de los autovalores **/
	vector<pair<int,int> > aux(n);

#pragma omp parallel for
    for(int i=0;i<n;i++)
		aux[i] = make_pair(A[i][i],i);
	sort(aux.begin(),aux.end()); /** Ordeno los autovalores **/
	reverse(aux.begin(),aux.end());
	matrizAuxiliar = allAuVec;

#pragma omp parallel for
    for(int i=0;i<n;i++)
	for(int j=0;j<n;j++)
	{
		auVec[i][j] = matrizAuxiliar[aux[i].second][j];
		/** Asigno a la i-esima columna el autovector correspondiente al i-esimo autovalor **/
	}
	/** Fin ordenamiento de autovectores **/
	return;
}

vector<vector<double> > tc;

vector<double> calctc(vector<double> imagen, int k) /** Calculo la transfomacion caracteristica de imagen con parametro k **/
{
	vector<double> res(k,0);
	int n = imagen.size();
    
	for(int i=0;i<k;i++)
	for(int j=0;j<n;j++)
	{
		res[i]+=av[i][j]*imagen[j];
	}
	return res;
}

void fillTC(int k) /** LLeno tc con las transformaciones caracteristicas **/
{
	int n = input.size();
	tc.resize(n,vector<double>(k,0));
#pragma omp parallel for
	for(int i=0;i<n;i++)
	{
		tc[i] = calctc(input[i],k);
	}
	return;
}

double dist(vector<double> &v1, vector<double> &v2, int norm)
{
	double res = 0;
	if(norm==0) /** Norma infinito de v1 - v2 | Distancia infinito de v1 a v2 **/
	{
	    for(int i=0;i<(int)v1.size();i++)
            res = max(res,v1[i]-v2[i]);
	}
	if(norm==1) /** Norma 1 de v1 - v2 | Distancias 1 de v1 a v2 **/
	{
	    for(int i=0;i<(int)v1.size();i++)
            res += abs(v1[i]-v2[i]);
	}
    
    if(norm==2) /** Norma 2 de v1 - v2 | Distancia 2 de v1 a v2 **/
    {        
        for(int i=0;i<(int)v1.size();i++)
            res += (v1[i]-v2[i])*(v1[i]-v2[i]);
        res = sqrt(res);
    }
	return res;
}

void usage()
{
    cout << "Uso: ./OCR <k> <imp> <norma> <training> <test>" << endl;
    cout << "donde k = cantidad de componentes principales a tomar de las transformaciones "<< endl;
    cout << "      imp = 0 (usando nearest neighbours, 1 usando distancia al promedio " << endl;
    cout << "      norma = 0 (norma infinito), 1 (norma 1), 2 (norma 2) " << endl;
    cout << "      training = (default 10000) cantidad de imagenes de entrenamiento, 1 a 30000 " << endl;
    cout << "      test = (default 500) cantidad de imagenes de test, 1 a 2000 " << endl;
    return;
}

int main(int argc, char* argv[])
{
    if(argc<4){ usage(); exit(1);}
    if(argc>6){ usage(); exit(1);}
    int max_k = atoi(argv[1]), imp = atoi(argv[2]), norm = atoi(argv[3]);
    int min_k = 1;
    /** k es el parametro k del enunciado, imp es la implementacion y norm es la norma que usamos para medir distancias **/
    
    int training_count = 10000; /** Usamos 10000 imagenes de entrenamiento **/    
    int test_count = 500; /** Usamos 500 imagenes de test **/
    if(argc>4)
        training_count=atoi(argv[4]); /** a menos que el parametro lo indique **/
    if(argc>5)
        test_count=atoi(argv[5]); /** a menos que el otro parametro lo indique **/
    int padding_count = 30000-training_count; /** paddeamos con imagenes para siempre usar las mismas de test **/


	FILE* v = fopen("../datos/trainingImages.txt","r");
	int n, t; /** dims de la matriz de training **/
    int g;  /** absorbedora de errores **/
	g = fscanf(v,"%d %d",&n,&t);ERRCANT(2);
    printf("Se van a leer: %d imgs Training, %d imgs Test \n",training_count,test_count);
    vector<vector<double> > testImages;
    
    input.clear();
	input.resize(training_count,vector<double>(t));
	testImages.resize(test_count,vector<double>(t));
    /** Comienzo lectura imagenes de entrenamiento y test **/
	for(int i=0;i<training_count;i++)
    {
        for(int j=0;j<t;j++)
        {
            g = fscanf(v,"%lf",&input[i][j]);ERRCANT(1);
        }
    }
    for(int i=0;i<padding_count;i++)
    {        
        for(int j=0;j<t;j++)
        {
            g = fscanf(v,"%*lf");
        }
    }
    for(int i=0;i<test_count;i++)
    {
		for(int j=0;j<t;j++)
		{
			g = fscanf(v,"%lf",&testImages[i][j]);ERRCANT(1);
		}
	}
    fclose(v);
	/** Fin lectura imagenes de entrenamiento y test **/
	/** Comienzo lectura labels de entrenamiento y test **/
    v = fopen("../datos/trainingLabels.txt","r");
    g= fscanf(v,"%d",&n); ERRCANT(1);
    vector<int> labels, testLabels;
    labels.resize(training_count);
    testLabels.resize(test_count);
    for(int i=0;i<training_count;i++)
    {
        g = fscanf(v,"%d",&labels[i]);ERRCANT(1);
    }
    
    for(int i=0;i<padding_count;i++)
    {
        g = fscanf(v,"%*d");
    }

    for(int i=0;i<test_count;i++)
    {
		g = fscanf(v,"%d",&testLabels[i]);ERRCANT(1);
	}
    
	fclose(v);
	/** Fin lectura labels de entrenamiento y test **/
	v = fopen("V.txt","r");
	if(v==NULL) /** Si V no existe la genero **/
	{
#ifdef PRECALC
        vector<vector<double> > Mx = generateMx(); // Genero Mx la matriz de covarianza 
        allAuVec = Id(Mx.size()); /** Inicializo la matriz de autovectores **/
        bool b = true;
        for(int its=1;its<MAXITERACIONES;its++)
        {
            
            eig(Mx,av); // Calculo los autovectores de la matriz de covarianza 
            b = sigoIterando(Mx);
            if (!b) break;
            
        }

        v = fopen("V.txt","w"); // Escribo la matriz V en un archivo 
        fprintf(v,"%d %d\n",(int)av.size(),(int)av[0].size());
        for(int i=0;i<(int)av.size();i++)
        {
            for(int j=0;j<(int)av[0].size();j++)
                fprintf(v,"%.6lf ",av[i][j]);
            fprintf(v,"\n");
        }
        fclose(v);
#endif
	}
	else /** Si V ya fue generada previamente la levanto del archivo **/
	{
	    int N,M;
	    g=fscanf(v,"%d %d",&N,&M);ERRCANT(2);
	    av.clear();
	    av.resize(N,vector<double>(M));
	    for(int i=0;i<N;i++)
        {
            for(int j=0;j<M;j++)
            {
                g=fscanf(v,"%lf",&av[i][j]);ERRCANT(1);
            }
        }

	}
#ifndef PRECALC
    vector<vector<double> > Mx = generateMx(); /** Genero Mx la matriz de covarianza **/
    allAuVec = Id(Mx.size()); /** Inicializo la matriz de autovectores **/
    for(int its=1;its<MAXITERACIONES;its++)
    {
        eig(Mx,av); /** Calculo los autovectores de la matriz de covarianza **/        
        bool b = sigoIterando(Mx);
#endif
        for(int k=min_k;k<max_k;k++)
        {
            fillTC(k); /** Genero las transformaciones caracteristicas de cada imagen de entrenamiento **/
            vector<double> vec;
            vector<pair<double,int> > distancias;    
            vector<int> cant(10);
            int bien = 0;
            int mal = 0;

            if(imp==0)
            {
                distancias.resize(training_count);
                /** La implementacion 0 toma las 100 imagenes mas cercanas y toma la mas repetida de esas 100 **/
                for(int i=0;i<test_count;i++) /** Iteramos sobre la imagenes de test **/
                {
                    vec = calctc(testImages[i],k); /** Calculamos la transformacion caracteristica de la imagen dada **/
                                
                    for(int j=0;j<training_count;j++)
                    {
                        distancias[j] = make_pair(dist(tc[j],vec,norm),j);
                        /** Guardamos en un par la distancia a cada imagen de entrenamiento con la imagen **/
                    }
                    sort(distancias.begin(),distancias.end()); /** Ordenamos por distancia **/
                    for(int j=0;j<10;j++)
                        cant[j] = 0;            
                    for(int j=0;j<100;j++)
                    {
                        cant[labels[distancias[j].second]]++;
                        /** Contamos cuantas imagenes hay de las 100 mas cercanas con cada label **/
                    }
                    int cualEs = 0;
                    for(int j=0;j<10;j++)
                        if(cant[j]>cant[cualEs])
                            cualEs = j;
                    if(testLabels[i]==cualEs)
                        bien++;
                    else
                        mal++;
                }
            }
            else if(imp == 1)
            {
                /** La implementacion 1 toma el promedio de las transformaciones de cada digito y compara distancia a cada promedio **/
                vector<double> promedio[10];
                int cuantos[10];
                memset(cuantos,0,sizeof(cuantos));
                /** Comienzo calculo promedios **/
                for(int i=0;i<10;i++)
                {
                    promedio[i].clear();
                    promedio[i].resize(k,0);
                }
                for(int i=0;i<training_count;i++)
                {
                    cuantos[labels[i]]++;
                    for(int j=0;j<k;j++)
                        promedio[labels[i]][j] += tc[i][j];
                }
                for(int i=0;i<10;i++)
                for(int j=0;j<k;j++)
                if(cuantos[i]!=0)
                    promedio[i][j] /= (double)cuantos[i];
                /** Fin calculo promedios **/
                for(int i=0;i<test_count;i++)
                {
                    vec = calctc(testImages[i],k);
                    distancias.resize(10);
                    for(int j=0;j<10;j++)
                    {
                        distancias[j] = make_pair(dist(promedio[j],vec,norm),j);
                        /** Comparamos distancia al promedio de cada digito **/
                    }
                    sort(distancias.begin(),distancias.end());
                    int cualEs = distancias[0].second;
                    /** Ordenamos y nos quedamos con el mas cercano **/
                    if(testLabels[i]==cualEs)
                        bien++;
                    else
                        mal++;
                }
            }
            cout << k << " "<< bien <<" "<<mal<<" " << (double)bien/(double)(bien+mal)<< endl;
        /** Imprimimos cantidad de hits, cantidad de misses y proporcion de hits **/
        }
#ifndef PRECALC
    }
#endif
    return 0;
}

\end{lstlisting}
