\section{Desarrollo}
La dificultad del desarrollo de este experimento fue un poco m\'as elevada a la de los dos trabajos pr\'acticos anteriores. En primer lugar, decidimos hacer todo el trabajo pr\'actico en Matlab ya que es mucho m\'as f\'acil que en C++ dado que la SVD viene ya implementada en Matlab. El primer problema que nos encontramos cuando implementamos el experimento en Matlab, fue que al calcular la SVD con 60.000 imagenes, se generaba una matriz de $60.000 \times 60.000$ doubles, que no entraba en memoria, es por eso que utilizamos s\'olo 40.000 imagenes para el experimento en Matlab.

Luego de tener el TP hecho y funcionando en Matlab, empezamos a programarlo en C++. Lo primero que notamos fue que no era necesario calcular la SVD, sino s\'olo la matriz $V$, por lo que no ten\'iamos que generar la matriz de $60.000 \times 60.000$ sino que s\'olo generabamos una de $60.000 \times 784$. A\'un as\'i, consideramos que 60.000 era un n\'umero muy grande y por lo tanto nos restringimos a operar con s\'olo 5.000 imagenes. Al ver los resultados nos dimos cuenta de que era una muestra significativa, y por lo tanto hicimos el trabajo pr\'actico usando 5.000 imagenes de entrenamiento y 1.000 imagenes de test.

\subsection{Matriz de Covarianza}
Lo primero que hicimos fue generar la matriz de covarianza con las 5.000 imagenes que tomamos para entrenamiento. Para eso seguimos la f\'ormula del enunciado. Primero generamos la matriz $X$