\section{Conclusiones}

\subsection{Intervalo de b\'usqueda}


Uno de los grandes problemas que afrontamos durante la la preparaci\'on de los 
resultados, fue que era muy facil errarle al intervalo en el que buscar
$\beta$. Lo primero que nos pas\'o fue
que no habiamos descubierto que la funci\'on estimadora de $\beta$ tenia una 
raiz en 0. Esto nos trajo grandes
dolores de cabeza, ya que no eran los valores que debieramos haber estimado. 
Para solucionarlo decidimos calcular
desde el intervalo 1.0 en adelante, y result\'o una buena decisi\'on, casi 
todos los experimentos convergian bien.

Otro problema vinculado fue el tama\~no del intervalo con relaci\'on a los 
m\'etodos implementados. 
En el caso de Regula Falsi, si buscabamos en un intervalo grande, como por 
ejemplo, [1,100], el algoritmo 
convergia increiblmente lento. El valor de $\beta$ que buscabamos estaba cerca 
del 6, entonces las 
sucesivas rectas secantes que trazaba Regula Falsi tenian pendiente muy cerca 
del 0, por lo cual en cada 
iteraci\'on, era dificil mejorar m\'as de $10^-6$. Pudimos comprobar esto al 
correr miles de pasos de Regula Falsi y ver que efectivamente convergia, pero
que cada iteraci\'on era realmente lenta.

Fue en este punto que tambien notamos la fortaleza de los metodos de orden de 
convergencia cuadratico,
en este caso, el m\'etodo de Newton. Incluso con intervalos tan, o m\'as 
grandes que los mismos de Regula Falsi, 
Newton convergia en pocas iteraciones, menos de una decima parte de las del 
otro m\'etodo.


\subsection{Precisi\'on en los c\'alculos}

Otro detalle que pudimos apreciar fue la importancia de la precision num\'erica 
a la hora de hacer las cuentas. 
Fue de especial interes para el trabajo el poder realizar todos estos m\'etodos 
con distintos valores 
de precisi\'on. Para hacer enfasis en este punto, logramos trazar las curvas 
para todos los valores de decimales 
entre 12 y 27, como vimos en la figura \ref{fig:AproxCaso1}.
Es notable ver como hasta cierto momento, la precision es de critica 
importancia, pero despues los valores convergen y 
agregar digitos no afectan los algoritmos.s

En el caso de la precisi\'on en Regula Falsi es mucho m\'as notorio. El 
algoritmo tiene menos restricciones en 
la funci\'on que puede usar para estimar, y su velocidad de convergencia es 
menor. Es notablemente m\'as impreciso
que Newton a mismo epsilon. Las soluciones posibles a este problema consisten 
una combinacion de: 
\begin{itemize}
    \item Aumentar la precisi\'on de los operandos (no siempre es posible 
cuando superamos la precisi\'on del hardware)
    \item Aumentar la cantidad de iteraciones para que converga (no garantiza 
mejor soluci\'on, y puede ser prohibitivamente caro en tiempo)
    \item Proponer distintos valores iniciales para el m\'etodo (no es claro 
como hacerlo de manera general)
\end{itemize}

\subsection{Dificultad en la implementaci\'on de algoritmos}

El experimento que qued\'o trunco fue la implementaci\'on del m\'etodo de la 
secante para el problema dado. 
Si bien la teor\'ia nos muestra los beneficios de cada m\'etodo para diferentes 
problemas, nos encontramos con otro 
tipo de dificultad relacionado a la destreza a la hora de programar y de hacer 
matem\'atica que dificult\'o el uso 
del m\'etodo de la secante y nos hizo inclinarnos por Regula falsi. Esto nos 
demuestra que a la hora de encarar un 
problema se debe tener en cuenta la dificultad de implementaci\'on de la 
soluci\'on propuesta para un problema dado. 
En la elecci\'on de m\'etodos num\'ericos existe una compensaci\'on entre 
dificultad de implementaci\'on, recursos necesarios 
y grado de error del resultado. Estos componentes deben ser tomados en cuenta a 
la hora de dise\~nar una soluci\'on.
