\section{Conclusiones}

Uno de los grandes problemas que afrontamos durante la la preparaci\'on de los resultados,
fue que era muy facil errarle al intervalo en el que buscar $\Beta$. Lo primero que nos pas\'o fue
que no habiamos descubierto que la funci\'on estimadora de $\Beta$ tenia una raiz en 0. Esto nos trajo grandes
dolores de cabeza, ya que no eran los valores que debieramos haber estimado. Para solucionarlo decidimos calcular
desde el intervalo 1.0 en adelante, y result\'o una buena decisi\'on, casi todos los experimentos convergian bien.

Otro problema vinculado fue el tama\~no del intervalo con relaci\'on a los m\'etodos implementados. 
En el caso de Regula Falsi, si buscabamos en un intervalo grande, como por ejemplo, [1,100], el algoritmo 
convergia increiblmente lento. El valor de $\Beta$ que buscabamos estaba cerca del 6, entonces las 
sucesivas rectas secantes que trazaba Regula Falsi tenian pendiente muy cerca del 0, por lo cual en cada 
iteraci\'on, era dificil mejorar m\'as de $10^-6$. Pudimos comprobar esto al correr miles de pasos de Regula Falsi y ver que efectivamente convergia, pero
que cada iteraci\'on era realmente lenta.

Fue en este punto que tambien notamos la fortaleza de los metodos de orden de convergencia cuadratico,
en este caso, el m\'etodo de Newton. Incluso con intervalos tan, o m\'as grandes que los mismos de Regula Falsi, 
Newton convergia en pocas iteraciones, menos de una decima parte de las del otro m\'etodo.



Otro detalle que pudimos apreciar fue la importancia de la precision num\'erica a la hora de hacer las cuentas. 
Fue de especial interes para el trabajo el poder realizar todos estos m\'etodos con distintos valores 
de precisi\'on. Para hacer enfasis en este punto, logramos trazar las curvas para todos los valores de decimales 
entre 12 y 27, como vimos en la figura \ref{fig:AproxCaso1}.
Es notable ver como hasta cierto momento, la precision es de critica importancia, pero despues los valores convergen y 
agregar digitos no afectan los algoritmos.

En el caso de la precisi\'on en Regula Falsi es mucho m\'as notorio. El algoritmo tiene menos restricciones en 
la funci\'on que puede usar para estimar, y su velocidad de convergencia es menor. Es notablemente m\'as impreciso
que Newton a mismo epsilon.
