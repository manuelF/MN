\section{Conclusiones}

\index{Conclusiones!Intervalo de b\'usqueda}
\subsection{Intervalo de b\'usqueda}

Uno de los grandes problemas afrontados durante la la preparaci\'on de los 
resultados, est\'a relacionado con la dificultad en la b\'usqueda de un intervalo
confiable en el cual buscar $\beta$. El primer problema fue el desconocimiento
de que la funci\'on estimadora de $\beta$ tiene na raiz en 0. Esto trajo una
gran confusi\'on, ya que los valores obtenidos difer\'ian ampliamente 
de los que debieramos haber estimado. Para solucionarlo se decidi\'o calcular
desde el intervalo 1.0 en adelante, y result\'o una buena decisi\'on, casi 
todos los experimentos resultaron en convergencias satisfactorias.

Otro problema vinculado es el tama\~no del intervalo con relaci\'on a los 
m\'etodos implementados. 
En el caso de Regula Falsi, buscando una ra\'iz en un intervalo grande, como por 
ejemplo, [1,100], el algoritmo 
converg\'ia visiblemente lento. El valor de $\beta$ buscado estaba cerca 
del 6, entonces las sucesivas rectas secantes que trazaba Regula Falsi ten\'ian 
pendiente muy cerca del 0, por lo cual en cada iteraci\'on, era dif\'icil 
mejorar m\'as de $10^-6$ de tolerancia. Pudimos comprobar esto al 
correr miles de pasos de Regula Falsi y ver que efectivamente converg\'iia, pero
que cada iteraci\'on era realmente lenta.

Fue en este punto que tambi\'en notamos la fortaleza de los m\'etodos de orden de 
convergencia cuadr\'atico, en este caso, el m\'etodo de Newton. 
Incluso con intervalos iguales, o m\'as 
grandes que los mismos de Regula Falsi, Newton converg\'ia en pocas iteraciones, 
menos de una decima parte de las del otro m\'etodo.

\index{Conclusiones!Precisi\'on en los c\'alculos}
\subsection{Precisi\'on en los c\'alculos}

Otro detalle que pudimos apreciar fue la importancia de la precision num\'erica 
a la hora de hacer las cuentas. 
Fue de especial interes para el trabajo el poder realizar todos estos m\'etodos 
con distintos valores 
de precisi\'on. Para hacer \'enfasis en este punto, logramos trazar las curvas 
para todos los valores de decimales 
entre 12 y 27, como vimos en las figuras \ref{fig:Fit4Y7Newton} y \ref{fig:FitCaso4Y7RegulaFalsi}.
Es notable ver como hasta cierto momento, la precision es de critica 
importancia, pero despues los valores convergen y 
agregar d\'igitos no afectan los algoritmos.

En el caso de la precisi\'on en Regula Falsi es mucho m\'as notorio. El 
algoritmo tiene menos restricciones en la funci\'on que puede usar para estimar, 
y su velocidad de convergencia es menor. Es notablemente m\'as impreciso
que Newton a mismo epsilon. Las soluciones posibles a este problema consisten 
una combinacion de: 
\begin{itemize}
    \item Aumentar la precisi\'on de los operandos (no siempre es posible 
cuando superamos la precisi\'on del hardware)
    \item Aumentar la cantidad de iteraciones para que converga (no garantiza 
mejor soluci\'on, y puede ser prohibitivamente caro en tiempo)
    \item Proponer distintos valores iniciales para el m\'etodo (no es claro 
como hacerlo de manera general)
\end{itemize}

\index{Conclusiones!Dificultad en la implementaci\'on de algoritmos}
\subsection{Dificultad en la implementaci\'on de algoritmos}

El experimento que qued\'o trunco fue la implementaci\'on del m\'etodo de la 
secante para el problema dado. 
Si bien la teor\'ia nos muestra los beneficios de cada m\'etodo para diferentes 
problemas, nos encontramos con otro 
tipo de dificultad relacionado a la destreza a la hora de programar y de hacer 
matem\'atica que dificult\'o el uso 
del m\'etodo de la secante y nos hizo inclinarnos por Regula falsi. Esto nos 
demuestra que a la hora de encarar un 
problema se debe tener en cuenta la dificultad de implementaci\'on de la 
soluci\'on propuesta para un problema dado. 
En la elecci\'on de m\'etodos num\'ericos existe una compensaci\'on entre 
dificultad de implementaci\'on, recursos necesarios 
y grado de error del resultado. Estos componentes deben ser tomados en cuenta a 
la hora de dise\~nar una soluci\'on.

\index{Conclusiones!Elecci\'on de criterios para comparaci\'on de m\'etodos}
\subsection{Elecci\'on de criterios para comparaci\'on de m\'etodos}

Un tema importante que nos surgi\'o de toda esta experimentaci\'on intensiva
con estos varios m\'etodos elegidos, fue intentar proponernos cual es el mejor
de estos m\'etodos. Por todo lo que hemos visto, la respuesta esta lejos de
ser sencilla. Influyen muchisimo m\'as algunas variables de lo que hubieramos
pensando. Por ejemplo, uno por instinto elige siempre un epsilon muy chico
para las comparaciones de tolerancia ($10^{-8}$ es bastante habitual en programas
reales de m\'etodos num\'ericos). Sin embargo, debemos poder representar al 
menos $2^{-30}$ para eso, con lo que se necesitaria un \texttt{double} como minimo,
y hay arquitecturas que no estan preparadas para trabajar con eso (o al menos 
no optimamente, como es el caso de CUDA de NVIDIA). 

Otra variable importante fue la cantidad de iteraciones m\'aximas que pueden 
usar los algoritmos, intimamente ligada al epsilon y la precisi\'on elegidas. 
Lo que notamos es que si teniamos poca precisi\'on de los n\'umeros o 
epsilon muy grande, la cantidad de iteraciones siempre estaban muy acotadas,
pero en cuanto se llega al limite de precisi\'on porque no se pueden representar
valores intermedios y todavia no se cumple la condicion de corte por el epsilon,
entonces se dispara el numero de iteraciones al m\'aximo, y sin mejoras, algo 
realmente malo. Como paliativo, se puede poner unas condiciones de corte como
``Si llevas varias iteraciones sin mejorar, y no cambian tus variables de input,
cort\'a porque no vas a volver a mejorar''. Sin embargo, esto no es m\'as que un
parche al problema, y es una invitaci\'on a repensar las condiciones en el
epsilon, en los tipos de dato que almacenamos y en la precisi\'on que pensamos
que necesitamos, si es imprescindible el n-esimo digito.

