\section{Desarrollo}
El primer m\'etodo elegido para obtener un despeje de $\beta$ fue el de Newton. La motivaci\'on para implementar este m\'etodo como primera opci\'on est\'a relacionada con la importancia que se le da al m\'etodo tanto en los textos consultados como en clase.

A priori la dificultad al emplear este m\'etodo est\'a ligada al c\'alculo de la derivada primera de la funci\'on a la que se le busca el cero en cada iteraci\'on. Si bien se puede decir que la iteraci\'on de Newton es la m\'as compleja de implementar, la divisi\'on del problema en subproblemas m\'as pequen/~os utilizando funciones auxiliares fue clave para una resoluci\'on que se refleja en un c\'odigo limpio y elegante.




Utilizando el despeje sugerido en clase:\\

$\frac{M(2\beta)}{M^2(\beta)}=1 + \beta(R(\beta)-R(0))$

De esta funci\'on se deriva la siguiente ecuaci\'on: \\

$0 = \frac{\beta}{n}\sum{i=1}{n}\log x_i - \log \sum{i=1}{n}x_i^{\beta} + \log(n\lambda)-\psi(\lambda)$

Luego, el m\'etodo general de Newton empleado se comporta del siguiente modo: \\

$\beta_{n+1} = \beta_{n} - \frac{f(\beta_{n})}{f'(\beta_{n})}$
