
\section{Desarrollo}

El desarrollo de los experimentos fue realizado en serie. 
El m\'etodo de Newton, que se cree fundamental tanto por lo visto en clase como 
por lo consultado en libros relativos al an\'alisis num\'erico, fue el primero 
en ser implementado. Este proceso fue r\'apido, no existieron grandes 
dificultades en la implementaci\'on y los resultados fueron satisfactorios 
desde el comienzo.

La decisi\'on sobre la implementaci\'on del m\'etodo de Regula falsi llega a
partir de las dificultades en la aproximaci\'on mediante el m\'etodo de la
Secante, dificultades mencionadas m\'as adelante en esta seccio\'on.

El desarrollo de los experimentos es explicado en el orden en que estos fueron
realizados. Si bien se realizaron ajustes a los m\'etodos luego de la primera
iteraci\'on, lo explicado a continuaci\'on da una noci\'on realista de los pasos
seguidos y las dificultades encontradas a la hora de la implementaci\'on de los
algoritmos de los m\'etodos mencionados. 

Antes de la explicaci\'on del
desarrollo de la experiencia se procede a un repaso por las herramientas
utilizadas y creadas para el an\'alisis, testeo y repetici\'on sistem\'atica de
los experimentos.

\subsection{Herramientas}

El lenguaje de programaci\'on utilizado para la implementaci\'on de los
experimentos es C++. La combinaci\'on de expresividad y manejo de estructuras de
bajo nivel es de gran interes a la hora de desarrollar un algoritmo que
implementa un m\'etodo num\'erico. Las estructuras de datos que provee la
librer\'ia est\'andar son suficientes para la manipulaci\'on de los datos e
incluso para t\'ecnicas de mejora experimentales probadas durante la
realizaci\'on del trabajo.

Para las rutinas y ploteo de gr\'aficos se utiliz\'o 
\href{http://www.gnu.org/software/octave/}{GNU Octave}, un entorno conformado
por software libre, multiplataforma y compatible con MATLAB. A trav\'es del
disen/~o de scripts de Octave, fue posible el ploteo en serie de un n\'umero
considerable de gr\'aficos en cada corrida de los tests.

El \"pegamento\" entre estas herramientas est\'a conformado por una conjunto de
scripts de bash. Gracias a estos procedimientos es posible, corriendo un solo
comando, realizar tests con argumentos variables, realizar ploteos y
comparaciones sobre los resultados.

La posibilidad de correr los experimentos con una variedad interesante de
par\'ametros es clave para entender el comportamiento de los mismos.


\subsection{M\'etodo de Newton}

El primer m\'etodo elegido para obtener un despeje de $\beta$ fue el de Newton. 
La motivaci\'on para implementar este m\'etodo como primera opci\'on est\'a 
relacionada con la importancia que se le da al m\'etodo tanto en los textos 
consultados como en clase.

A priori, la dificultad al emplear este m\'etodo est\'a ligada al c\'alculo de 
la derivada primera de la funci\'on a la que se le busca el cero en cada 
iteraci\'on. Si bien se puede decir que la iteraci\'on de Newton es la m\'as 
compleja de implementar, la divisi\'on del problema en subproblemas 
m\'as pequen\~os, utilizando funciones auxiliares, fue clave para una 
resoluci\'on que se refleja en un c\'odigo limpio y elegante.

Previo a la implementaci\'on del m\'etodo se realizaron los c\'alculos 
matem\'aticos correspondientes a la funci\'on analizada. 
A continuaci\'on se encuentran los pasos que justifican la utilizaci\'on 
de la f\'ormula.

Utilizando el despeje sugerido en clase:

\begin{center}
$\frac{M(2\beta)}{M^2(\beta)}=1 + \beta(R(\beta)-R(0))$\\
\end{center}

De esta funci\'on se deriva la siguiente ecuaci\'on: 

\begin{center}
$0 = \frac{\beta}{n}\sum{i=1}{n}\log x_i - \log \sum{i=1}{n}x_i^{\beta} 
+ \log(n\lambda)-\psi(\lambda)$\\
\end{center}

Luego, el m\'etodo general de Newton empleado se comporta del siguiente modo: 
\begin{center}

$\beta_{n+1} = \beta_{n} - \frac{f(\beta_{n})}{f'(\beta_{n})}$\\

\end{center}

Habiendo obtenido la ecuaci\'on correspondiente al m\'etodo de Newton para esta 
funcio\'on se procedi\'o a la implementaci\'on del problema en el lenguaje 
de programaci\'on C++.

Los par\'ametros de entrada elegidos para el programa que calcula la 
aproximaci\'on de $\beta$ son los siguientes:

\begin{itemize}
  \item \textbf{Precisi\'on}: Cantidad de d\'igitos utilizados en el c\'alculo. 
La variaci\'on de este par\'ametro conlleva una variaci\'on en la precisi\'on 
del resultado del c\'alculo y en el tiempo de ejecuci\'on del programa. 
El m\'aximo valor es el elegido como control y corresponde a 52 d\'igitos de 
precisi\'on.

  \item \textbf{Iteraciones m\'aximas}: M\'axima cantidad de iteraciones de 
Newton. Este criterio de parada previene ciclos infinitos ya sea por divergencia
del m\'etodo como por problemas de precisi\'on en c\'alculos de punto flotante. 
Por defecto la m\'axima cantidad es de 10 iteraciones.

  \item \textbf{Rango de $\beta$}: Control de la tolerancia. Por defecto 
se asume que $\beta$ se encuentra entre 0.0 y 0.2. Sin embargo, estos 
valores se pueden ajustar con el fin de modificar la tolerancia del m\'etodo.
\end{itemize}

Esta distribuci\'on de los par\'ametros de entrada es clave para el testeo 
sistem\'atico y la comparativa entre corridas del mismo sistema e 
interm\'etodo a la hora de comparar con el de Regula falsi.

Experimentando la variaci\'on en la precisi\'on del algoritmo se destaca una 
diferencia significativa tomando valores entre los 15 y 25 d\'igitos de 
precisi\'on. Tomando valores mayores a 25 d\'igitos la precisi\'on del 
algoritmo no aumenta significativamente, alcanzando valores similares a los 
de la precisi\'on m\'axima impuesta de 52 d\'igitos. 
Esta observaci\'on se interpreta, a primera vista, como una aceleraci\'on 
significativa del error hacia valores cercanos al cero.

\subsection{M\'etodo de la Secante}

El segundo m\'etodo elegido fue el m\'etodo de la Secante. Este se presenta
similar al m\'etodo de Newton y posee un orden de convergencia superlineal. 
R\'apidamente qued\'o descartado debido a problemas relacionados al c\'alculo 
de cocientes, en donde el denominador se acerca a cero a un nivel 
pr\'acticamente indistinguible al nivel de representaci\'on de la computadora.

\subsection{M\'etodo de Regula falsi}

Habiendo implementado el m\'etodo de la Secante, y para subsanar el problema 
de la divisi\'on por cero, se procedi\'o a implementar el m\'etodo de 
Regula Falsi, tambi\'en conocido como Regla Falsa.

De la misma forma que con el m\'etodo de Newton, partiendo de la ecuaci\'on:
\begin{center}

$\frac{M(2\beta)}{M^2(\beta)}=1 + \beta(R(\beta)-R(0))$\\
\end{center}

para despejar $\beta$ y luego utilizando las mismas f\'ormulas que en el 
caso anterior para despejar $\lambda$ y $\sigma$.

La formula resultante para el m\'etodo de Regula Falsi es:
\begin{center}

$\beta_n = \beta_{n-1} - \frac{f(\beta_{n-1}) (\beta_{n-1}-\beta_{n-2})}
{f(\beta_{n-1}) - f(\beta_{n-2})}$\\
\end{center}

Se tom\'o la decisi\'on de elegir $f(\beta_{n-1})$ y $f(\beta_n)$ \'o 
$f(\beta_{n-2})$ y $f(\beta_n)$ seg\'un el signo de $f(\beta_n)$ de modo 
tal de que los dos n\'umeros que sean tenidos en cuenta tengan distinto signo 
al aplicarle $f$.

El principal problema asociado a este m\'etodo se vio en los casos 1 y 2 donde
la convergencia se daba con un grado de precisi\'on de aproximadamente 20
d\'igitos, pero al utilizar menos el m\'etodo se comportaba de manera peculiar,
perdiendo convergencia a medida que se agregan d\'igitos de precisi\'on.

Es por eso que investigamos y decidimos implementar una mejora conocida como el
``M\'etodo de Illinois''.

\subsection{Algoritmo de Illinois}

Luego de implementar el algoritmo de Regula Falsi, que tiene orden de 
convergencia lineal, el an\'alisis tanto gr\'afico como anal\'itico demostr\'o
que la convergencia era muy lenta. 
Llegado este punto se decidi\'o implementar el \textit{algoritmo de Illinois}, una mejora
de Regula falsi que tiene orden de convergencia $3^{1/3}$. 
La ecuaci\'on de la iteraci\'on del algoritmo de Illinois es muy 
parecida a la de Regula falsi.

En caso de que la ante\'ultima iteraci\'on del m\'etodo sea la que es 
reemplazada por la nueva iteraci\'on anterior la f\'ormula 
es similar a la de Regula Falsi

\begin{center}
$\beta_n = \frac{f(\beta_{n-2})\cdot \beta_{n-1} - f(\beta_{n-1})\cdot\beta_{n-2}}
{f(\beta_{n-2})-f(\beta_{n-1})}$\\
\end{center}

En caso de que la \'ultima iteraci\'on del m\'etodo sea la reemplazada 
por la nueva iteraci\'on en el paso anterior, la f\'ormula es:
\begin{center}
$\beta_n = \frac{\frac{1}{2}f(\beta_{n-2})\cdot \beta_{n-1} - 
f(\beta_{n-1})\cdot\beta_{n-2}}
{\frac{1}{2}f(\beta_{n-2})-f(\beta_{n-1})}$ \\
\end{center}

Lo interesante acerca de la implementaci\'on del algoritmo de Illinois es
que, con un peque\~no y aparentemente trivial cambio en el c\'odigo se consigue una
convergencia de $\frac{1}{\sqrt[3]{n}}$. Illinois ayuda a la convergencia en casos
donde la recta secante es casi vertical, dificultando el acercamiento del
algoritmo al cero buscado.\\

Una explicaci\'on intuitiva de por qu\'e converge Illinois es la siguiente: 

Sean $s$ y $t$ dos iteraciones del m\'etodo. En el pr\'oximo paso nos movemos a 

\begin{center}
$\frac{f(s)*t-f(t)*s}{f(s)-f(t)}$\\
\end{center}

que es un promedio ponderado de $s$ y $t$, ponderando a cada uno por $f$ 
evaluado en el otro punto (cambiando el signo de uno de ellos para que 
tengan el mismo signo). Esto hace que nos movamos a un punto intermedio 
entre $s$ y $t$.

Ahora notemos que si siempre reemplazamos $s$ y $t$ se conserva a lo largo 
de las iteraciones, esto quiere decir que ese punto intermedio tiene siempre el 
mismo signo que $s$, y por lo tanto nos hace pensar que $t$ es muy cercano al 
cero de $f$. Es por esto que, en ese caso, $f(s)$ lo ponderamos por un medio, 
ya que probablemente, al estar $s$ m\'as lejos que $t$ del cero esto nos 
acerque m\'as r\'apido al cero. Esta idea s\'olo ayuda a hacer el 
m\'etodo m\'as r\'apido pero no modifica el hecho de que el m\'etodo converge.

\subsection{Criterios de Parada}	

Los criterios de parada utilizados en ambos casos son equivalentes.

Para ambos m\'etodos se fija $\epsilon$ en $10^{-4}$ y se corre el m\'etodo 
hasta que en dos iteraciones consecutivas el resultado difiera en menos de 
$\epsilon$ o bien hasta llegar a un n\'umero de iteraciones m\'aximas, 
que por defecto se fija en 10 iteraciones, valor elegido luego de la 
experimentaci\'on de diferentes valores del par\'ametro por su consistencia.

La elecci\'on del $\epsilon$ no es caprichosa, sino que se debe a que es la 
potencia de 10 m\'as chica que podemos elegir de modo tal que con la m\'inima 
presici\'on con la que iteramos (15 bits) pueda ser posible cortar por esta 
condici\'on y no corte siempre por cantidad de iteraciones, ya que 
$10^{-5}$ es menos que la diferencia entre dos n\'umeros representables 
consecutivos en el orden de los n\'umeros con los que trabajamos en el TP.
