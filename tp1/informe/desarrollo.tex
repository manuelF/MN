\section{Desarrollo}

El desarrollo de los experimentos fue realizado en serie. El m\'etodo de Newton, que se cree fundamental tanto por lo 
visto en clase como por lo consultado en libros relativos al an\'alisis num\'erico, fue el primero en ser implementado. 
Este proceso fue r\'apido, no existieron grandes dificultades en la implementaci\'on y los resultados fueron 
satisfactorios desde el comienzo.

La decisi\'on sobre la implementaci\'on del m\'etodo de Regula falsi llega a partir de las dificultades en la 
aproximaci\'on mediante el m\'etodo de la secante, dificultades mencionadas m\'as adelante en la seccio\'on.

Como los m\'etodos fueron implementados de uno a la vez, la explicaci\'on del desarrollo de los experimentos est\'a 
organizada del mismo modo.

\subsection{M\'etodo de Newton}

El primer m\'etodo elegido para obtener un despeje de $\beta$ fue el de Newton. La motivaci\'on para implementar este 
m\'etodo como primera opci\'on est\'a relacionada con la importancia que se le da al m\'etodo tanto en los textos 
consultados como en clase.

A priori, la dificultad al emplear este m\'etodo est\'a ligada al c\'alculo de la derivada primera de la funci\'on a la 
que se le busca el cero en cada iteraci\'on. Si bien se puede decir que la iteraci\'on de Newton es la m\'as compleja 
de implementar, la divisi\'on del problema en subproblemas m\'as pequen\~os utilizando funciones auxiliares fue clave 
para una resoluci\'on que se refleja en un c\'odigo limpio y elegante.

Previo a la implementaci\'on del m\'etodo se realizaron los c\'alculos matem\'aticos correspondientes a la funci\'on 
analizada. A continuaci\'on se encuentran los pasos que justifican la utilizaci\'on de la f\'ormula.

Utilizando el despeje sugerido en clase:\\

$\frac{M(2\beta)}{M^2(\beta)}=1 + \beta(R(\beta)-R(0))$

De esta funci\'on se deriva la siguiente ecuaci\'on: \\

$0 = \frac{\beta}{n}\sum{i=1}{n}\log x_i - \log \sum{i=1}{n}x_i^{\beta} + \log(n\lambda)-\psi(\lambda)$

Luego, el m\'etodo general de Newton empleado se comporta del siguiente modo: \\

$\beta_{n+1} = \beta_{n} - \frac{f(\beta_{n})}{f'(\beta_{n})}$

Habiendo obtenido la ecuaci\'on correspondiente al m\'etodo de Newton para esta funcio\'on se procedi\'o a la 
implementaci\'on del problema en el lenguaje de programaci\'on C++.

Los par\'ametros de entrada elegidos para el programa que calcula la aproximaci\'on de $\beta$ son los siguientes:

\begin{itemize}
  \item \textbf{Precisi\'on}: Cantidad de d\'igitos utilizados en el c\'alculo. La variaci\'on de este par\'ametro 
conlleva una variaci\'on en la precisi\'on del resultado del c\'alculo y en el tiempo de ejecuci\'on del programa. El 
m\'aximo valor es la que usamos como control y corresponde a 52 d\'igitos de precisi\'on.

  \item \textbf{Iteraciones m\'aximas}: M\'axima cantidad de iteraciones de Newton. Este criterio de parada previene 
ciclos infinitos ya sea por divergencia del m\'etodo como por problemas de precisi\'on en c\'alculos de punto flotante. 
Por defecto la m\'axima cantidad es de 10 iteraciones.

  \item \textbf{Rango de $\beta$}: Control de la tolerancia. Por defecto asumimos que $\beta$ se encuentra entre 0.0 y 
0.2 pero estos valores se pueden ajustar para modificar la tolerancia del m\'etodo.
\end{itemize}

Esta distribuci\'on de los par\'ametros de entrada es clave para el testeo sistem\'atico y la comparativa entre 
corridas del mismo sistema e interm\'etodo a la hora de comparar con el de Regula falsi.

Experimentando la variaci\'on en la precisi\'on del algoritmo se destaca una diferencia significativa tomando valores 
entre los 15 y 25 d\'igitos de precisi\'on. Tomando valores mayores a 25 d\'igitos la precisi\'on del algoritmo no 
aumenta significativamente, alcanzando valores similares a los de la precisi\'on m\'axima impuesta de 52 d\'igitos. 
Esta observaci\'on se interpreta, a primera vista, como una aceleraci\'on significativa del error hacia valores 
  cercanos al cero.

\subsection{M\'etodo de la Secante}

El segundo m\'etodo elegido fue el m\'etodo de la Secante, ya que es muy parecido al m\'etodo de Newton y tiene orden 
de convergencia superlineal. R\'apidamente qued\'o descartado debido a problemas relacionados al c\'alculo de 
cocientes, en donde el denominador se acerca a cero a un nivel pr\'acticamente indistinguible al nivel de 
representaci\'on de la computadora.

\subsection{M\'etodo de Regula falsi}

Habiendo implementado el m\'etodo de la Secante, y para subsanar el problema de la divisi\'on por cero, se 
procedi\'o a implementar el m\'etodo de Regula Falsi, tambi\'en conocido como Regla Falsa.

De la misma forma que con el m\'etodo de Newton, partiendo de la ecuaci\'on:

$\frac{M(2\beta)}{M^2(\beta)}=1 + \beta(R(\beta)-R(0))$

para despejar $\beta$ y luego utilizando las mismas f\'ormulas que en el caso anterior para despejar $\lambda$ y 
$\sigma$.

La formula resultante para el m\'etodo de Regula Falsi es:

$\beta_n = \beta_{n-1} - \frac{f(\beta_{n-1}) (\beta_{n-1}-\beta_{n-2})}{f(\beta_{n-1}) - f(\beta_{n-2})}$

Se tom\'o la decisi\'on de elegir $f(\beta_{n-1})$ y $f(\beta_n)$ o $f(\beta_{n-2})$ y $f(\beta_n)$ seg\'un el 
signo de $f(\beta_n)$ de modo tal de que los dos n\'umeros que sean tenidos en cuenta tengan distinto signo al 
aplicarle $f$.

\subsection{Criterios de Parada}

Los criterios de parada utilizados en ambos casos son equivalentes.

Para el m\'etodo de Newton se fija $\epsilon$ en $10^{-9}$ y se corre el m\'etodo hasta que en dos iteraciones 
consecutivas el resultado difiera en menos de $\epsilon$ o bien hasta llegar a un n\'umero de iteraciones m\'aximas, 
que por defecto se fija en 10 iteraciones, valor elegido luego de la experimentaci\'on de diferentes valores del 
par\'ametro por su consistencia.

Para el m\'etodo de Regula Falsi se utiliza el mismo m\'etodo para comparar el $\beta$ de una iteraci\'on y el 
$\beta$ anterior, el cu\'al se toma en cuenta para la siguiente iteraci\'on.
