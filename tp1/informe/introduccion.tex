\section{Introducci\'on te\'orica}

El siguiente trabajo se centra en el estudio comparativo de m\'etodos
num\'ericos para la b\'usqueda de ceros de funciones. 

La funci\'on a estudiar es la Digamma Generalizada (D$\Gamma$G). 
Esta funci\'on es una distribuci\'on que aparece frencuentemente en la f\'isica,
ya que modela la propagaci\'on de se\~nales mediante la serie harm\'onica.

Una distribuci\'on de probabilidad de una variable es una funci\'on que asigna
a cada posible valor que puede tomar la variable, la probabilidad de que esta
tome ese valor. Esta tiene como dominio a los numeros reales mayores a
0 y los mapea a valores reales en el intervalo [0,1].

La funci\'on Digamma Generalizada refiere a la siguiente expresi\'on:

\begin{center}
\begin{math}
f_{\ominus}(x) = \frac{\beta x^{\beta \lambda -1}}{\sigma ^{\beta \lambda}
\Gamma(\lambda)}\exp {-(\frac{x}{\sigma})^{\beta}} \hspace{10 mm} con x \in \mathbb{R}_{>0}
\end{math}
\end{center}

Normalmente, previo a la realizaci\'on de estudios experimentals, no suele
existir un conocimiento completo sobre el conjunto de variables que modelan
el sistema. Ciertos par\'ametros est\'an controlados, pero muchos quedan libres. 
La motivacion de este trabajo es poder estimar, dado un conjunto acotado de datos, 
las variables que rigen la distribuci\'on D$\Gamma$G. 

Estas variables son $\sigma$, $\beta$ y $\lambda$. Los m\'etodos empleados para 
su estimaci\'on ser\'an detallados mas adelante. Los 3 par\'ametros corresponden
a n\'umeros reales mayores a 0.

\begin{itemize}
  \begin{item} $\sigma$ Es un par\'ametro de escala, es decir, a mayor valor del
par\'ametro m\'as dispersa ser\'a la funci\'on.\end{item}
  \begin{item} $\beta$ Es el par\'ametro que se quiere estimar. Este es un
par\'ametro de forma. Esto significa que las modificaciones en $\beta$ modifican
las formas de la funci\'on pero no la corren respecto a los ejes. \end{item}
  \begin{item} $\lambda$ Es otro par\'ametro de forma que incide, junto con
$\beta$ en la forma de la funci\'on.\end{item}
\end{itemize}

El proposito de este trabajo es comparar las t\'ecnicas de obtenci\'on de ra\'ices 
de funci\'ones y c\'alculo de errores vistas en la materia, 
con una aplicaci\'on pr\'actica, concreta y sencilla de entender.

La base de la experiencia consiste en la obtenci\'on de la variable $\beta$ en
primer t\'ermino. Esta, como ya fue despejada en clase, es una funci\'on
de una variable, ya que los datos $x_i$ son fijos (par\'ametros). 
Esto marca la pauta de que el valor de $\beta$ debe ser despejado
num\'ericamente. 

M\'as adelante se verifican las hip\'otesis de Newton, para as\'i poder aplicar
el m\'etodo correspondiente con certeza de que se cumplen las precondiciones.

El m\'etodo de Newton es comparado con el de Regula Falsi (Regla falsa). Esto
permite desarrollar un an\'alisis comparativo basadao en experiencias que,
modificando una variable y dejando constantes las otras, permite deducir
aspectos del comportamiento de ambos m\'etodos, sus fortalezas y debilidades.

El m\'etodo de Newton-Raphson para la b\'usqueda de ceros de funciones parte de
una primera estimaci\'on $x_{0}$ de un cero de la funci\'on en cuestion. Luego
las sucesivas iteraciones se obtienen a partir de la f\'ormula

\begin{center}
$x_{n+1} = x{n} - \frac{f(x_{n})}{f'(x_{n})}$
\end{center}

En cada iteraci\'on de Newton la aproximaci\'on del m\'etodo al cero deseado se
da por la intersecci\'on de la recta tangente del punto elegido en el paso 
anterior con el eje de abscisas. Es por eso que se requiere del c\'alculo de la
derivada de la funci\'on la cual se est\'a analizando. El c\'alculo de la
derivada es el procedimiento m\'as complejo dentro del desarrollo del m\'etodo y
motivo suficiente para no utilizar el m\'etodo en muchos casos.

Por otra parte, el m\'etodo de Regula falsi (Regla falsa) que combina otros 2
como lo son el de Bisecci\'on y el de la Secante, se aprovecha del teorema de
Bolzano para encontrar ceros de funciones.

Este m\'etodo parte de un intervalo [a\_{0},b\_{0}] donde $f(a) y f(b)$ tienen
distinto signo (siendo $f$ la funci\'on a analizar). En cada paso se calcula un
nuevo punto que acorta el intervalo de b\'usqueda del siguiente modo

\begin{center}
$c\_{k} = \frac{f(b\_{k})a\_{k} - f(a\_{k})b\_{k}}{f(b\_{k}) - f(a\_{k})}$
\end{center}

En cada iteraci\'on $c\_{k}$ representa la interesecci\'on de la recta $(a,
f(a\_{k})) (b, f(b\_{k}))$ con el eje horizontal.

Este m\'etodo no resulta ser muy conveniente para funciones con curvas suaves,
sin embargo implementando una mejor conocida como el algoritmo de Illinois, que
ser\'a detallado m\'as adelante en el trabajo, el m\'etodo mejora
considerablemente su convergencia.

La r\'apida y satisfactoria implementaci\'on del m\'eotodo de Newton permite,
durante la realizaci\'on de las experiencias, del ajuste del m\'etodo de Regula
falsi, basado en los resultados obtenidos en experimentos previos.
