\section{Introducci\'on te\'orica}

El siguiente trabajo es un estudio sobre la funci\'on Digamma Generalizada (D$\Gamma$G). Esta funci\'on es una
distribuci\'on que aparece frencuentemente en la f\'isica, ya que modela la propagaci\'on de se\~nales mediante la
serie harm\'onica.

Normalmente, como en cualquier estudio experimental, no suele haber conocimiento completo sobre todas las variables
que modelan el sistema. Hay parametros que uno controla, pero muchos quedan libres. La motivacion de este trabajo,
es poder estimar, dados ciertos datos, las variables que rigen esta distribuci\'on. Estas son $\sigma$, $\beta$ y $\lambda$
y los metodos que ser\'an empleados para su estimaci\'on ser\'an detallados mas adelante.

El proposito de este trabajo es unir las t\'ecnicas de obtencion de raices de funci\'ones y calculo de errores
que vimos en la materia, con una aplicaci\'on pr\'actica, concreta y sencilla de entender.

La base del trabajo consiste en la obtenci\'on de $\beta$ primeramente. Esta, como ya fue despejada en clase, es una funci\'on
de una variable, ya que los datos $x_i$ son fijos (par\'ametros). Esto marca la pauta de que debemos despejar num\'ericamente el
valor de $\beta$. M\'as adelante verificamos las hipotesis de Newton, asi vemos que estamos en las precondiciones,
y podemos aplicar la estimaci\'on de $\beta$.
