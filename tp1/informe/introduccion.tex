\section{Introducci\'on te\'orica}

El siguiente trabajo se centra en el estudio comparativo de m\'etodos
num\'ericos para la b\'usqueda de ceros de funciones. 

La funci\'on a estudiar es la Digamma Generalizada (D$\Gamma$G). 
Esta funci\'on es una distribuci\'on que aparece frencuentemente en la f\'isica,
ya que modela la propagaci\'on de se\~nales mediante la serie harm\'onica.

Normalmente, previo a la realizaci\'on de estudios experimentals, no suele
existir un conocimiento completo sobre el conjunto de variables que modelan
el sistema. Ciertos par\'ametros est\'an controlados, pero muchos quedan libres. 
La motivacion de este trabajo es poder estimar, dado un conjunto acotado de datos, 
las variables que rigen la distribuci\'on D$\Gamma$G. 
Estas variables son $\sigma$, $\beta$ y $\lambda$. Los m\'etodos empleados para 
su estimaci\'on ser\'an detallados mas adelante.

El proposito de este trabajo es comparar las t\'ecnicas de obtenci\'on de ra\'ices 
de funci\'ones y c\'alculo de errores vistas en la materia, 
con una aplicaci\'on pr\'actica, concreta y sencilla de entender.

La base de la experiencia consiste en la obtenci\'on de la variable $\beta$ en
primer t\'ermino. Esta, como ya fue despejada en clase, es una funci\'on
de una variable, ya que los datos $x_i$ son fijos (par\'ametros). 
Esto marca la pauta de que el valor de $\beta$ debe ser despejado
num\'ericamente. 

M\'as adelante se verifican las hip\'otesis de Newton, para as\'i poder aplicar
el m\'etodo correspondiente con certeza de que se cumplen las precondiciones.

El m\'etodo de Newton es comparado con el de Regula Falsi (Regla falsa). Esto
permite desarrollar un an\'alisis comparativo basadao en experiencias que,
modificando una variable y dejando constantes las otras, permite deducir
aspectos del comportamiento de ambos m\'etodos, sus fortalezas y debilidades.

La r\'apida y satisfactoria implementaci\'on del m\'eotodo de Newton permite,
durante la realizaci\'on de las experiencias, del ajuste del m\'etodo de Regula
falsi, basado en los resultados obtenidos en experimentos previos.
