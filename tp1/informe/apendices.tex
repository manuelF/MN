\section{Apendice A, Codigo del TP}
\begin{lstlisting}

#include <algorithm>
#include <cmath>
#include <cstdio>
#include <iomanip>
#include <iostream>
#include <queue>
#include <set>
#include <sstream>
#include <string>
#include <vector>

#include "TFloat.h"
#include <math.h>

#ifdef __gnu_linux__
#include "timetools.h"
#endif
using namespace std;

vector < TFloat > valores;

int n, t;
double epsilon = 1.e-4;
int maximoIteraciones = 10;
TFloat pot(TFloat base, TFloat exp)
{
    double result = pow(base.dbl(), exp.dbl());
    return TFloat(result, t);
}

TFloat log(TFloat d)
{
    double result = log(d.dbl());
    return TFloat(result, t);
}

TFloat M(TFloat s)
{
    TFloat res = TFloat(0., t);
    for (int i = 0; i < n; i++)
	res = res + pot(valores[i], s);
    res = res / TFloat(n, t);
    return res;
}

TFloat MSombrero(TFloat s)
{
    TFloat res = TFloat(0., t);
    for (int i = 0; i < n; i++)
	res = res + pot(valores[i], s) * log(valores[i]);
    res = res / TFloat(n, t);
    return res;
}

TFloat MPrima(TFloat s)
{
    // / Resulta ser que Mprima es igual a Msombrero
    return MSombrero(s);
}

TFloat MSombreroPrima(TFloat s)
{
    TFloat res = TFloat(0., t);
    for (int i = 0; i < n; i++)
	res = res + pot(valores[i], s) * log(valores[i]) * log(valores[i]);
    res = res / TFloat(n, t);
    return res;
}

TFloat R(TFloat s)
{
    return MSombrero(s) / M(s);
}

TFloat RPrima(TFloat s)
{
    return (MSombreroPrima(s) * M(s) -
	    MSombrero(s) * MPrima(s)) / (M(s) * M(s));
}

TFloat Lambda(TFloat beta)
{
    vector < TFloat > logvals(n);
    vector < TFloat > xbeta(n);
    TFloat sumalogvals(0.0, t);
    TFloat sumaxbeta(0.0, t);
    TFloat sumalogxbeta(0.0, t);
    for (int i = 0; i < n; i++)
      {
	  logvals[i] = log(valores[i]);
	  xbeta[i] = pot(valores[i], beta);
      }
    for (int i = 0; i < n; i++)
      {
	  sumalogvals = sumalogvals + logvals[i];
	  sumaxbeta = sumaxbeta + xbeta[i];
	  sumalogxbeta = sumalogxbeta + logvals[i] * xbeta[i];
      }
    TFloat lambda =
	pot(beta *
	    (((sumalogxbeta) / (sumaxbeta)) -
	     ((sumalogvals) / (TFloat(n, t)))),
	    TFloat(-1, t));
    return lambda;

}

TFloat Sigma(TFloat beta, TFloat lambda)
{
    TFloat sigma;
    TFloat xbetasuma(0.0, t);

    for (int i = 0; i < n; i++)
      {
	  xbetasuma = xbetasuma + pot(valores[i], beta);
      }
    sigma =
	pot((xbetasuma / (TFloat(n, t) * lambda)),
	    (TFloat(1.0, t) / beta));

    return sigma;
}

TFloat f(TFloat beta)
{
    TFloat Mbeta = M(beta);
    TFloat res =
	M(beta * 2) / (Mbeta * Mbeta) - beta * (R(beta) - R(0)) - 1.;
    return res;
}


TFloat fprima(TFloat beta)
{

    /** Mprima(beta*2)*2 = Derivada de M(beta*2)
     *  M(beta)*Mprima(beta)*2 = Derivada de M(beta)*M(beta) o M(beta)^2
     *  Derivada de M(beta*2)/M(beta)^2 =
     *      (Mprima(beta*2)*M(beta)*M(beta)*2 - 
		M(beta*2)*M(beta)*Mprima(beta)*2)/(M(beta)*M(beta)*M(beta)*M(beta))
     *  Derivada de beta * (R(beta) - R(0)) =
     *      beta * (Rprima(beta)) + R(beta)-R(0) 
     */

    TFloat Mbeta = M(beta);
    TFloat Mbeta2 = Mbeta * Mbeta;
    // Aca podemos simplificar varios terminos sacando factores comunes
    return (TFloat(2, t)) * (MPrima(beta * 2) * Mbeta -
			     M(beta * 2) * MPrima(beta)) / (Mbeta *
							    Mbeta2) -
	(beta * RPrima(beta) + R(beta) - R(0));
}


TFloat newton(TFloat beta, TFloat beta2, int &iteraciones)
{
    iteraciones = 0;

    while (iteraciones < maximoIteraciones
	   && abs(beta.dbl() - beta2.dbl()) > epsilon)
      {
	  iteraciones++;
	  beta2 = beta;
	  beta = beta - f(beta) / fprima(beta);
      }
    return beta;
}

TFloat illinois(TFloat beta, TFloat beta2, int &iteraciones)
{
    iteraciones = 0;
    TFloat beta3;
    TFloat fbeta = f(beta);
    TFloat fbeta2 = f(beta2);
    for (int i = 0; i < 10; i++)
	if (fbeta.dbl() > 0)
	  {
	      beta = beta / 2.;
	      fbeta = f(beta);
	  }
    for (int i = 0; i < 10; i++)
	if (fbeta2.dbl() < 0)
	  {
	      beta2 = beta2 * 2;
	      fbeta2 = f(beta2);
	  }
    if (fbeta.dbl() * fbeta2.dbl() > 0)
      {
	  cerr << setprecision(15);
	  cerr << "==================fruta follows===================\n";
	  cerr << beta.dbl() << ": " << fbeta.dbl() << "   " << beta2.
	      dbl() << ": " << fbeta2.dbl() << endl;
      }
    int side = -1;
    while (iteraciones < maximoIteraciones
	   && fabs(beta.dbl() - beta2.dbl()) > epsilon)
      {
	  iteraciones++;
	  beta3 = (fbeta * beta2 - fbeta2 * beta) / (fbeta - fbeta2);
	  TFloat fbeta3 = f(beta3);
	  if (fbeta2.dbl() * fbeta3.dbl() > 0)
	    {
		beta2 = beta3;
		fbeta2 = fbeta3;
		if (side == -1)
		    fbeta = fbeta / 2;
		side = -1;
	    }
	  else if (fbeta.dbl() * fbeta3.dbl() > 0)
	    {
		beta = beta3;
		fbeta = fbeta3;
		if (side == 1)
		    fbeta2 = fbeta2 / 2;
		side = 1;
	    }
      }
    cerr << abs(beta2.dbl() - beta.dbl()) << " " << iteraciones << endl;
    return beta3;
}

void uso()
{
    cout <<
	"\"./MN\" precision = 52, iteraciones maximas = 10, Beta entre 0.0 y 10.0"
	<< endl;
    cout <<
	"\"./MN <t>\" precision = t, iteraciones maximas = 10, Beta entre 0.0 y 10.0"
	<< endl;
    cout <<
	"\"./MN <t> <n> <b1> <b2> <m> \" precision = t, iteraciones maximas= n, Beta entre b1 y b2, <m> metodo (0 Newton - 1 RegulaFalsi)"
	<< endl;
    cout <<
	"\"./MN <t> <n> <b1> <b2> <m> <e> \" precision = t, iteraciones maximas= n, Beta entre b1 y b2, <m> metodo (0 Newton - 1 RegulaFalsi), epsilon =<e>"
	<< endl;
}

int main(int argc, char *argv[])
{
    TFloat beta = TFloat(10., 52);
    TFloat beta2 = TFloat(1., 52);
    int metodo = 0;		// 0 Newton 1 RegulaFalsi
    switch (argc)
      {
	case 1:
	    t = 52;
	    break;
	case 2:
	    t = atoi(argv[1]);
	    break;
	case 7:
	    epsilon = atof(argv[6]);
	case 6:
	    metodo = atoi(argv[5]);
	case 5:
	    t = atoi(argv[1]);
	    maximoIteraciones = atoi(argv[2]);
	    beta = TFloat(atof(argv[3]), t);
	    beta2 = TFloat(atof(argv[4]), t);
	    break;
	default:
	    uso();
	    return 1;
	    break;
      }
    cin >> n;
    valores.resize(n);
    double db;
    for (int i = 0; i < n; i++)
      {
	  cin >> db;

	  valores[i] = TFloat(db, t);
      }

    int iteraciones = 0;
    TFloat outBeta;
    TFloat outLambda;
    TFloat outSigma;

    cout << setprecision(15);

#ifdef __gnu_linux__
    timespec startTime;
    timespec endTime;

    clock_gettime(CLOCK_PROCESS_CPUTIME_ID, &startTime);
#endif

    if (metodo == 0)
      {
	  outBeta = newton(beta, beta2, iteraciones);
      }
    else if (metodo == 1)
      {
	  outBeta = illinois(beta, beta2, iteraciones);
      }

    outLambda = Lambda(outBeta);
    outSigma = Sigma(outBeta, outLambda);

#ifdef __gnu_linux__
    clock_gettime(CLOCK_PROCESS_CPUTIME_ID, &endTime);
    const timespec delta = diff(startTime, endTime);
#endif
    cout << outBeta.dbl() << " " << outLambda.dbl() << " " << outSigma.
	dbl() << " " << iteraciones;
#ifdef __gnu_linux__
    cout << " " << (delta.tv_sec * 1000 * 1000 * 1000 +
		    delta.tv_nsec) / (1000 * 1000);
#endif
    cout << endl;
    return 0;
}
\end{lstlisting}
