\section{Resultados}

Para po

Lo primero que hicimos fue plotear las distribuciones con datos conocidos y ver como se ajustaban a los
histogramas, sin aproximarlas. Esto nos sirvio para entender que forma debieran dar, aproximadamente, las curvas
ajustadas estimadas luego.

%Caso 1: Caso 2
\begin{figure} [H]
\begin {center}
\includegraphics[width=300pt]{plots/Caso1.png}
\includegraphics[width=300pt]{plots/Caso2.png}
\end {center}
\caption{Traza exacta de la funcion digamma usando los parametros conocidos}
\label{fig:FitCaso3Newton}
\end{figure}

Luego procuramos encontrar valores de cantidad de digitos decimales binarios interesantes. Notamos que el m\'etodo de Newton
para valores muy bajos de cantidad de digitos no convergia ($t < 12$ en promedio); Por lo que nos concentramos en valores
ligeramente mas grandes, cosa de poder contemplar cuan cerca esta la distribuci\'on estimada de la real.

Los valores que usamos de precision son 13, 16, 19, 30. Estos fueron elegidos porque convergian, y porque consideramos que estan
suficientemente separados entre s\'i como para ser significativos. Creamos un script para automatizar la generaci\'on de estos gr\'aficos,
para poder hacer los experimentos repetibles y consistentes. Estos invocan al m\'etodo con algunos parametros, obtienen los
valores estimados, la cantidad de iteraciones que realizo el algoritmo elegido y la cantidad de milisegundos que tard\'o este proceso.


\begin{figure} [H]

\includegraphics[width=200pt]{plots/Newton-13-caso3.png}
\includegraphics[width=200pt]{plots/Newton-16-caso3.png}
\includegraphics[width=200pt]{plots/Newton-19-caso3.png}
\includegraphics[width=200pt]{plots/Newton-30-caso3.png}

\caption{Ajuste usando distintas precisiones (13, 16, 19, 30) con Newton al caso 3, X1}
\label{fig:FitCaso3Newton}
\end{figure}

\begin{figure} [H]

\includegraphics[width=200pt]{plots/Newton-13-caso4.png}
\includegraphics[width=200pt]{plots/Newton-16-caso4.png}
\includegraphics[width=200pt]{plots/Newton-19-caso4.png}
\includegraphics[width=200pt]{plots/Newton-30-caso4.png}

\caption{Ajuste usando distintas precisiones (13, 16, 19, 30) con Newton al caso 4, X2}
\label{fig:FitCaso4Newton}
\end{figure}


Tambien trazamos, para los valores conocidos, como es la aproximacion de las distintas variables, $\beta, \lambda, \sigma$.

\begin{figure} [H]

\includegraphics[width=200pt]{plots/Beta-Caso1.png}
\includegraphics[width=200pt]{plots/Lambda-Caso1.png}
\includegraphics[width=200pt]{plots/Sigma-Caso1.png}

\caption{Aproximacion a $\beta, \lambda, \sigma$ en el Caso 1}
\label{fig:AproxCaso1}
\end{figure}

\begin{figure} [H]

\includegraphics[width=200pt]{plots/Beta-Caso2.png}
\includegraphics[width=200pt]{plots/Lambda-Caso2.png}
\includegraphics[width=200pt]{plots/Sigma-Caso2.png}

\caption{Aproximacion a $\beta, \lambda, \sigma$ en el Caso 2}
\label{fig:AproxCaso1}
\end{figure}