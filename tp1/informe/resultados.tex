\section{Resultados}

%TODO: Primer p\'arrafo introductorio con la posta de lo que nos dio!

La primera aproximaci\'on a los resultados fue dada por el ploteo de las distribuciones con datos conocidos, permitiendo observar como se ajustan los histogramas sin llevar a cabo ninguna observaci\'on. Este experimento sirvi\'o para dar una idea de las formas que debieran seguir las curvas ajustadas en los subsiguientes experimentos.

\subsection{M\'etodo de Newton}

\begin{figure} [H]
\begin {center}
\includegraphics[width=300pt]{plots/Caso1Hist.png}
\includegraphics[width=300pt]{plots/Caso2Hist.png}
\end {center}
\caption{Traza exacta de la funci\'on digamma usando los par\'ametros conocidos}
\label{fig:FitCaso3Newton}
\end{figure}

Una vez determinada la forma ''deseada'' de las curvas se procedi\'o a encontrar valores de cantidad de d\'igitos decimales relevantes para los experimentos. Notamos que el m\'etodo de Newton para valores muy bajos de cantidad de digitos no convergia ($t < 12$ en promedio); Por lo que nos concentramos en valores
ligeramente mas grandes, cosa de poder contemplar cuan cerca esta la distribuci\'on estimada de la real.

Los valores que usamos de precision son 15, 17, 22, 27. Estos fueron elegidos porque convergian, y porque consideramos que estan
suficientemente separados entre s\'i como para ser significativos. Creamos un script para automatizar la generaci\'on de estos gr\'aficos,
para poder hacer los experimentos repetibles y consistentes. Estos invocan al m\'etodo con algunos parametros, obtienen los
valores estimados, la cantidad de iteraciones que realizo el algoritmo elegido y la cantidad de milisegundos que tard\'o este proceso.


\begin{figure} [H]
$\begin{array}{cc}
\includegraphics[width=200pt]{plots/Newton-15-caso3.png} &
\includegraphics[width=200pt]{plots/Newton-17-caso3.png} \\
\includegraphics[width=200pt]{plots/Newton-22-caso3.png} &
\includegraphics[width=200pt]{plots/Newton-27-caso3.png}
\end{array}$
\caption{Ajuste usando distintas precisiones (15, 17, 22, 27) con Newton al caso 3, X1}
\label{fig:FitCaso3Newton}
\end{figure}

\begin{figure} [H]
$\begin{array}{cc}
\includegraphics[width=200pt]{plots/Newton-15-caso4.png} &
\includegraphics[width=200pt]{plots/Newton-17-caso4.png} \\
\includegraphics[width=200pt]{plots/Newton-22-caso4.png} &
\includegraphics[width=200pt]{plots/Newton-27-caso4.png}
\end{array}$
\caption{Ajuste usando distintas precisiones (15, 17, 22, 27) con Newton al caso 4, X2}
\label{fig:FitCaso4Newton}
\end{figure}


Tambien trazamos, para los valores conocidos, como es la aproximacion de las distintas variables, $\beta, \lambda, \sigma$ usando el m\'etodo de Newton.
Aca ploteamos todas las precisiones entre 10 y 27 y fuimos analizando como iban variando $\beta$ (y $\lambda$ y $\sigma$ en consecuencia).

Como se puede observar en los casos 1 y 2, el valor de $\sigma$ se mantiene estable no siendo la precisi\'on de los c\'alculos un limitante y convergiendo muy velozmente. Sin embargo las aproximaciones tanto de $\beta$ como de $\lambda$ comienzan a estabilizarse en la convergencia a partir de la utilizaci\'on de 17 d\'igitos de precisi\'on. Es evidente que el rango de d\'igitos de precisi\'on en estos casos debe mantenerse entre unos 15 y 20 d\'igitos para maximizar la eficiencia del algoritmo.  

\begin{figure}
$\begin{array}{c}
\includegraphics[width=400pt]{plots/Caso1-Newton.png} \\
\includegraphics[width=400pt]{plots/Caso2-Newton.png}
\end{array}$

\caption{Aproximacion a $\beta, \lambda, \sigma$ en los Casos 1 y 2 usando Newton}
\end{figure}

En el Caso 3 vemos una anomal\'ia en los valores de $\beta$ y $\lambda$. En este caso tomar entre 15 y 20 d\'igitos de precisi\'on no es suficiente para obtener valores relevantes. Estos se estabilizan recien a partir de los 22 d\'igitos de precisi\'on.


En los casos 4, 5, 6, 8 y 9 tambi\'en observamos que a partir de 20 d\'igitos de precisi\'on comienza a converger el valor de $\beta$. En el caso 7 $\beta$ comienza a converger a partir de 22 d\'igitos de precisi\'on.

\begin{figure}
$\begin{array}{cc}
\includegraphics[width=220pt]{plots/Caso3-Newton.png} &
\includegraphics[width=220pt]{plots/Caso4-Newton.png} \\
\includegraphics[width=220pt]{plots/Caso5-Newton.png} &
\includegraphics[width=220pt]{plots/Caso6-Newton.png} \\
\includegraphics[width=220pt]{plots/Caso7-Newton.png} &
\includegraphics[width=220pt]{plots/Caso8-Newton.png} \\
\includegraphics[width=220pt]{plots/Caso9-Newton.png} &\\
\end{array}$

\caption{Aproximacion a $\beta, \lambda, \sigma$ en los Casos 3 al 9 usando Newton}
\label{fig:AproxCasosNewton}
\end{figure}

%%%%%%%%%%%%%%%%%%%%%%%%%%%%%%%%%%%%%%%%%%%%%%%%%%%%%%%%%%%%%%%%%%%%%
\subsection{M\'etodo Regula Falsi - Illinois}


Trazamos los mismos plots para Regula Falsi, en su variante Illinois.

\begin{figure} [H]
$\begin{array}{cc}
\includegraphics[width=200pt]{plots/RegulaFalsi-15-caso3.png} &
\includegraphics[width=200pt]{plots/RegulaFalsi-17-caso3.png} \\
\includegraphics[width=200pt]{plots/RegulaFalsi-22-caso3.png} &
\includegraphics[width=200pt]{plots/RegulaFalsi-27-caso3.png}
\end{array}$
\caption{Ajuste usando distintas precisiones (15, 17, 22, 27) con RegulaFalsi al caso 3, X1}
\label{fig:FitCaso3RegulaFalsi}
\end{figure}

\begin{figure} [H]
$\begin{array}{cc}
\includegraphics[width=200pt]{plots/RegulaFalsi-15-caso4.png} &
\includegraphics[width=200pt]{plots/RegulaFalsi-17-caso4.png} \\
\includegraphics[width=200pt]{plots/RegulaFalsi-22-caso4.png} &
\includegraphics[width=200pt]{plots/RegulaFalsi-27-caso4.png}
\end{array}$
\caption{Ajuste usando distintas precisiones (15, 17, 22, 27) con RegulaFalsi al caso 4, X2}
\label{fig:FitCaso4Newton}
\end{figure}


Tambien trazamos, para los valores conocidos, como es la aproximacion de las distintas variables, $\beta, \lambda, \sigma$ usando el m\'etodo RegulaFalsi.
Aca ploteamos todas las precisiones entre 10 y 27 y fuimos analizando como iban variando $\beta$ (y $\lambda$ y $\sigma$ en consecuencia).

Como se puede observar en los casos 1 y 2, el valor de $\sigma$ se mantiene estable no siendo la 
precisi\'on de los c\'alculos un limitante y convergiendo muy velozmente. 
Sin embargo las aproximaciones tanto de $\beta$ como de $\lambda$ comienzan a estabilizarse en la convergencia 
a partir de la utilizaci\'on de 17 d\'igitos de precisi\'on. Es evidente que el rango de d\'igitos de precisi\'on en 
estos casos debe mantenerse entre unos 15 y 20 d\'igitos para maximizar la eficiencia del algoritmo.  

\begin{figure}
$\begin{array}{c}
\includegraphics[width=400pt]{plots/Caso1-RegulaFalsi.png} \\
\includegraphics[width=400pt]{plots/Caso2-RegulaFalsi.png}
\end{array}$

\caption{Aproximacion a $\beta, \lambda, \sigma$ en los Casos 1 y 2 usando RegulaFalsi}
\end{figure}

En el Caso 3 vemos una anomal\'ia en los valores de $\beta$ y $\lambda$. En este caso tomar entre 15 y 20 d\'igitos de precisi\'on
no es suficiente para obtener valores relevantes. Estos se estabilizan recien a partir de los 22 d\'igitos de precisi\'on.




\begin{figure}
$\begin{array}{cc}
\includegraphics[width=220pt]{plots/Caso3-RegulaFalsi.png} &
\includegraphics[width=220pt]{plots/Caso4-RegulaFalsi.png} \\
\includegraphics[width=220pt]{plots/Caso5-RegulaFalsi.png} &
\includegraphics[width=220pt]{plots/Caso6-RegulaFalsi.png} \\
\includegraphics[width=220pt]{plots/Caso7-RegulaFalsi.png} &
\includegraphics[width=220pt]{plots/Caso8-RegulaFalsi.png} \\
\includegraphics[width=220pt]{plots/Caso9-RegulaFalsi.png} & \\

\end{array}$

\caption{Aproximacion a $\beta, \lambda, \sigma$ en los Casos 3 y al 8 usando RegulaFalsi}
\label{fig:AproxCasosRegulaFalsi}



\end{figure}

%%%%%%%%%%%%%%%%%%%%%%%%%%%%%%%%%%%%%%%%%%%%%%%%%%%%%%%%%

\subsection{$\lambda$ y $\sigma$}

\indent Para calcular $\lambda$ y $\sigma$ utilizamos las f\'ormulas (1) y (2) dadas en el enunciado. Pudimos observar que, al depender estas dos de $\beta$, empiezan a converger a medida que $\beta$ converge.

$$\big[ \beta \big( \frac{\sum_{i=1}^{n}x_i^\beta \log x_i}{ \sum_{i=1}^{n} x_i^\beta} - \frac{\sum_{i=1}^{n} \log x_i}{n} \big)\big]^{-1}$$

$$\sigma = \big( \frac{\sum_{i=1}^{n} x_i^\beta}{n\lambda}\big)^{\frac{1}{\beta}}$$

Para $\lambda$ observamos que al tener el factor $\beta$ multiplicando a la $-1$, esto es como dividir por $\beta$, y este factor es el que m\'as pesa a la hora de la dependencia entre $\beta$ y $\lambda$ ya que al crecer $\beta$ siempre decrece $\lambda$ y al decrecer $\beta$ siempre crece $\lambda$. En algunos casos observamos que una peque\~na variaci\'on en $\beta$ implica una variaci\'on muy grande en $\lambda$.

$\sigma$ en cambio se comporta de manera igual que beta, es decir, o ambos crecen o ambos decrecen. Esto se debe, probablemente, a que hay un factor $\lambda^{\frac{1}{\beta}}$ dividiendo, y dado que $\lambda$ crece mucho cuando $\beta$ decrece y viceversa, esto suma mucho para la variaci\'on de $\sigma$. Adem\'as de esto, la media que se calcula (por ejemplo, el promedio si $\beta = 1$ o la media cuadr\'atica si $\beta = 2$) es una funci\'on creciente en $\beta$ que es igual a la norma $\beta$ del vector de entrada (los $x_i$).
