\section{Resultados}

TODO: Primer p\'arrafo introductorio con la posta de lo que nos dio!

La primera aproximaci\'on a los resultados fue dada por el ploteo de las distribuciones con datos conocidos, permitiendo observar como se ajustan los histogramas sin llevar a cabo ninguna observaci\'on. Este experimento sirvi\'o para dar una idea de las formas que debieran seguir las curvas ajustadas en los subsiguientes experimentos.

%Caso 1: Caso 2
\begin{figure} [H]
\begin {center}
\includegraphics[width=300pt]{plots/Caso1.png}
\includegraphics[width=300pt]{plots/Caso2.png}
\end {center}
\caption{Traza exacta de la funci\'on digamma usando los par\'ametros conocidos}
\label{fig:FitCaso3Newton}
\end{figure}

Una vez determinada la forma \"deseada\" de las curvas se procedi\'o a encontrar valores de cantidad de d\'igitos decimales relevantes para los experimentos. Notamos que el m\'etodo de Newton para valores muy bajos de cantidad de digitos no convergia ($t < 12$ en promedio); Por lo que nos concentramos en valores
ligeramente mas grandes, cosa de poder contemplar cuan cerca esta la distribuci\'on estimada de la real.

Los valores que usamos de precision son 13, 16, 19, 30. Estos fueron elegidos porque convergian, y porque consideramos que estan
suficientemente separados entre s\'i como para ser significativos. Creamos un script para automatizar la generaci\'on de estos gr\'aficos,
para poder hacer los experimentos repetibles y consistentes. Estos invocan al m\'etodo con algunos parametros, obtienen los
valores estimados, la cantidad de iteraciones que realizo el algoritmo elegido y la cantidad de milisegundos que tard\'o este proceso.


\begin{figure} [H]

\includegraphics[width=200pt]{plots/Newton-13-caso3.png}
\includegraphics[width=200pt]{plots/Newton-16-caso3.png}
\includegraphics[width=200pt]{plots/Newton-19-caso3.png}
\includegraphics[width=200pt]{plots/Newton-30-caso3.png}

\caption{Ajuste usando distintas precisiones (13, 16, 19, 30) con Newton al caso 3, X1}
\label{fig:FitCaso3Newton}
\end{figure}

\begin{figure} [H]

\includegraphics[width=200pt]{plots/Newton-13-caso4.png}
\includegraphics[width=200pt]{plots/Newton-16-caso4.png}
\includegraphics[width=200pt]{plots/Newton-19-caso4.png}
\includegraphics[width=200pt]{plots/Newton-30-caso4.png}

\caption{Ajuste usando distintas precisiones (13, 16, 19, 30) con Newton al caso 4, X2}
\label{fig:FitCaso4Newton}
\end{figure}
