\section{Resultados}

Para po

Lo primero que hicimos fue plotear las distribuciones con datos conocidos y ver como se ajustaban a los
histogramas, sin aproximarlas.

%Caso 1: Caso 2

Luego procuramos encontrar valores de cantidad de digitos decimales binarios interesantes. Notamos que el m\'etodo de Newton
para valores muy bajos de cantidad de digitos no convergia ($t < 12$ en promedio); Por lo que nos concentramos en valores
ligeramente mas grandes, cosa de poder contemplar cuan cerca esta la distribuci\'on estimada de la real.

Los valores que usamos de precision son 15, 20, 25, 40. Estos fueron elegidos porque convergian, y porque consideramos que estan
suficientemente separados entre s\'i como para ser significativos. Creamos un script para automatizar la generaci\'on de estos gr\'aficos,
para poder hacer los experimentos repetibles y consistentes. Estos invocan al programa con algunos parametros, obtienen los
valores estimados, la cantidad de iteraciones que realizo el algoritmo elegido y la cantidad de milisegundos que tard\'o este proceso.


